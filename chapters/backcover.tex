%!TEX root=../main.tex
\pagestyle{empty}
\newgeometry{top=1cm,bottom=1cm,left=1cm,right=1cm}

\begin{flushleft}
\includegraphics[width=300pt]{SMEMAG.pdf}

\vspace{20pt}

\begin{mdframed}
\begin{otherlanguage}{french}
\textbf{Titre :} Traitement de la localisation des déformations par champ d'endommagement pour les structures en béton armé soumises à un chargement dynamique violent

\textbf{Mots clés :} rupture dynamique fragile, modèles d'endommagement à gradient, approche variationnelle, implémentation numérique

\textbf{Résumé :} Une bonne tenue mécanique des structures du génie civil en béton armé sous chargements dynamiques sévères est primordiale pour la sécurité et nécessite une évaluation précise de leur comportement en présence de propagation de fissures dynamiques. Dans ce travail, on se focalise sur la modélisation constitutive du béton assimilé à un matériau élastique-fragile endommageable. La localisation des déformations sera régie par un modèle d'endommagement à gradient où un champ scalaire réalise une description régularisée des phénomènes de rupture dynamique. La contribution de cette étude est à la fois théorique et numérique. Elle est organisée à l'aide de quatre sujets thématiques : extension dynamique, lien avec des approches champs de phase, meilleure compréhension des approaches d'endommagement à gradient comme un modèle de rupture et validation expérimentale. On propose une formulation variationnelle des modèle d'endommagement à gradient en dynamique. Une définition rigoureuse de plusieurs taux de restitution d'énergie dans le modèle d'endommagement est donnée et on démontre que la pointe de la fissure dynamique est régie par un critère de Griffith généralisé ainsi que son interprétation asymptotique. On décrit ensuite une implémentation numérique efficace basée sur une discrétisation par éléments finis standards en espace et la méthode de Newmark en temps. Les résultats de simulation obtenus issu des calculs parallèles sont discutés d'un point de vue computationnel et physique. Les lois constitutives d'endommagement et les formulations d'asymétrie en traction/compression sont comparées par rapport à leur aptitude à modéliser la rupture fragile. Les propriétés spécifiques du modèle d'endommagement à gradient en dynamiques sont analysées séparément pour la nucléation, l'initiation, la propagation, l'arrêt, le branchement et la bifurcation du défaut. Applications sur des vraies structures en béton avec  éventuelles armatures en acier sont considérées à la fin de cette étude.
\end{otherlanguage}
\end{mdframed}

\vspace{20pt}

\begin{mdframed}
\textbf{Title:} Treatment of strain localization using a phase-field method for reinforced concrete structures under a violent dynamic loading

\textbf{Keywords:} dynamic brittle fracture, gradient damage models, variational approach, numerical implementation

\textbf{Abstract:} In civil engineering, mechanical integrity of the reinforced concrete structures under severe transient dynamic loading conditions is of paramount importance for safety and calls for an accurate assessment of structural behaviors in presence of dynamic crack propagation. In this work, we focus on the constitutive modeling of concrete regarded as an elastic-damage brittle material. The strain localization evolution is governed by a gradient-damage approach where a scalar field achieves a smeared description of dynamic fracture phenomena. The contribution of the present work is both theoretical and numerical and is classified using four thematic subjects: going dynamical, bridging the link with phase-field approaches, better understanding of gradient-damage modeling of fracture and experimental validation. We propose a variationally consistent formulation of dynamic gradient damage models. A formal definition of several energy release rate concepts in the gradient damage model is given and we show that the dynamic crack tip equation of motion is governed by a generalized Griffith criterion as well as its asymptotic interpretation. We then give an efficient numerical implementation of the model based on a standard finite-element spatial discretization and the Newmark time-stepping methods. Simulations results obtained with parallel computing are discussed both from a computational and physical point of view. Different damage constitutive laws and tension-compression asymmetry formulations are compared with respect to their aptitude to approximate brittle fracture. Specific properties of the dynamic gradient damage model are investigated independently for defect nucleation, initiation, propagation, arrest, kinking and branching. Applications to real concrete structures with possible steel reinforcements are considered at the end.
\end{mdframed}
\end{flushleft}

\vfill

\begin{minipage}[b]{0.6\textwidth}
\small
\color{color02}
\textbf{Université Paris-Saclay} \\
Espace Technologique / Immeuble Discovery  \\
Route de l'Orme aux Merisiers RD 128 / 91190 Saint-Aubin, France
\end{minipage}
\hfill
\begin{minipage}[b]{0.35\textwidth}
\hfill
\includegraphics[width=45pt]{PSaclay2.pdf}
\end{minipage}