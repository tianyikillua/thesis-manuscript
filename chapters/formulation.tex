\chapter{Dynamic Gradient Damage Models}
\minitoc

\section{Variational Framework Based on Physical Principles} \label{sec:formulation}
Give two descriptions of the surface energy

\section{Extension to Large Displacements}
We have implicitly supposed a hyperelastic behavior for the underlying gradient damage material through the definition of a strain energy function in \eqref{eq:elastic}. Use of hypoelastic materials is also frequent in dynamic calculations due to their relatively low computational cost: only the stress increment $\Delta\sig_t$ needs to be calculated given a strain increment $\Delta\eps_t$. However from a theoretic point of view, a good objective rate of the stress tensor should be carefully chosen for the hypoelastic law to be physically sound, which may complicates its numerical implementation \cite{SimoPister:1984}. Under quasi-static hypothesis authors of \cite{PieroLancioniMarch:2007,MieheSchaenzelUlmer:2015} use a Lagrangian strain measure based on the right Cauchy-Green tensor $\vec{F}_t^\mT\vec{F}_t$ for the finite-strain extension of phase-field models. It is a natural choice since the current configuration $\Omega_t$ is not known in advance for quasi-static calculations and the static equilibrium is written either in the initial reference configuration $\Omega=\Omega_0$ (total Lagrangian formulation) or in the last known reference configuration (updated Lagrangian formulation). In explicit dynamics however, dynamic momentum balance can be directly prescribed in the current configuration $\Omega_t$ which is calculated from the last iteration following the temporal discretization scheme. For this reason in this work we will use the Eulerian Hencky logarithmic strain tensor \cite{XiaoBruhnsMeyers:1997}
\begin{equation} \label{eq:logstrain}
\eps(\vec{u}_t)=\vec{h}_t=\log\vec{V}_t=\sum_i(\log\lambda_i)\vec{n}_i\otimes\vec{n}_i
\end{equation}
where $\vec{V}_t$ is the left stretch tensor from the polar decomposition $\vec{F}_t=\mathbb{I}+\nabla\vec{u}_t=\vec{V}_t\vec{R}_t$. Based on this strain measure, a simple Hookean type hyperelastic model \cite{XiaoChen:2002} is adopted
\begin{align}
\psi_0(\vec{h}_t) &= \frac{1}{2}\lambda(\tr\vec{h}_t)^2+\mu\vec{h}_t\cdot\vec{h}_t, \label{eq:soundpsi} \\
\vtau_0(\vec{h}_t) &= \frac{\partial\psi_0}{\partial\vec{h}}(\vec{h}_t)=\lambda(\tr\vec{h}_t)\mathbb{I}+2\mu\vec{h}_t \label{eq:henckystress}
\end{align}
where we emphasize that it is the Kirchhoff stress $\vtau_0(\vec{h}_t)=J_t\sig_0(\vec{h}_t)$ with $J_t=\det\vec{F}_t$ the Jacobian determinant and not the Cauchy stress $\sig_0$ that is derived from this strain energy $\psi_0$.

\section{Tension-Compression Asymmetry Considerations}

\section{Links between Damage and Fracture}

\subsection{Griffith Revisited}
