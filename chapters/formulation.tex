\chapter{Dynamic Gradient Damage Models}
\minitoc

\section{Variational Framework Based on Physical Principles} \label{sec:formulation}
Give two descriptions of the surface energy

\subsection{Comparison with phase-field approaches} \label{sec:phasefields}
The phase-field model studied in \cite{KarmaKesslerLevine:2001,HakimKarma:2009} constitutes another continuum and regularized approach for quasi-static and dynamic fracture problems. A specific damage-like\footnote{Indeed, the function $1-\phi$ can be seen as the damage using their definition of the phase field $\phi$.} scalar \emph{phase} field $\phi$ is introduced to continously separate the broken state $\phi=0$ and the sound one $\phi=1$. In a quasi-static setting, the governing equations for the displacement $\vec{u}$ and the phase fields $\phi$ can be obtained in a semi-variational way from the total energy $\mathcal{E}(\vec{u},\phi)$, which, using notations used in \cite{HakimKarma:2009}, gives
\begin{align}
0 &= \frac{\partial\mathcal{E}}{\partial\vec{u}}(\vec{u},\phi) \label{eq:PF_u} \\
\chi^{-1}\dot{\phi} &= -\frac{\partial\mathcal{E}}{\partial\phi}(\vec{u},\phi) \label{eq:PF_phi}
\end{align}
where \eqref{eq:PF_u} describes the static equilibrum of the body with potential diffuse or localized damage zones and \eqref{eq:PF_phi} is the standard Ginzburg-Landau equation with $\chi>0$ a kinetic or mobility \cite{KuhnMuller:2010} coefficient controlling the (physical) additional total energy disspation in the form of heat during the crack propagation, as can be seen by following (corrected) equation based on (13) in \cite{HakimKarma:2009}
\begin{equation} \label{eq:PF_dissipation}
\dot{\mathcal{E}}=-\chi\left(\frac{\partial\mathcal{E}}{\partial\phi}(\vec{u},\phi)\right)^2\leq 0.
\end{equation}

Note that the Griffith-like crack creation is the only dissipation mechanism in our gradient damage model and an energy balance condition is added in the formulation to ensure that all the elastic energy released will be used to supplement crack propagation. This is the major formulational difference between our model and the \emph{dissipative} phase field models because of \eqref{eq:PF_dissipation}. A parallel consequence of the appearance of a kinetic coefficient $0<\chi<\infty$ in \eqref{eq:PF_phi}, as discussed in \cite{Bourdin:2011}, is that an evolutionary parabolic equation \eqref{eq:PF_phi} governing the phase field is coupled with the elliptic static equilibrium problem \eqref{eq:PF_u}. Physically it means that the crack can evolve solely with a rate determined by $\chi$, even if the structure is in static equilibrium at $t=T$ with all external loading frozen for all $t>T$. With a physical time being introduced into the model (the dimension of the kinetic coefficient is $[\mathrm{T}]^{-1}$), the coupled system \eqref{eq:PF_u}-\eqref{eq:PF_phi} isn't well suited for quasi-static computations, as numerically the static equilibrium  $\vec{K}\vec{u}=\vec{F}$ should be combined with a specific time-stepping scheme (the explicit Euler scheme in \cite{HakimKarma:2009}) to integrate the evolution problem for the phase field.

Other differences exist and concern mainly the dependence of the total energy on the phase field $\phi\mapsto\mathcal{E}(\vec{u},\phi)$. Our formulation is general in the sense that the stiffness degradation $\alpha\mapsto a(\alpha)$ and the local damage dissipation $\alpha\mapsto w(\alpha)$ are only required to verify some physical properties based on which three particular constitutive laws are studied in this paper. As can be been in the subsequent numerical experiments, all these models are similar in essence and can be used to investigate brittle fracture problems. On the contrary, the dissipative phase field models \cite{KarmaKesslerLevine:2001,HakimKarma:2009} seem to favour a particular set of constitutive functions. As long as these functions verify the properties, they can be seen to be contained in our general gradient damage model.

\section{Extension to Large Displacements}
We have implicitly supposed a hyperelastic behavior for the underlying gradient damage material through the definition of a strain energy function in \eqref{eq:elastic}. Use of hypoelastic materials is also frequent in dynamic calculations due to their relatively low computational cost: only the stress increment $\Delta\sig_t$ needs to be calculated given a strain increment $\Delta\eps_t$. However from a theoretic point of view, a good objective rate of the stress tensor should be carefully chosen for the hypoelastic law to be physically sound, which may complicates its numerical implementation \cite{SimoPister:1984}. Under quasi-static hypothesis authors of \cite{PieroLancioniMarch:2007,MieheSchaenzelUlmer:2015} use a Lagrangian strain measure based on the right Cauchy-Green tensor $\vec{F}_t^\mT\vec{F}_t$ for the finite-strain extension of phase-field models. It is a natural choice since the current configuration $\Omega_t$ is not known in advance for quasi-static calculations and the static equilibrium is written either in the initial reference configuration $\Omega=\Omega_0$ (total Lagrangian formulation) or in the last known reference configuration (updated Lagrangian formulation). In explicit dynamics however, dynamic momentum balance can be directly prescribed in the current configuration $\Omega_t$ which is calculated from the last iteration following the temporal discretization scheme. For this reason in this work we will use the Eulerian Hencky logarithmic strain tensor \cite{XiaoBruhnsMeyers:1997}
\begin{equation} \label{eq:logstrain}
\eps(\vec{u}_t)=\vec{h}_t=\log\vec{V}_t=\sum_i(\log\lambda_i)\vec{n}_i\otimes\vec{n}_i
\end{equation}
where $\vec{V}_t$ is the left stretch tensor from the polar decomposition $\vec{F}_t=\mathbb{I}+\nabla\vec{u}_t=\vec{V}_t\vec{R}_t$. Based on this strain measure, a simple Hookean type hyperelastic model \cite{XiaoChen:2002} is adopted
\begin{align}
\psi_0(\vec{h}_t) &= \frac{1}{2}\lambda(\tr\vec{h}_t)^2+\mu\vec{h}_t\cdot\vec{h}_t, \label{eq:soundpsi} \\
\vtau_0(\vec{h}_t) &= \frac{\partial\psi_0}{\partial\vec{h}}(\vec{h}_t)=\lambda(\tr\vec{h}_t)\mathbb{I}+2\mu\vec{h}_t \label{eq:henckystress}
\end{align}
where we emphasize that it is the Kirchhoff stress $\vtau_0(\vec{h}_t)=J_t\sig_0(\vec{h}_t)$ with $J_t=\det\vec{F}_t$ the Jacobian determinant and not the Cauchy stress $\sig_0$ that is derived from this strain energy $\psi_0$.

\section{Tension-Compression Asymmetry Considerations}

\section{Links between Damage and Fracture}

\subsection{Griffith Revisited}
