\chapter{Introduction} \label{chap:introduction}
\minitoc

This chapter exposes the reader to the general physical context and outlines the motivation and objectives of the present work. The fundamental background of dynamic brittle fracture are first recalled in \cref{sec:dynafrac}. Several modeling approaches are discussed from a physical and numerical point of view in \cref{sec:models}. An introductory presentation of gradient damage models is given in \cref{sec:graddamage}.

\section{Dynamic Brittle Fracture} \label{sec:dynafrac}
The concept of cracks constitutes the \emph{raison d'être} of fracture mechanics\index{Fracture}. Specifically, fracture mechanics focuses on the evolution of cracks as well as their impact on the structural behaviors. The objective of fracture mechanics is to better understand different crack evolution phases by providing their corresponding governing physical criteria. They can then be used by civil engineers and material scientists to optimize the structural dimensioning and design, and to readjust the chemical composition to ensure integrity of the composite. From a kinematic point of view, appearance of cracks often leads to separation of the body into two or more pieces, where the displacement vector defined in the initial configuration becomes discontinuous across them. The temporal and spatial evolution of cracks can be characterized by several stages or events which are summarized as follows. The temporal evolution focuses on \emph{when} cracks propagate:
\begin{itemize}
\item \textbf{Nucleation and/or initiation} concerns the appearance of a propagating crack inside a body (or on its boundary) due to external loading. \emph{Nucleation} refers to the formation of cracks from a perfectly unflawed configuration. From a material point of view, the nucleation event should be considered as a macroscopic modeling simplification since micro cracks or flaws may be present at a lower scale and may eventually evolve into macro-cracks under the influence of external loading. These material or structural imperfections are in general not accounted for in a continuum mechanics approach and we consider that a body is initially sound when stress singularity is absent from an elastic modeling viewpoint. On the other hand, crack \emph{initiation} is well defined and refers to the time at which the existing macro-crack or the defect begins to propagate in the structure.

\item \textbf{Propagation} phase is
the most dangerous part of defect evolution for industrial applications as in constitutes a threat to structural integrity. Crack propagation is systematically accompanied by a material-dependent energy consumption, which is balanced by a release of the total mechanical energy. This energy balance concept is the cornerstone of several theoretical models of fracture mechanics.

\item \textbf{Arrest}
\end{itemize}

Meanwhile, the spatial evolution refers to the path along which the crack propagates, \emph{i.e.} how cracks propagate:
\begin{itemize}
\item \textbf{Curving and kinking}
\item \textbf{Branching}
\item \textbf{Other topology changes}
\end{itemize}


\begin{figure}[htbp]
\centering
\includegraphics[width=0.6\textwidth]{nexus5.pdf}
\caption{Several fracture mechanics phenomena displayed by the \emph{post-mortem} crack patterns on the broken screen of a Google Nexus 5 phone obtained after an unintentional drop test.}
\end{figure}

Cracks can be defined as a macroscopic manifestation of material defects at a microscopic scale. Hence different materials are in general associated with a different failure mechanism.

Dynamic fracture\index{Fracture!Dynamic fracture}. Put a photo of circ. stress distrbution in dynamics.

\section{Different Modeling Approaches to Fracture Mechanics}  \label{sec:models}
We give firstly a quick review on mainstream computational methods within the Continuum Mechanics framework for (dynamic) crack propagation in quasi-brittle materials. Their respective numerical implementation put aside, the physical (mathematical) models can be roughly classified into two categories:
\begin{enumerate}
\item Discrete approach involving displacement discontinuities
\item Continuous approach consisting in regularizing the crack by smooth fields
\end{enumerate}

\subsection{Discrete modeling of cracks}
In the first approach lies the classical Linear Elastic Fracture Mechanics pioneered by \cite{Griffith:1921} based on local crack tip singularities, namely the stress intensity factors $K_i$ or their energetic interpretations $G$, namely the total strain energy release rate, thanks to Irwin's formula. Theses local values are to be compared with a material dependent property, called the fracture toughness $K_\mc$ or $\gc$, which furnishes the desired relationship giving the crack location as a function of external loading. From the physical point of view its drawbacks can be roughly summarized \cite{FrancfortMarigo:1998} by the inability to initiate a crack in structures lacking enough initial strong singularities, to determine itself solely (without additional hypothesis such as the principle of local symmetry) the crack path which is often and more importantly must be presupposed for plane crack propagation problems, and finally to account for brutal propagation involving temporel discontinuities in quasi-static settings. The cohesive zone (interface) model as used in \cite{DebruyneLaverneDumouchel:2012,RuizOrtizPandolfi:2000} postulates a direct relationship between the traction acting on the crack lips $\vec{t}$ and the opening displacement $\llbracket\vec{u}\rrbracket$. It regularizes the initial Griffith theory by introducing a critical stress $\sigma_\mc$ into the model, thus making possible the initiation of cracks in perfectly sound structures. The third point can be circumvented by considering inertial effects in a dynamic setting \cite{Freund:1990}, where the crack velocity can be shown to be bounded by a certain limiting value depending on the problem nature: the Rayleigh wave speed in mode I propagation and the shear wave speed in mode II and III situations. Crack path prediction, now including not only kinking but also branching or eventual arrest of cracks in dynamic fracture, still needs to be addressed separately as the only crack tip equation that we have $G(v)=\gc(v)$ is not changed, where compared to quasi-static models a direct dependence on the crack tip speed is now embedded into the local crack tip singularities $G(v)$ as well as the apparent fracture toughness $\gc(v)$. Nevertheless, the theoretic foundation of this dynamic version of Griffith's theory has been somehow questioned \cite{ValorosoDebruyneLaverne:2014} according to some well-designed experimental work on dynamic fracture. The limiting crack speed is hardly observed and microscopic effects seems to play an essential role in the macroscopic behavior of cracks \cite{Ravi-ChandarKnauss:1984a}.

Small scale yielding
Problem of stress infinity -> introduction of plasticity, or

Numerically the first approach often demands tracking explicitly the current crack geometry (lips and tip) and representing precisely the discontinuous displacement across the cracks. Classical remeshing techniques \cite{ShahaniAmini:2009} are obviously needed to open a propagating crack which is associated with a change of topology of the initial mesh. The accurate evaluation of local singularities is also a major issue in the classical Finite Element Method, whose convergence performance is largely bounded by these singularities. Since the advent of the eXtended-Finite Element Method and some geometrical tracking tools (as level sets) as used in \cite{RethoreGravouilCombescure:2005,GregoireMaigreRethoreCombescure:2007,MenouillardRethoreCombescureBung:2006}, cracks can now be freely incorporated into the physical model based on a fixed mesh. Crack tip singularities are directly embebed in the local interpolation functions near the cracks and the convergence rate is considerably improved. Nevertheless, it should be noted that the inherent limitations of the physical model are still present and a more powerful formulation of fracture is needed to bypass the cited pitfalls of classical Griffith theory.

It is the Variational Approach to Fracture as well as its gradient damage regularization \cite{BourdinFrancfortMarigo:2008} that settles down a complete unified framework covering the onset and the space-time propagation of cracks with possible complex topologies. A mere retranslation of Griffith's original idea, the variational formulation focuses on global energetic quantities and sees the crack evolution as a minimization movement of the sum of the stored elastic energy and the crack surface energy. Stated in this way, we are dealing with what is called a Free Discontinuity Problem where the crack path with possible jumps in time can be directly predicted. Several mathematical or physical subtleties arise on the precise minimization structure (global or local minimization) as well as on the form of the crack surface energy, but the major concern lies on its effective numerical implementation as we have to presuppose certain general but well-defined \emph{topology} of the crack set. That's when a mathematically sound $\ell$-parametrized two-field functional approximation \cite{BourdinFrancfortMarigo:2000} enters the scene. It consists of introducing an auxiliary continuous scalar field representing the unknown location of cracks while preserving a certain link with the former sharp interface problem within the framework of $\Gamma$-convergence theories intensively studied in the context of image segmentation. Initially the approximation treats only the Griffith surface energy, but it is also possible \cite{ContiFocardiIurlano:2015} to consider some different cohesive surface energies using similar techniques. Classical finite element methods can therefore be used to discretize the displacement and the new scalar field and numerical optimization techniques can be applied to solve the fracture evolution by minimizing the total potential energy \cite{PieroLancioniMarch:2007,MauriniBourdinGauthierLazarus:2013,HossainHsuehBourdinBhattachary:2014,MesgarnejadBourdinKhonsari:2014}.

\subsection{Continuum modeling of cracks}
On the other hand, we can regard the regularized formulation as a genuine physical model \emph{per se} and the introduced crack-indicator field as damage from the mechanical point of view, as it turns out that the two-field energy functional resembles that of the gradient damage model \cite{PhamMarigo:2010,PhamMarigo:2010-1,PhamAmorMarigoMaurini:2011,SicsicMarigoMaurini:2013}. The governing equations for damage are also formulated in a variational setting involving three principles of irreversibility, stability and energy balance. The irreversibility prevents a damaged zone from self-healing, the stability criterion compares the current state with other neighboring perturbed states, which is physically sounder than a global minimization one, and the energy conservation limits the only dissipation mechanism to damage. The presence of the damage gradient renders the model non-local and can be seen as a regularization of classical Continuum Damage Mechanics to avoid spurious mesh dependence, which provides an internal length $\ell$ of the model controlling the damage band. Compared to the former crack regularization point of view where this parameter $\ell$ is viewed as a pure numerical parameter, here it can be shown to be related directly to the critical stress similar to cohesive zone models, thus giving its material parameter nature. The links of this gradient damage model with the fracture model can be done in more mechanical languages by a separation of scale \cite{SicsicMarigo:2013,LorentzCuvilliezKazymyrenko:2011,LorentzCuvilliezKazymyrenko:2012}. Moreover, the tension-compression asymmetry observed in several brittle materials such as concrete can be easily modeled and incorporated into the gradient damage model without preoccupying with its $\Gamma$-convergence fracture model counterpart, as already did in \cite{AmorMarigoMaurini:2009,FreddiRoyer-Carfagni:2010,LancioniRoyer-Carfagni:2009}. This point is rather appreciated in dynamics, as the tensile and compressive waves propagate and reflect within the structure.

The dynamic extension \cite{Bourdin:2011} of the regularized variational approach to fracture consists of formally replacing the static equilibrium by the dynamic one in a time-discrete setting. However there the gradient damage interpretation is not acknowledged and the functional approximations used present some mechanical issues (more precisely the lack of a pure elastic zone), as is also the case for other similar methods \cite{BordenVerhooselScottHughesLandis:2012,HofackerMiehe:2012,SchlueterWillenbuecherKuhnMueller:2014}. The scope of this thesis is thus focused on the further exploration of the dynamic gradient damage models. More precisely, we will formulate the dynamic problem in lines with the initial quasi-static formulation \cite{PhamMarigo:2010-1} and consider several tension-compression asymmetry models. Numerical implementation is then detailed covering the spatial and temporal discretization as well as some additional numerical issues therein. Some representative simulation examples will be then given to highlight some of its characteristics. In the first case the crack path will be enforced and we compare with the available analytical results to establish a connection of our gradient damage model with the dynamic Griffith model, hoping to provide some indications of the possible dynamic extension of the quasi-static theoretic results \cite{SicsicMarigo:2013}. We then investigate the quasi-static limit of our dynamic model when the loading speed tends to zero, by considering material inhomogeneities. In the second case we let the cracks choose its own path and we study firstly in an 1d setting interactions of the damage and the elastic waves to highlight some differences between the dynamic model and the quasi-static one. A simplified 2d setting (semi-infinite domain) will be used to investigate dynamic crack path prediction involving the mysterious branching and comparaison with experimental results \cite{Ravi-ChandarKnauss:1984a,Ravi-ChandarKnauss:1984b} will be made. Finally we give some industrial examples to roughly verify and validate this dynamic model with other numerical or experimental results \cite{SongWangBelytschko:2008}.

Macroscopic modeling (continuum mechanics)
\begin{itemize}
\item First approach : LEFM, Variational approach to fracture (quasi-statics of F-M and dynamics of Larsen) with Griffith surface energy or cohesive models
\item Second approach : phase-field models (mechanical community), dissipative phase-field models (physical community), Peridynamics
\item Combination of two approaches : TLS
\end{itemize}

\section{Gradient Damage Models in a Nutshell} \label{sec:graddamage}
See the gradient damage model as
- a numerical regularization of the sharp interface
- a regularization of strain localization of classical damage models, cite paper of Lorentz
- Why damage gradient ? cite Strain gradient vs damage

Illuyshin can be used to justify variational models
Justify and give motivations of the gradient
Asymptotic development
Via Gamma-convergence results

Explain differences between brittle and ductile fracture
3d validation?
%Highlight the differences with respect to existing phase-field methods w(alpha)=alpha^2

\subsection{Comparaison with the dissipative phase-field approaches}

The phase-field model studied in \cite{KarmaKesslerLevine:2001,HakimKarma:2009} constitutes another continuum and regularized approach for quasi-static and dynamic fracture problems. A specific damage-like\footnote{Indeed, the function $1-\phi$ can be seen as the damage using their definition of the phase field $\phi$.} scalar \emph{phase} field $\phi$ is introduced to continously separate the broken state $\phi=0$ and the sound one $\phi=1$. In a quasi-static setting, the governing equations for the displacement $\vec{u}$ and the phase fields $\phi$ can be obtained in a semi-variational way from the total energy $\mathcal{E}(\vec{u}_t,\phi_t)$, which, using notations used in \cite{HakimKarma:2009}, gives
\begin{align}
0 &= \frac{\partial\mathcal{E}}{\partial\vec{u}}(\vec{u}_t,\phi_t) \label{eq:PF_u} \\
\chi^{-1}\dot{\phi}_t &= -\frac{\partial\mathcal{E}}{\partial\phi}(\vec{u}_t,\phi_t) \label{eq:PF_phi}
\end{align}
where \eqref{eq:PF_u} describes the static equilibrum of the body with potential diffuse or localized damage zones and \eqref{eq:PF_phi} is the standard Ginzburg-Landau equation with $\chi>0$ a kinetic or mobility \cite{KuhnMuller:2010} coefficient controlling the (physical) additional total energy disspation in the form of heat during the crack propagation, as can be seen by following (corrected) equation based on (13) in \cite{HakimKarma:2009}
\begin{equation} \label{eq:PF_dissipation}
\dot{\mathcal{E}}_t=-\chi\left(\frac{\partial\mathcal{E}}{\partial\phi}(\vec{u}_t,\phi_t)\right)^2\leq 0.
\end{equation}

Note that the Griffith-like crack creation is the only dissipation mechanism in our gradient damage model and an energy balance condition is added in the formulation to ensure that all the elastic energy released will be used to supplement crack propagation. This is the major formulational difference between our model and the \emph{dissipative} phase field models because of \eqref{eq:PF_dissipation}. A parallel consequence of the apparence of a kinetic coefficient $0<\chi<\infty$ in \eqref{eq:PF_phi}, as discussed in \cite{Bourdin:2011}, is that an evolutionary parabolic equation \eqref{eq:PF_phi} governing the phase field is coupled with the elliptic static equilibrum problem \eqref{eq:PF_u}. Physically it means that the crack can evolve solely with a rate determined by $\chi$, even if the structure is in static equilibrum at $t=T$ with all external loading frozen for all $t>T$. With a physical time being introduced into the model (the dimension of the kinetic coefficient is $[\mathrm{T}]^{-1}$), the coupled system \eqref{eq:PF_u}-\eqref{eq:PF_phi} isn't well suited for quasi-static computations, as numerically the static equilibrum  $\vec{K}\vec{u}=\vec{F}$ should be combined with a specific time-stepping scheme (the explicit Euler scheme in \cite{HakimKarma:2009}) to integrate the evolution problem for the phase field.

Other differences exist and concern mainly the dependence of the total energy on the phase field $\phi_t\mapsto\mathcal{E}(\vec{u}_t,\phi_t)$. Our formulation is general in the sense that the stiffness degration $\alpha\mapsto a(\alpha)$ and the local damage dissipation $\alpha\mapsto w(\alpha)$ are only required to verify some physical properties based on which three particular constitutive laws are studied in this paper. As can be been in the subsequent numerical experiments, all these models are similar in essence and can be used to investigate brittle fracture problems. On the contrary, the dissipative phase field models \cite{KarmaKesslerLevine:2001,HakimKarma:2009} seem to favour a particular set of constitutive functions. As long as these functions verify the properties, they can be seen to be contained in our general gradient damage model.

\section*{Summary of this Chapter} \label{sec:summarychap1}
\begin{itemize}
\item Give an adequate variational framework for dynamic gradient damage models
\item Propose a large-strain extension in an explicit dynamics  context
\item Give a systematic analysis/better understanding of gradient damage models, not just use it to do real complex simulations
\item How it can be used to approximate brittle fracture
\item Use of gradient damage models to explore/understand dynamic fracture phenomena (size effect, branching, velocity effects, crack arrest\ldots)
\end{itemize}