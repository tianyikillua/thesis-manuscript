%!TEX root=../main.tex
\chapter{Introduction} \label{chap:introduction}
\minitoc

This chapter exposes the reader to the general physical context and outlines the motivation and objectives of the present work. The fundamental background of dynamic brittle fracture are first recalled in \cref{sec:dynafrac}. Several modeling approaches to fracture mechanics are discussed from a physical and numerical point of view in \cref{sec:discretemodels}. General remarks on a gradient damage approach are given in \cref{sec:graddamage}.

\section{Physics of Dynamic Brittle Fracture in a Nutshell} \label{sec:dynafrac}
The concept of cracks constitutes the \emph{raison d'être} of fracture mechanics\index{Fracture}. Specifically, fracture mechanics focuses on the evolution of cracks as well as their impact on the structural behaviors. The objective of fracture mechanics is to better understand different crack evolution phases by providing their corresponding governing physical criteria. They can then be used by civil engineers and material scientists to optimize the structural dimensioning and design, and to readjust the chemical composition to ensure integrity of the composite for instance. From a kinematic point of view, cracks $\Gamma$ are naturally defined in the reference configuration as a moving interface in the uncracked configuration $\Omega$, see \cref{fig:crackedconfig}. Due to external loading conditions, the deformed configuration $\vec{\varphi}(\Omega\setminus\Gamma)$ of the cracked body may be described by the usual displacement vector $\vec{u}$. The presence of cracks often leads to separation of the body into two or more pieces, where the displacement vector defined in the reference configuration becomes discontinuous across them. This constitutes a major difficulty when modeling cracks and their evolutions in a continuum mechanics framework, since displacements are in general considered to be at least continuous inside the body.
\begin{figure}[htbp]
\centering
\includegraphics[width=0.6\textwidth]{crakced_config.pdf}
\caption{Current cracked reference configuration $\Omega\setminus\Gamma$ and  its deformation defined by the displacement vector $\vec{u}$} \label{fig:crackedconfig}
\end{figure}

Cracks can be regarded as a macroscopic manifestation of material defects at a microscopic scale. Hence different materials are in general associated with a different failure mechanism. In the present work only brittle fracture phenomenon is considered, as opposed to ductile fracture.
\begin{itemize}
\item Generally speaking brittle fracture\index{Fracture!Brittle fracture} occurs without significant deformation of the material. Structural failure with such materials is accompanied by little energy dissipation. Quasi-brittle materials, by definition, satisfy these characteristics. It concerns ceramics, glass, rock, concrete and some polymers such as polymethyl-methacrylate (PMMA). Metals may as well observe a brittle behavior at lower temperatures.

\item Ductile fracture\index{Fracture!Ductile fracture}, on the contrary, is accompanied by moderate plastic (inelastic) deformation which takes place before the ultimate failure. It concerns mostly metals at room or higher temperatures. The ductile-to-brittle transition depends on the temperature, on the composition, but also on the strain rate the material is subject to, see for example \cite{Kalthoff:2000}.

\item To discriminate between brittle and ductile fracture, near-tip behaviors of the mechanical fields can be analyzed, see \cref{fig:brittleductile}. Brittle fracture can be characterized by a globally nearly elastic behavior, possibly expect inside a small region, called fracture process zone, where non-elastic effects (plasticity, damage, \ldots) take place. It is called the small-scale yielding condition\index{Small-scale yielding} where the elasticity dominates the structural behavior and the crack evolution. On the other hand, plasticity plays an essential role in ductile fracture, since a significant plastic zone surrounds the crack tip. Inelastic material behaviors must be taken into account in order to predict the evolution of the cracked body.
\end{itemize} 
\begin{figure}[htbp]
\centering
\includegraphics[width=0.8\textwidth]{brittle-ductile.pdf}
\caption{Comparison between brittle fracture (a) and ductile fracture (b) in terms of near-tip behaviors of the mechanical fields} \label{fig:brittleductile}
\end{figure}

In this work inertial effects are taken into account in the structural analysis of cracked bodies. This is the object of dynamic fracture\index{Fracture!Dynamic fracture}. As opposed to the traditional quasi-static approach, the framework of dynamic fracture focuses on some specific problem settings and may present some theoretic advantages.
\begin{itemize}
\item The introduction of kinetic energy brings a physical time scale to the fracture problem. Inertial effects could not be ignored if one wants to analyze the transient behavior of structures due to external dynamic loadings such as impact or the interactions between stress waves and the crack \cite{Ravi-ChandarKnauss:1984b}.

\item Even though the structure is subject to slowly applied loads such that the quasi-static hypothesis is verified, the crack itself may still propagates at a speed comparable to that of the mechanical waves. In the classical fracture mechanics theory, such situations refer to an \emph{unstable propagation} since crack evolution is no longer controllable through external hard or soft devices applied to the body. A complete theoretic framework for analyzing such unstable propagations necessarily includes dynamics.
\end{itemize}

Dynamic fracture is not only reserved for industrial civil structures. It also concerns daily objects whenever they are subject to extreme loading conditions. A broken screen of a smartphone due to impact is illustrated in \cref{fig:nexus5}.
\begin{figure}[htbp]
\centering
\includegraphics[width=0.7\textwidth]{nexus5.pdf}
\caption{Several (dynamic) fracture mechanics phenomena displayed by the \emph{post-mortem} crack patterns on the broken screen of a Google Nexus 5 phone obtained after an unintentional drop test.} \label{fig:nexus5}
\end{figure}
The screen is made of glass and the failure can be characterized as brittle fracture. The temporal and spatial evolution of cracks can be characterized by several stages or events which are summarized as follows. The temporal evolution focuses on \emph{when} cracks propagate:
\begin{itemize}
\item \textbf{Nucleation and/or initiation} concerns the appearance of a propagating crack inside a body (or on its boundary) due to external loading. \emph{Nucleation} refers to the formation of cracks from a perfectly flawless configuration. From a material point of view, the nucleation event should be considered as a macroscopic modeling simplification since micro cracks or flaws may be present at a lower scale and may eventually evolve into macro-cracks under the influence of external loading. These material or structural imperfections are in general not accounted for in a continuum mechanics approach and we consider that a body is initially sound when stress singularity is absent from an elastic modeling viewpoint. On the other hand, crack \emph{initiation} refers to the time at which the existing macro-crack or the defect begins to propagate in the structure.

\item \textbf{Propagation}, being stable or not, is
the most dangerous part of defect evolution for industrial applications as in constitutes a threat to structural integrity. Crack propagation is systematically accompanied by an energy consumption characterized by the fracture toughness\index{Fracture toughness} of the material. It measures the energy required to open a crack of unit surface. This energy consumption is is balanced by a release of the total mechanical energy. This energy balance concept is the cornerstone of several theoretical models of fracture mechanics \cite{Griffith:1921,Mott:1947}. According to experiments performed on brittle materials \cite{Ravi-ChandarKnauss:1984a}, there exists a terminal velocity for crack propagation depending on the solicitation modes.

\item \textbf{Arrest} refers to a propagating crack that becomes stationary in a continuous or abrupt fashion. In the latter case, arrest can no longer be considered as a time reversal of the crack initiation process \cite{Ravi-ChandarKnauss:1984c}.
\end{itemize}

Meanwhile, the spatial evolution refers to the path along which the crack propagates, \emph{i.e.} how cracks propagate. In a two-dimensional setting, the crack path can be characterized by the following concepts:
\begin{itemize}
\item \textbf{Curving and kinking} concerns curvature evolution of the crack path. When idealizing the crack as a mathematical curve $l\mapsto\vec{\gamma}(l)$, crack curving refers to a tangent that varies continuously along the path, as opposed to kinking\index{Kinking} where a discontinuous change of crack propagation direction takes place, see \cref{fig:kink}. This last can be considered as a theoretic modeling of a crack that suddenly deviates from its initial propagation direction.
\begin{figure}[htbp]
\centering
\includegraphics[width=0.35\textwidth]{kinking.pdf}
\caption{Curved crack path versus kinked crack path} \label{fig:kink}
\end{figure}

\item \textbf{Branching} refers\index{Branching} to the splitting of a primary propagating crack into two or several branches. From a macroscopic modeling viewpoint, it involves a topology change of the crack set, since additional crack tips are created after such process. This point of view of crack branching is experimentally recorded by \cite{Schardin:2012}. On the other hand, by investigating the microstructure of fracture process zone, it is observed in \cite{Ravi-ChandarKnauss:1984,Ravi-ChandarKnauss:1984a,SharonGrossFineberg:1995,SharonFineberg:1996} that such macro-branching phenomenon is always preceded by the so-called micro-branching attempts. It corresponds to a dynamic instability reviewed in \cite{FinebergMarder:1999} where micro cracks develop and interact with the primary single crack when propagating above a critical velocity. More energy is dissipated along the main crack (see \cite{SharonGrossFineberg:1996}), which provides a physical interpretation of using an apparent velocity-dependent fracture toughness for the primary crack.
\end{itemize}
Remark that other more complex topology changes could affect the spatial path of the crack set, which include coalescence (merging) of several cracks for instance.

\section{Different Modeling Approaches to Fracture Mechanics} \label{sec:discretemodels}
A non-comprehensive review of mainstream physical and computational models of fracture mechanics is given here. They will be discussed with respect to their aptitude to approximate dynamic brittle fracture phenomena. The discussion is intentionally limited to approaches formulated within the Continuum Mechanics framework where the body occupies a connected subset $\Omega$ of the Euclidean space as its reference configuration. The kinematics and forces that the body experiences can be described by material fields defined on $\Omega$. A finer description at a lower scale, such as molecular dynamics models \cite{AbrahamBrodbeckRafeyRudge:1994}, lattice dynamics calculations \cite{MarderGross:1995} and the discrete elements method \cite{HentzDonzeDaudeville:2004}, is not considered here.

Based on the kinematic description of cracks in the continuum body, different fracture mechanics models can be classified into the following three categories:
\begin{enumerate}
\item \textbf{Discrete modeling approach} where the crack is considered as an explicit sharp interface in the body across which the displacement vector is discontinuous. The advantage of the sharp-interface description of cracks lies in the explicit definition of a crack surface in the body, which leads to an unambiguous and quantifiable evolution of the crack front. It includes but is not limited to the classical Griffith's theory\index{Griffith's theory} of fracture mechanics \cite{Freund:1990}, the Variational Approach to\index{Variational Approach to Fracture} Fracture \cite{FrancfortMarigo:1998,BourdinFrancfortMarigo:2008,Larsen:2010} and the cohesive zone\index{Cohesive zone model} models \cite{Barenblatt:1962}.

\item \textbf{Smeared modeling approach} where strong discontinuities are regularized by strain localizations within a finite and thin band. This approach is referred as the ``element deletion method'' in a comparative study on finite element methods for dynamic fracture \cite{SongWangBelytschko:2008}. A review paper \cite{ BorstRemmersNeedlemanAbellan:2004} compares the discrete and the smeared modeling approaches from a fracture modeling viewpoint. The smeared description of cracks no longer refers to a certain topology of the crack as compared to the discrete modeling approach. Precisely, it provides an approximation of the crack topology which may become particularly complex due to branching and coalescence phenomena. The gradient damage\index{Gradient damage model} model \cite{PhamMarigo:2010-1,PhamAmorMarigoMaurini:2011} formulated in the rate-independent evolution framework in the sense of \cite{Mielke:2005} falls into this category. It admits other physics-based formulations such as \cite{Comi:1999} or variational formulations like \cite{LorentzAndrieux:1999}. The phase-field models\index{Phase-field models} originated from the mechanical community \cite{HofackerMiehe:2012,MieheWelschingerHofacker:2010,BordenVerhooselScottHughesLandis:2012} and the physical community \cite{HakimKarma:2009,KarmaKesslerLevine:2001} are also similar in essence to gradient damage approaches. We observe that the gradient of the damage field or the phase field is introduced in these models. It can be considered as a non-local regularization of conventional mathematically ill-posed local damage models, see \cite{LorentzAndrieux:2003}. Other non-local regularizations will be discussed in detail in \cref{sec:graddamage}. Finally, the peridynamic\index{Peridynamics} approach is also gaining popularity in the last years (see \cite{Silling:2010aa} for a review on its theory and applications). It can be regarded as a \emph{strongly} non-local continuum mechanics model that generalizes other integral-type or gradient-enhanced non-local models reviewed in \cite{LorentzAndrieux:2003}.

\item A combination of the previous two approaches where a \textbf{transition} between a smeared description and a discrete description of cracks is achieved. The work of \cite{CuvilliezFeyelLorentzMichel-Ponnelle:2012} concerns a transition between the gradient damage model and the cohesive zone model. The Thick Level Set approach\index{Thick level set approach} introduced in \cite{MoesStolzBernardChevaugeon:2011,MoreauMoesPicartStainier:2015} provides another unified framework incorporating a discontinuous crack description surrounded by continuous strain-softening regions.
\end{enumerate}

In the sequel, the Griffith's theory as synthesized in the monograph \cite{Freund:1990} is first recalled. It constitutes the most classical approach to fracture mechanics and provides a reference model for comparisons with other formulations. With the help of modern tools of the Calculus of Variations, its main idea based on energetic competition is formalized and extended to a general setting within the the Variational Approach to Fracture \cite{BourdinFrancfortMarigo:2008}, of which an introductory presentation is then given. Finally we turn to the main objective of this present work and provide a general presentation and physical motivations of the gradient damage model. It will be compared with other phase-field approaches and non-local methods indicated above. 

\section{Griffith's Theory of Dynamic Fracture}
Several formulations of the Griffith's theory of dynamic fracture mechanics exist. The Newtonian approach \cite{Freund:1990} is the most classical one and is herein summarized. The Eshelbian point of view \cite{Eshelby:1975} exploits the symmetry possessed by an generalized action integral but the derived so-called energy-momentum tensor still needs to be combined with local momentum and energy balance conditions to produce the crack equation of motion \cite{Maugin:1994,Adda-BediaAriasAmarLund:1999}.

The fundamental assumption underlying the Griffith's theory of fracture concerns the energy dissipation of a propagating crack $\Gamma$. It is modeled as a sharp-interface surface in the bulk $\Omega$. Griffith postulates in his pioneering work \cite{Griffith:1921} that the creation of a crack calls for an energy consumption that is proportional to its total area $\abs{\Gamma}$ which characterizes the amount of energy needed to break the atomic bonds on the crack surface at a microscopic scale. The crack surface can thus be regarded to possess a surface energy which reads
\begin{equation} \label{eq:StGriffith}
\mathcal{S}=\gc\cdot\abs{\Gamma}
\end{equation}
where $\gc$ is called the fracture toughness\index{Fracture toughness}, \emph{i.e.} the energy required to create a crack of unit surface in the body $\Omega$. Griffith assumes that $\gc$ is a material constant that characterizes the resistance of the material to crack formation.

\subsection{Boundary-value evolution problem}
The boundary-value evolution problem is obtained by considering local momentum equilibrium in the uncracked bulk and an energy flux integral entering into the crack tip which balances the energy dissipated due to crack propagation \cite{NakamuraShihFreund:1985,Cherepanov:1989}. Consider a two-dimensional homogeneous and isotropic cracked body as illustrated in \cref{fig:crackedconfig}. In this case the crack can be parametrized by its current arc-length denoted by $l$. We place ourselves under the small displacement hypothesis for formulational simplicity, which leads to the definition of the linearized strain tensor $\eps=\eps(\vec{u})=\frac{1}{2}(\nabla\vec{u}+\nabla^\mT\vec{u})$. Away from the crack, the classical elastodynamic equation governs the kinematics of the body, which in absence of body forces reads
\begin{equation} \label{eq:elastogriffith}
\begin{aligned}
& \rho\ddot{\vec{u}}=\div\sig\quad\text{in $\Omega\setminus\Gamma$} \\
& \sig\vec{n}=\vec{F}\quad\text{on $\partial\Omega_F$}
\end{aligned}
\end{equation}
where $\rho$ refers to the material density and $\vec{F}$ denotes the surface traction density applied on the subset $\partial\Omega_F$ of the boundary. The stress tensor $\sig=\tens{A}\eps$ admits an explicit expression via the use of Lamé coefficients
\[
\sig=\lambda\tr(\eps)\mathbb{I}+2\mu\eps
\]
with $\mathbb{I}$ the identity tensor of rank 2. Plugging this expression into the dynamic equilibrium equation in the bulk gives the Navier's equations of motion
\begin{equation} \label{eq:navier}
\rho\ddot{\vec{u}}=(\lambda+\mu)\nabla(\div\vec{u})+\mu\div(\nabla\vec{u})
\end{equation}
where $\div(\nabla\vec{u})$ denotes the vectorial Laplacian of $\vec{u}$. Suppose that the displacement is irrotational $\operatorname{rot}\vec{u}=\vec{0}$, then \eqref{eq:navier} reduces to
\[
\ddot{\vec{u}}=c_\mathrm{d}^2\div(\nabla\vec{u})
\]
where $c_\mathrm{d}=\sqrt{(\lambda+2\mu)/\rho}$ is the dilatational wave speed. On the other hand, considering equivoluminal waves that satisfy $\div\vec{u}=\vec{0}$ in \eqref{eq:navier}, we obtain
\[
\ddot{\vec{u}}=c_\mathrm{s}^2\div(\nabla\vec{u})
\]
with $c_\mathrm{s}=\sqrt{\mu/\rho}$ denoting the shear wave speed. For a general wave evolution, it can be partitioned into a purely dilatational component and a purely shearing component, see for example \cite{Sternberg:1960aa}. Suppose that the crack evolution $t\mapsto\Gamma$ is known, the displacement time evolution problem can then completed by the Dirichlet boundary conditions of $\vec{u}$ prescribed on a subset $\partial\Omega_U$ of the boundary, as well as a set of initial conditions $(\vec{u}_0,\dot{\vec{u}}_0)$ defined on the initial cracked configuration $\Omega\setminus\Gamma_0$.

In presence of a crack $\Gamma$, the displacement and stress present a well-known $\mathcal{O}(r^{1/2})$ and $\mathcal{O}(r^{-1/2})$ asymptotic behaviors at the crack tip when the elastodynamic equation \eqref{eq:navier} is solved in the bulk of $\Omega\setminus\Gamma$. In the case of an in-plane fracture problem, these two fields admit the following near-tip form
\begin{equation} \label{eq:singularform}
\begin{aligned}
\vec{u}(r,\theta) &\approx \frac{K_\RN{1}(t)\sqrt{r}}{\sqrt{2\pi}\mu}\vec{\Theta}_\RN{1}(\theta,\dot{l})+\frac{K_\RN{2}(t)\sqrt{r}}{\sqrt{2\pi}\mu}\vec{\Theta}_\RN{2}(\theta,\dot{l})+\ldots \\
\sig(r,\theta) &\approx \frac{K_\RN{1}(t)}{\sqrt{2\pi r}}\vec{\Sigma}_\RN{1}(\theta,\dot{l})+\frac{K_\RN{2}(t)}{\sqrt{2\pi r}}\vec{\Sigma}_\RN{2}(\theta,\dot{l})
\end{aligned}
\end{equation}
where the $K's$ are the stress intensity factors\index{Stress intensity factors}. Compared to the quasi-static regime, the angular functions $\vec{\Theta}$'s and $\vec{\Sigma}$'s depend on the current crack speed. When the crack propagates $\dot{l}>0$, the near tip behaviors for the velocity and the acceleration fields develop the following steady state form
\begin{equation} \label{eq:steadystatecondition}
\dot{\vec{u}}(\vec{x})\approx -\dot{l}\nabla\vec{u}\vtau=\mathcal{O}(r^{-1/2})\qquad\text{and}\qquad\ddot{\vec{u}}(\vec{x})\approx -\dot{l}\nabla\dot{\vec{u}}\vtau=\mathcal{O}(r^{-3/2}).
\end{equation}
In particular, the asymptotic expansion of the velocity reads
\begin{equation} \label{eq:vasymp}
\dot{\vec{u}}(r,\theta)\approx\frac{\dot{l}K_\RN{1}(t)}{\sqrt{2\pi r}\mu}\vec{V}_\RN{1}(\theta,\dot{l})+\frac{\dot{l}K_\RN{2}(t)}{\sqrt{2\pi r}\mu}\vec{V}_\RN{2}(\theta,\dot{l}).
\end{equation}

We now turn to the governing equation of the crack growth in the Griffith's theory. Assume that the crack propagates currently in the direction $\vtau$. Based on the thermodynamic energy balance law, the rate of energy that flows into the crack region delimited by an arbitrary contour $C$ encircling the crack tip (see \cref{fig:crackedconfig}) can be evaluated by the following energy flux\index{Energy flux}
\begin{equation} \label{eq:energyflux}
F=\int_C\Bigl((\sig\vec{n})\cdot\dot{\vec{u}}+\bigl(\psi(\eps)+\kappa(\dot{\vec{u}})\bigr)\dot{l}(\vec{n}\cdot\vtau)\Bigr)\ds.
\end{equation}
where $\psi$ and $\kappa$ denote respectively the elastic energy density and the kinetic energy density and $\vec{n}$ is the normal vector pointing out of the contour $C$. The thickness of the body $\Omega$ is neglected and quantities are defined per unit thickness as usual for plane problems. The first term in \eqref{eq:energyflux} stands for the rate of work applied to the crack region inside $C$ while the second terms corresponds to the energy transport due to crack propagation. A detailed derivation of \eqref{eq:energyflux} can be found for example in \cite{Freund:1972,NakamuraShihFreund:1985}. From this energy flux, a dynamic energy release rate $G$ that\index{Energy release rate!Dynamic energy release rate} corresponds to the amount of energy released per unit crack extension can be defined by dividing \eqref{eq:energyflux} by the current crack velocity $\dot{l}$ and taking a contour thank shrinks onto the crack tip. It is physically meaningful since near the crack tip the energy flux is indeed path-independent due to the steady state condition \eqref{eq:steadystatecondition}. If $r$ denotes the maximum distance of $C$ to the crack tip, we have
\begin{equation} \label{eq:Jdyn}
G=\lim_{r\to 0}\int_{C_r}\vec{J}\vec{n}\cdot\vtau\ds\quad\text{with}\quad\vec{J}=\Bigl(\psi\bigl(\eps(\vec{u})\bigr)+\kappa(\dot{\vec{u}})\Bigr)\mathbb{I}-\nabla\vec{u}^\mT\sig.
\end{equation}
It can be regarded as the dynamic extension of the classical $J$-integral\index{J@$J$-integral} in the sense of \cite{Cherepanov:1967aa,Rice:1968aa}. By using the asymptotic near-tip behavior of the fields \eqref{eq:singularform} and \eqref{eq:vasymp}, the dynamic energy release rate can be related to the stress intensity factors via the following equation
\begin{equation} \label{eq:GasafunctionofK}
G=\frac{1-\nu^2}{E}\left(A_\RN{1}(\dot{l})K_\RN{1}(t)^2+A_\RN{2}(\dot{l})K_\RN{2}(t)^2\right)
\end{equation}
where $A$'s are two universal material-dependent functions \cite[p.~234]{Freund:1990}. This is the generalization of the Irwin's formula \cite{Irwin:1957aa} since when the crack is stationary $\dot{l}\to 0$, these two functions converge to 1\index{Irwin's formula}.

Due to the fundamental assumption of a Griffith crack \eqref{eq:StGriffith}, the amount of energy consumed per unit crack advance in the case of a sharp-interface surface is simply $\gc$, a material constant. The stress-free condition $\sig\vec{n}=\vec{0}$ is found on the crack lip. Owing to the energy balance of the cracked body, when the crack propagates the following Griffith criterion holds
\begin{equation} \label{eq:gtgc}
G=\gc
\end{equation}
The Griffith's criterion provides an equation of motion of the crack tip. Several consequence of \eqref{eq:gtgc} derived in \cite{Freund:1990} include but are not limited to
\begin{itemize}
\item The limiting speed for an in-plane crack is the Rayleigh wave speed $c_\mathrm{R}$ which depends only on the Poisson's ratio. It is defined as the root of the following Rayleigh equation
\begin{equation} \label{eq:rayleigh}
R(c)=4\alpha_\mathrm{d}\alpha_\mathrm{s}-(1+\alpha_\mathrm{s}^2)^2=0
\end{equation}
where $\alpha_\mathrm{d}=\sqrt{1-c^2/c_\mathrm{d}^2}$ and $\alpha_\mathrm{s}=\sqrt{1-c^2/c_\mathrm{s}^2}$. Approximation methods exist to give an explicit expression of the Rayleigh wave speed, see for example \cite{RoyerClorennec:2007}. Its evolution as a function of $\nu$ is provided in \cref{fig:rayleigh}.
\begin{figure}[htbp]
\centering
\includegraphics[width=0.5\textwidth]{cR.pdf}
\caption{Rayleigh wave speed as a function of the Poisson's ratio. Comparison between the exact solution of \eqref{eq:rayleigh} and the approximation provided in \cite{RoyerClorennec:2007}} \label{fig:rayleigh}
\end{figure}

\item The limiting speed for a mode-III crack is the shear wave speed $c_\mathrm{s}$.
\end{itemize}

\subsection{Theoretical and experimental critiques}
From the physical point of view, the main drawbacks of the Griffith's theory as a modeling approach to dynamic brittle fracture concerns crack nucleation and crack path prediction, see \cite{FrancfortMarigo:1998} for a discussion on these points for the quasi-static Griffith's theory which applies also in the dynamic case. Remark however that through the introduction of inertia effects, the Griffith's theory accompanied with the dynamic energy release rate \eqref{eq:Jdyn} is able to account for the classically termed ``brutal'' or ``unstable propagation'' cases where cracks propagate at a velocity comparable to the material sound speed such that the quasi-static hypothesis no long holds. On the contrary, such propagations which may involve ``temporal'' discontinuities can not be considered by the quasi-static Griffith's theory, see \cite{FrancfortMarigo:1998}.

\paragraph{Nucleation} The Griffith criterion \eqref{eq:gtgc} based on the competition between the energy release rate and the material fracture toughness fails to predict crack nucleation from a body that lacks enough stress singularities, see \cite{Marigo:2010}. Moreover, it is known that the remote tensile stress $\sigma$ needed to initiate a pre-existing crack of length $l$ inside an infinite domain scales with $1/\sqrt{l}$. When $l$ is large this size effect is validated experimentally \cite{Griffith:1921}, however for small cracks $\sigma$ tends to infinity and it is surely not the correct behavior for real materials which possess a maximal stress. 

To circumvent the crack nucleation deficiency present in the classical Griffith's theory, several possibilities can be considered.
\begin{enumerate}
\item The first one concerns the introduction of a \emph{strength} criterion which bounds the maximal stress magnitude in the body. Still adopting a sharp-interface description of cracks, the cohesive zone\index{Cohesive zone model} model \cite{Barenblatt:1962,ElicesGuineaGomezPlanas:2002} falls into this category. It revisits the Griffith's modeling of cracks \eqref{eq:StGriffith} by providing a new description of the crack surface energy
\begin{equation} \label{eq:cohesiveSt}
\mathcal{S}=\int_{\Gamma}\phi(\llbracket\vec{u}\rrbracket)\ds
\end{equation}
where the potential $\phi$ characterizes the local material toughness that corresponds to a displacement jump $\llbracket\vec{u}\rrbracket$ on the crack lip. The derivative of $\phi$ gives then the traction acting on the crack lips. It regularizes the initial Griffith theory by introducing a critical/maximal stress $\sigma_\mc=\norm{\phi'(\vec{0})}$ that the material can support. When the displacement jump becomes sufficiently big, the potential $\phi$ converges to the Griffith fracture toughness $\gc$, corresponding to a completely open crack  portion free of stress traction.

\item Crack nucleation with the Griffith's surface energy \eqref{eq:StGriffith} can be predicted in the Variational Approach to Fracture through the use of global minimizations. It will be discussed in \cref{sec:fm98}.

\item Finally thanks to an evolution criterion of the damage or phase field variable, crack nucleation is also possible in the smeared modeling approaches to be discussed in \cref{sec:graddamage}.
\end{enumerate}

\paragraph{Path} We observe that the Griffith criterion \eqref{eq:gtgc} is just a scalar equation governing the temporal evolution of the crack arc length. It is due to the fundamental assumption of the Griffith's theory concerning the crack topology: a single crack surface that propagates along an arbitrary but given path without branching or other topology changes. Path prediction itself is not part of the Griffith's theory and must be determined by additional physics-motivated criteria.
\begin{itemize}
\item Concerning crack kinking\index{Kinking}, several models compete with each other: the Principle of Local Symmetry \cite{GolDstein:1974aa}, the $G$-max criterion \cite{Hussain:1974aa} and the $\sigma_{\theta\theta}$-max criterion \cite{Erdogan:1963aa}, for instance. Although they predict similar in-plane kinking angles indistinguishable from conventional experiments, they indeed give incompatible results from a theoretic point of view, see for example \cite{ChambolleFrancfortMarigo:2009} for a comparison between the PLS and the $G$-max criterion. Furthermore, kinking prediction in presence of a mode-III component becomes even more tedious \cite{Pham:2016aa}. These criteria were developed initially for quasi-static crack kinking problems, however they are also frequently used in dynamic situations \cite{GregoireMaigreRethoreCombescure:2007,HaboussaGregoireElguedjMaigreCombescure:2011}. A fully dynamic criterion that predicts crack kinking is still an on-going research subject \cite{Adda-Bedia:2003aa}.

\item As far as the crack branching is concerned, currently there exists only necessary indications or conditions for the macro or micro-branching phenomena within the framework of Griffith's theory \eqref{eq:gtgc}, see \cite{Ravi-ChandarKnauss:1984a,KatzavAdda-BediaArias:2007}. In \cite{Yoffe:1951}, Yoffe analyzes the stress distribution ahead of the crack tip in mode-I situations and finds that when the crack velocity exceeds approximately 60\% of the Rayleigh wave speed, the maximum $\sigma_{\theta\theta}$ stress is no longer situated in front of the crack tip, see \cref{fig:yoffe}. This could be regarded as a necessary condition which reflects a redistribution of the stress tensor after branching. However the critical speed found is strictly larger than the experimentally found one, where $v_\mathrm{c}\approx 0.4c_\mathrm{R}$, see \cite{FinebergMarder:1999}.
\begin{figure}[htbp]
\centering
\includegraphics[width=0.5\textwidth]{yoffe.pdf}
\caption{The hoop stress variation as a function of the angle with the propagation direction, for several velocities. The Poisson's ratio is taken to be 0.2} \label{fig:yoffe}
\end{figure}

Another approach is based on the Eshelby's energetic approach. According to \cite{Eshelby:1970}, crack branching is possible only if the crack has acquired enough energy so that after branching the Griffth criterion \eqref{eq:gtgc} holds separately for branched cracks. A limiting critical branching speed implies then a vanishing velocity of the branches. Based on this concept as well as the Principle of Local Symmetry \cite{GolDstein:1974aa}, authors of \cite{KatzavAdda-BediaArias:2007} derive a necessary condition of crack branching and predict a critical velocity $v_\mathrm{c}\approx 0.46c_\mathrm{R}$. This provides hence a better approximation of the experimentally found critical speed. Nevertheless crack branching phenomena contain microscopic effects \cite{Ravi-ChandarKnauss:1984} and are of 3-d nature \cite{FinebergMarder:1999}. Better understanding this dynamic instability is still an active on-going research subject, see \cite{BouchbinderGoldmanFineberg:2014,FinebergBouchbinder:2015}.
\end{itemize}

\paragraph{Propagation} The limiting crack speed (the Rayleigh wave speed for in-plane crack propagation, for example) is hardly observed in experiments, see \cite{Ravi-ChandarKnauss:1984a} where the terminal velocity is found to be only half of $c_\mathrm{R}$. It could be explained by the micro-branching phenomena described in \cref{sec:dynafrac}. When this dynamic instability phenomenon is suppressed, the Griffith criterion \eqref{eq:gtgc} predicts a crack evolution conforming to the experiment where the crack speed reaches indeed the theoretic limiting speed, see \cite{SharonFineberg:1999,FinebergBouchbinder:2015}. Otherwise, either we should no longer consider the crack surface as a sharp-interface but as an ensemble of micro-voids or micro-cranks along the main crack, \emph{i.e.} a smeared description of cracks, either the fracture toughness $\gc$ used in the Griffith criterion \eqref{eq:gtgc} should become velocity-dependent to account for more energy dissipation when the main crack propagates \cite{SharonGrossFineberg:1996}.

\subsection{Numerical aspects}
Numerically we need to account for displacement discontinuities across a sharp-interface crack in a finite element setting. Traditionally the crack can only be positioned along the element edges and remeshing is needed when the crack geometry evolves since it involves a topology change of the initial mesh. The accurate evaluation of local singularities is also a major issue with the classical $C^0$ finite element method, since the convergence rate is significantly bounded by these singularities. Since the advent of the eXtended-Finite Element Method \cite{Moes:1999aa} coupled with the level-set geometrical tracking framework \cite{Stolarska:2001aa}, cracks can now be freely incorporated into the numerical model based on a fixed mesh. Discontinuities and crack tip singularities can be taken into account in the local interpolation functions and the convergence rate is considerably improved. Nevertheless, it should be noted that the inherent limitations of the physical model as outlined above are still present.

\section{Variational Approach to Fracture} \label{sec:fm98}
The objective of the Variational Approach to Fracture\index{Variational Approach to Fracture} is to settle down a complete and unified brittle fracture theory within the Continuum Mechanics framework, which is capable of predicting the onset and the space-time evolution of sharp-interface cracks with possible complex topologies which the previous sharp-interface fracture theories fail to deliver.

While the pioneering paper \cite{FrancfortMarigo:1998} formalizes the mathematical ideas of the variational approach to fracture, another paper \cite{BourdinFrancfortMarigo:2000} indicates some possible directions concerning its effective numerical implementation in a finite element context. A comprehensive review paper \cite{BourdinFrancfortMarigo:2008} summarizes the characteristics of crack evolution predicted by the variational model, both for the Griffith's surface energy model \eqref{eq:StGriffith} and the cohesive description of cracks \eqref{eq:cohesiveSt}. These papers have also raised continuous interests in the mathematics communities toward a better mathematical precision and understanding of the model, see \cite{Negri:2010aa} for a review on different yet similar variational approaches to fracture.

The theory is initially proposed in a quasi-static setting. A formulation of a sharp-interface dynamic fracture model constitutes still an on-going research subject within the mathematics community, see \cite{Larsen:2010}.

\subsection{Crack evolution as an energy minimization movement} \label{sec:fm98sharp}
Griffith's theory is essentially based on an energetic competition between an structural energy release rate defined as the derivative of the potential energy with respect to the crack length, and a fracture toughness as a material constant characterizing macroscopic toughness. It can be readily transformed to an equivalent criterion based on energy minimization concepts, by using the total energy of the cracked body. However we recognize that the main drawbacks of the Griffith's theory (and other previous sharp-interface fracture model such as the cohesive zone model) lie in an \emph{a priori} assumption of the crack topology upon which the concept of energy release rates relies. It's the constraint that the Variational Approach to Fracture\index{Variational Approach to Fracture} pioneered by \cite{FrancfortMarigo:1998} aims to overcome.

A mere retranslation of Griffith's original idea, the variational formulation focuses on global energetic quantities of the cracked body where cracks $\Gamma$ are now considered to be an arbitrary closed subdomain of the body $\Omega$ in a $\mathrm{dim}$-dimensional configuration. Two energies can then be defined.

The first one concerns the surface energy
\begin{equation} \label{eq:SGamma}
\mathcal{S}(\Gamma)=\int_\Gamma\gc\D{\mathcal{H}^{\mathrm{dim}-1}}\,,
\end{equation}
where $\mathcal{H}^{\mathrm{dim}-1}$ denotes the $\mathrm{dim}-1$ dimensional Hausdorff measure, which yields a non-zero finite value when $\Omega$ is a sharp-interface surface or curve for 3-d or 2-d problems. It can be considered as a generalization of the Griffith's surface energy \eqref{eq:StGriffith} for a large class of crack topologies. Remark that cohesive effects can be as well accounted for, by replacing the $\gc$ constant by the potential $\phi$ in \eqref{eq:cohesiveSt}.

The second energy is the elastic stored energy of the body at equilibrium corresponding to the crack state $\Gamma$. It reads
\begin{equation} \label{eq:EGamma}
\mathcal{E}(\Gamma)=\inf_{\vec{u}\in\mathcal{C}}\int_{\Omega\setminus\Gamma}\psi\bigl(\eps(\vec{u})\bigr)\dx\,,
\end{equation}
where $\mathcal{C}$ is an appropriate function space that takes into account the Dirichlet boundary conditions prescribed on a portion of the boundary. Unilateral effects, see for example \cite{FrancfortMarigo:1998,AmorMarigoMaurini:2009}, may be as well considered. The total potential energy of the cracked body is the sum of the previous two energetic quantities in absence of body forces and surface tractions, which leads to
\begin{equation} \label{eq:PES}
\mathcal{P}(\Gamma)=\mathcal{E}(\Gamma)+\mathcal{S}(\Gamma).
\end{equation}

The Variational Approach to Fracture sees the crack evolution as a minimization movement of the total energy under an irreversibility condition to prevent self-healing of cracks. Formally, with an arbitrary temporal discretization $t^n$ with $1\leq n\leq N$, the governing equations are given by
\begin{equation} \label{eq:fm98}
\Gamma^{n-1}\subset\Gamma^n\,,\quad \mathcal{P}(\Gamma^n)\leq\mathcal{P}(\Gamma)\text{ for all $\Gamma\supset\Gamma^{n-1}$}\,,
\end{equation}
where $\Gamma^0$ denotes an initial given crack set. As a mathematical modeling of fracture, \eqref{eq:fm98} provides a unified and systematic approach to predict arbitrary crack evolution. In particular, crack nucleation from an initially perfectly sound structure possible even with the Griffith's surface energy \eqref{eq:StGriffith}, by comparing the elastic energy of the uncracked body and the total energy of a testing cracked configuration. Concerning the crack path, the variational approach retrieves exactly the Griffith's criterion \eqref{eq:gtgc} when the crack topology is constrained along a certain predefined surface. However its scope is even further and is fully capable of delivering a crack path of complex topologies without \emph{a priori} presuppositions, see \cite{BourdinFrancfortMarigo:2008}.

The attentive readers can not fail to realize the precise minimization structure underlying \eqref{eq:fm98}: that of global minimization. As a mathematical model of brittle fracture, it is indeed a convenient postulate to fulfill the objectives of predicting crack nucleation and path in a unified framework. Nevertheless, as is already indicated by the same authors in \cite{FrancfortMarigo:1998,ChambolleFrancfortMarigo:2009}, global minimization remains far from being a physics-based principle. The major concerns refer to the presence of possible energy barriers in the dependence of the total potential energy $\mathcal{P}$ with respect to the crack set $\Gamma$, see for example the point $B$ in \cref{fig:globalmin}.
\begin{figure}[htbp]
\centering
\includegraphics[width=0.6\textwidth]{global_min.pdf}
\caption{Variational approach to fracture mechanics: global or local minimization?} \label{fig:globalmin}
\end{figure}
Suppose that the crack at the previous time step is described by $\Gamma^{n-1}$. Due to the loading increment at the current time step $t^n$, the crack will evolve into a new configuration. Assume the behavior of the potential energy described in \cref{fig:globalmin}, according to global minimization \eqref{eq:fm98}, the crack will \emph{instantaneously} propagate through the body $\Omega$ described by the crack set $\Gamma^n_\mathrm{glo}$, even though by doing so it must pass across an energy barrier $B$ corresponding to an intermediate crack state which costs more energy than the previous time step, while actually attaining \emph{continuously} the final state $\Gamma^n_\mathrm{glo}$. The contradiction lies in the fact that while the crack tests every configuration possible to minimize the total potential energy, it does not know if there exists a physically feasible \emph{path} in the configuration space to arrives at that global minimum. In the same situation, a more intuitive solution will be to occupy the configuration $\Gamma^n_\mathrm{loc}$ which corresponds to a local minimum in the potential energy curve. This meta-stability concept is widely invoked in solid mechanics, see for example \cite{Nguyen:2000}. However as compared to global minimization postulate where crack path prediction is completely topology-free, local minimization calls for a precise definition of a \emph{distance} between two arbitrary crack states, \emph{i.e.} defining a certain metric or topology to the crack admissible space. If the topology is too trivial, no improvement can be made compared to the Griffith's theory. If it is too rich, mathematical difficulties may arise. The Variational Approach to Fracture based on local minimization is still an active on-going research topic, see for example \cite{CharlotteFrancfortMarigoTruskinovsky:2000,Negri:2010aa,ChambolleFrancfortMarigo:2009,Marigo:2010} for a discussion on this point.

\subsection{Elliptic regularization} \label{sec:ellipticregul}
Stated in the form of \eqref{eq:fm98}, the Variational Approach to Fracture can be regarded as a Free Discontinuity Problem where the unknown crack set introduces displacement discontinuities somewhere in the body, see \cite{Braides:1998aa} for a mathematical treatment on this topic. The total potential energy \eqref{eq:PES} which we minimize in the fracture mechanics context resembles the Mumford--Shah functional \cite{Mumford:1989aa} defined in the image segmentation context. This latter energy functional admits the so-called Ambrosio--Tortorelli\index{Ambrosio--Tortorelli regularization} regularization \cite{Ambrosio:1990aa} which can be regarded as an elliptic approximation converging to the initial model in a certain sense. Based on these two observations, authors of \cite{BourdinFrancfortMarigo:2000} introduce a mathematically-sound $\ell$-parametrized two-field functional approximation of \eqref{eq:PES}. As we shall see in \cref{sec:graddamage}, this lays down a theoretic foundation of the latter gradient damage approaches \cite{PhamAmorMarigoMaurini:2011} and phase-field models \cite{MieheHofackerWelschinger:2010}.

Their method consists of introducing an auxiliary continuous scalar continuous field defined on $\Omega$, which we note here $\alpha\in[0,1]$ anticipating its interpretation as a damage variable, representing the unknown location of cracks $\Gamma$. We adopt the convention that $\alpha(\vec{x})=1$ when $\vec{x}\in\Gamma$ and $\alpha(\vec{x})=0$ otherwise. The approximation of the discrete crack by a continuous field is parametrized by a small numerical parameter $\ell$, see \cref{fig:fiss3}. As $\ell\to 0$, the $\alpha$-field should collapse into a sharp-interface description of $\Gamma$. Remark that the introduction of such fields constitute exactly a smeared description of cracks which we reviewed in \cref{sec:discretemodels}. However here it is mainly motivated by a numerical implementation of the initial sharp-interface variational model \eqref{eq:fm98}.
\begin{figure}[htbp]
\centering
\includegraphics[width=0.55\textwidth]{fiss3.pdf}
\caption{Approximation of the discrete crack $\Gamma$ by a continuous field $\alpha$ parametrized by $\ell$} \label{fig:fiss3}
\end{figure}

The regularized energy functional of \eqref{eq:PES} reads
\begin{equation} \label{eq:Pualpha}
\mathcal{P}_\ell(\vec{u},\alpha)=\int_\Omega(1-\alpha)^2\psi\bigl(\eps(\vec{u})\bigr)\dx+\int_\Omega\gc\left(\frac{\alpha^2}{4\ell}+\ell\nabla\alpha\cdot\nabla\alpha\right)\dx
\end{equation}
from which we readily recognize its first part as a regularized version of the elastic energy \eqref{eq:EGamma} when using $\vec{u}$ at equilibrium at a fixed crack state $\alpha$. The second part of \eqref{eq:Pualpha} corresponds to an approximation of the Griffith crack surface energy \eqref{eq:SGamma} of the Ambrosio--Tortorelli type. Note however that an extension to the cohesive surface energy \eqref{eq:cohesiveSt} is also possible, see \cite{ContiFocardiIurlano:2015}. The evolutions of the $(\vec{u},\alpha)$ couple inherits from the original sharp-interface model \eqref{eq:fm98}: at every time step $t^n$, they achieve a global minimum of the total regularized potential functional $\mathcal{P}_\ell$, under the condition that the evolution of $\alpha$ is irreversible, \emph{i.e.} $\dot{\alpha}\geq 0$.

The convergence of this regularized model toward the initial model is made in a special setting involving $\Gamma$-convergence theories \cite{Braides:2002}. Specifically, one can prove that the global minimum attained in the regularized model \eqref{eq:Pualpha} converges to that of the original sharp-interface model \eqref{eq:PES}, when $\ell\to 0$. Using this fundamental result, the real crack evolution governed by the global minimization principle can thus be predicted through a $\ell$-parametrized approximation functional. 

Classical $C^0$ finite element methods can therefore be used to discretize in space the displacement $\vec{u}$ and the newly introduced scalar field $\alpha$. We observe that $\mathcal{P}_\ell$ is not convex in the couple $(\vec{u},\alpha)$. It is not a surprising result since the prediction of crack nucleation or further evolution essentially lies in its non-convexity. From a numerical point of view, we remark that the regularized functional $\mathcal{P}_\ell$ is separately convex in both variables, which leads intuitively to the use of an alternate minimization procedure proposed in \cite{BourdinFrancfortMarigo:2000}. At every time step, it consists of alternately minimizing $\mathcal{P}_\ell$ at fixed $\alpha$ and then at fixed $\vec{u}$ until numerical convergence upon an appropriate error norm and a tolerance. A detailed numerical analysis of the discretized model is performed in \cite{BourdinFrancfortMarigo:2008}. Recall finally that here the parameter $\ell$ serves as a purely numerical parameter which should be as small as possible. The Variational Approach to Fracture along with its numerical implementation based on the regularized functional \eqref{eq:Pualpha} has been applied to investigate unilateral effects \cite{PieroLancioniMarch:2007,AmorMarigoMaurini:2009}, drying \cite{MauriniBourdinGauthierLazarus:2013}, thin films debonding \cite{Baldelli:2014} and experimental validation problems \cite{MesgarnejadBourdinKhonsari:2014}.

\subsection{Extension to dynamics}
Formulating a variational model to dynamic fracture in the same spirit of the Francfort--Marigo model \eqref{eq:fm98} is still an active on-going research and has not reached a relatively mature state. Let us just note that a naive adaption of the (global) energy minimization principle \eqref{eq:fm98} by including the kinetic energy into the total potential energy \eqref{eq:PES} is doomed in dynamics, due to the hyperbolic nature of the dynamic wave equation. According to \cite{Larsen:2010}, a reasonable variational dynamic fracture model thats frees itself with an \emph{a priori} crack topology, \emph{i.e.} the Griffith's theory \eqref{eq:gtgc}, should verify the following conditions:
\begin{enumerate}
\item The displacement $\vec{u}$ satisfies the elastodynamic equation \eqref{eq:elastogriffith} in the uncracked bulk $\Omega\setminus\Gamma$ just like the Griffith case.

\item If the crack surface energy \eqref{eq:SGamma} is taken into account, then the system ensures an energy balance such that the rate of change of the total energy is equal to external power applied to the body via hard or soft devices.

\item Since a stationary crack state (accompanied by a corresponding displacement evolution) always verifies the previous two conditions, we should add an additional crack evolution criterion enforcing the crack to propagate if it is able to.
\end{enumerate}

It is exactly the third item that raises mathematical and mechanical modeling difficulties. Observe again that at an arbitrary time step a separate minimization of the total energy with respect to the crack does not make sense in such dynamic setting, since the elastic energy (and the kinetic energy) is determined by the displacement (and the velocity) field defined by the current crack state and is not affected by a virtual testing crack variation. Several examples of the third item are given in \cite{Larsen:2010}. A particular crack evolution law that merits mentioning is the following Maximal Dissipation (MD) condition
\begin{enumerate}
\setcounter{enumi}{2}
\item If a testing crack evolution $t\mapsto K_t$ accompanied with its corresponding displacement evolution verifies 1. and 2., and if $K_t\supset\Gamma_t$ for all $t$, then the real crack evolution is given by $\Gamma_t=K_t$ for all $t$.
\end{enumerate}
It can be seen that (MD) furnishes the worst crack scenario possible with respect to set inclusion, which resembles global minimization. However existence theories of such evolutions from a mathematical point of view are still absent and call for on-going researches.

On the contrary, an extension to the dynamic setting is indeed possible for the regularized fracture model \eqref{eq:Pualpha}. A separate minimization of the total energy is possible due to the fact that the elastic energy depends explicitly on the scalar field $\alpha$. Based on this observation, authors of \cite{Bourdin:2011} propose the following dynamic regularized fracture model:
\begin{enumerate}
\item $\vec{u}_t$ satisfies the wave equation in the bulk $\Omega$ with an elastic energy indicated by \eqref{eq:Pualpha}, \emph{i.e.} with a stress tensor modulated by $(1-\alpha)^2$.

\item $\alpha_t$ realizes a separate minimization of the regularized functional $\alpha\mapsto\mathcal{P}_\ell(\vec{u},\alpha)$ under the condition that $\alpha_t$ is a non-decreasing function of time.
\end{enumerate}
From a formulational point of view, these two equations can be embedded into a generalized action integral $\mathcal{A}_\ell(\vec{u},\alpha)$ of which the precise definition will be given in \cref{sec:formulation}. At a certain given parameter $\ell$, this model produces satisfactory crack nucleation, propagation and branching phenomena and seems to have a certain link with the Griffith's theory \eqref{eq:gtgc}. Recall that the regularized functional \eqref{eq:Pualpha} is obtained in a \emph{top-down} approach since it is considered as an approximation of the sharp-interface functional \eqref{eq:PES}. The $\ell$ parameter is of purely numerical nature and should be chosen as small as possible. On the contrary, the dynamic regularized fracture model \cite{Bourdin:2011} here symbolized by $\mathcal{A}_\ell$ is directly obtained by adapting the original regularized one $\mathcal{P}_\ell$. On the one hand, the role played by this internal length $\ell$ is not yet clear. On the other hand, the sharp-interface model in the limit $\ell\to 0$ of $\mathcal{A}_\ell$ is neither obvious, see \cite{Bourdin:2011}. The dynamic extension of the original Variational Approach to Fracture \cite{BourdinFrancfortMarigo:2008} can be summarized in \cref{fig:fourmodels}.
\begin{figure}[htbp]
\centering
\[
\begin{CD}
\mathcal{P}(\Gamma) @>\text{dynamics}>> \text{?} \\
@VV\text{$\ell$-regul.}V @AA{\ell\to 0}A \\
\mathcal{P}_\ell(\vec{u},\alpha) @>\text{dynamics}>> \mathcal{A}_\ell(\vec{u},\alpha)
\end{CD}
\]
\caption{In the quasi-static setting the regularized two-field fracture model \eqref{eq:Pualpha} is regarded as an approximation of the sharp-interface Francfort--Marigo model \eqref{eq:fm98} and convergence is achieved when $\ell\to 0$. In the dynamic case, we merely know how to add inertial effects into $\mathcal{P}_\ell$ to obtain a dynamic regularized model $\mathcal{A}_\ell$. Formulating a precise sharp-interface dynamic fracture model or investigating the limit of $\mathcal{A}_\ell$ as $\ell\to 0$ are at present a formidable challenge both from the mechanics and mathematics point of view.} \label{fig:fourmodels}
\end{figure}

\section{Gradient Damage Modeling of Cracks} \label{sec:graddamage}
In this section we give some general remarks of a gradient-damage modeling of cracks regarded as a smeared description approach to fracture. We motivate the presence of the damage gradient by providing two interdependent interpretations. A literature study on current phase-field models of dynamic brittle fracture then raises the main objectives of the present work.

\subsection{Interpretation as a phase-field approach to fracture}
The current formulation of the gradient damage model in the rate-independent evolution framework in the sense of \cite{Mielke:2005} is achieved in \cite{PhamMarigo:2010,PhamMarigo:2010-1}. We refer the readers to \cite{PhamAmorMarigoMaurini:2011} and references therein for a thorough review of its variational and constitutive ingredients as well its properties especially when applied to brittle fracture. Details of the formulation is omitted and we will mainly motivate the construction of such damage models interpreted as a phase-field approach of fracture.

The scalar field $\alpha$ introduced in the elliptic regularization (see \cref{sec:ellipticregul}) of the Variational Approach to Fracture is now interpreted as a damage variable ranging from 0 (intact material) to 1 (totally damaged material). It can be now regarded as a genuine smeared description of cracks and no longer as a numerical artifact during the $\ell$-regularization.

The elastic-damage evolution is governed by several physics-motivated principles based on the definition of a potential energy $\mathcal{P}(\vec{u},\alpha)$ of the body $\Omega$. Adopting the notation used in a review paper \cite{PhamAmorMarigoMaurini:2011} of gradient damage models, in absence of external work potential, the potential energy $\mathcal{P}(\vec{u},\alpha)$ reads
\begin{equation} \label{eq:Pgd}
\mathcal{P}(\vec{u},\alpha)=\int_\Omega\left( a(\alpha)\psi\bigl(\eps(\vec{u})\bigr)+w(\alpha)+\frac{1}{2}w_1\ell^2\nabla\alpha\cdot\nabla\alpha\right)\dx
\end{equation}
where $\alpha\mapsto a(\alpha)$ and $\alpha\mapsto w(\alpha)$ denote two damage constitutive functions and $w_1=w(1)$. Here $\ell$ is a material scalar parameter which controls as well the \emph{band} of the damage region as can been seen from \cref{fig:fiss3}. The two-field $(\vec{u},\alpha)$ evolution is then governed by the following
\begin{definition}[Quasi-Static Gradient Damage Evolution Law]
\begin{enumerate}
\item[]
\item \textbf{Irreversibility}: the damage $t\mapsto\alpha_t$ is a non-decreasing function of time.

\item \textbf{Meta-stability}: the current state $(\vec{u}_t,\alpha_t)$ must be stable in the sense that for all $\vec{v}\in\mathcal{C}_t$ and all $\beta\in\mathcal{D}(\alpha_t)$, there exists a $\overline{h}>0$ such that for all $h\in[0,\overline{h}]$ we have
\begin{equation} \label{eq:qsstability}
\mathcal{P}\bigl(\vec{u}_t+h(\vec{v}-\vec{u}_t),\alpha_t+h(\beta-\alpha_t)\bigr)\geq\mathcal{P}_t(\vec{u}_t,\alpha_t).
\end{equation}
The spaces $\mathcal{C}_t$ and $\mathcal{D}(\alpha_t)$ denote respectively the kinematically admissible space taking into account the Dirichlet boundary conditions and the admissible space for damage evolution reinforcing in particular the irreversibility condition, see \cite{PhamAmorMarigoMaurini:2011}.

\item \textbf{Energy balance}: the only energy dissipation is due to damage such that we have the following energy balance
\begin{equation} \label{eq:qseb}
\mathcal{P}(\vec{u}_t,\alpha_t)=\mathcal{P}(\vec{u}_0,\alpha_0)+\int_0^t\left(\int_\Omega\sig_s\cdot\eps(\dot{\vec{U}}_s)\dx\right)\D{s}.
\end{equation}
\end{enumerate}
\end{definition}

The formulation is variational in essence, which guarantees generality of the approach in which specific material constitutive behaviors can be taken into account, see \cite{PhamMarigoMaurini:2011,PhamMarigo:2013}. Similar to the Variational Approach to Fracture, the complex crack evolution, including initiation, nucleation, propagation and branching, can be predicted within a unified framework. Several properties of the gradient damage model also rely on the fundamental variational principles on which the gradient damage model is based, see for example \cite{SicsicMarigo:2013,SicsicMarigoMaurini:2013}. It also permits a direct numerical implementation based on mathematical optimization methods, for example in \cite{AmorMarigoMaurini:2009,PhamAmorMarigoMaurini:2011}. The variational nature of damage models can be justified via the following two approaches.
\begin{enumerate}
\item On the one hand, thanks to the Drucker-Ilyushin postulate \cite{Marigo:2002}, one shows that the strain work of an elastic-damage material is necessarily a state function independent of the strain path undertaken, see for example \cite{PhamMarigo:2010}. The state function is the sum of the elastic energy density and a damage dissipation potential. The evolution laws for the $(\vec{u},\alpha)$ can thus be obtained by considering variations of the potential energy containing the integral of the previous state function over the body $\Omega$.

\item On the other hand, the gradient damage evolution laws can also be interpreted as a Generalized Standard Material (with an extension of the classical point-wise normality rule to the structural scale) in the sense of \cite{Halphen:1975aa} which is automatically variational in nature, see the work of \cite{LorentzAndrieux:1999,LorentzBenallal:2005}.
\end{enumerate}

The presence of the damage gradient $\nabla\alpha$ in the potential energy \eqref{eq:Pgd} highlights the non-locality of the model. Two interdependent interpretations can be given to motivate the construction of the gradient damage models. It is schematized in \cref{fig:twointerpretationdamage}.
\begin{figure}[htbp]
\centering
\[
\begin{CD}
\mathcal{P}_0(\vec{u},\alpha) @>\text{+$\nabla\alpha$}>> \mathcal{P}(\vec{u},\alpha) \\
@. @AA\text{$\ell$-regul.}A \\
@. \mathcal{P}(\Gamma)
\end{CD}
\]
\caption{The gradient damage model represented by the potential energy $\mathcal{P}$ admits two different interpretations: on the one hand, it can be considered as a regularization of local damage models  denoted by $\mathcal{P}_0$ (by setting $\ell=0$ in $\mathcal{P}$); on the other hand, it admits a rigorous link with the sharp-interface Francfort-Marigo variational model $\mathcal{P}$ defined in \eqref{eq:fm98}} \label{fig:twointerpretationdamage}
\end{figure}

\paragraph{1. Regularization of conventional damage models} As the name indicates, gradient damage models belongs to the general theory of continuum damage mechanics\index{Damage mechanics} initiated by the pioneer work of \cite{Kachanov:1958} and then formalized in the thermodynamic framework through the work of \cite{LemaitreChaboche:1978} and others. The damage variable $\alpha$ can be regarded as a macroscopic characterization of the effective area of the cross-section where defect takes place at a microscopic scale. Compared to these conventional damage models, the gradient damage approach \eqref{eq:Pgd} consists of introducing an additional damage gradient which \emph{regularizes} the conventional mathematically and mechanically ill-posed local strain-softening mechanism. Either in the quasi-static setting \cite{Benallal:1993} or in the dynamic setting \cite{Bazant:1985aa}, local models could produce extreme strain localization within a vanishing band. Hence material failure occurs without any energy dissipation. This implies a spurious mesh dependence of the finite-element results, since the mesh size determines the localization band and the energy dissipated in producing such strain localization. An additional characteristic length must be introduced through the use of some non-local operators in the material constitutive behavior or in the damage criterion, see \cite{LorentzAndrieux:2003} for a thorough review of several approaches. Different non-local models for example presented in \cite{PeerlingsBorstBrekelmansVree:1996,PeerlingsBorstBrekelmansGeers:1998,Comi:1999,Comi:2001,BorstRemmersNeedlemanAbellan:2004} can all be considered as a smeared approach to fracture.

According to \cite{LorentzAndrieux:2003}, the gradient penalty operator which consists of introducing the gradient of the concerned variable into the total energy of the body is favored compared to the integral-type convolution operator. The question remains as to whether the strain gradient (\cite{PeerlingsBorstBrekelmansVree:1996}) or the damage gradient (\cite{LorentzBenallal:2005,PhamAmorMarigoMaurini:2011}) is to be introduced. A tangible answer is provided in the work of \cite{LeMauriniMarigoVidoli:2015}, where the damage-regularized model should be preferred compared to the strain-regularized model. Indeed, in the latter case the damage variable may continue developing into a diffuse region surrounding the strain localization. Energy dissipation in such process is not physics-based from a fracture modeling point of view. On the contrary, in gradient damage models strain localization is accompanied by a stationary damage profile corresponding to a finite and definite energy dissipation due to crack nucleation, which can be interpreted as the material toughness. Finally we want to mention that the peridynamic approach as reviewed in \cite{Silling:2010aa}, as a strongly non-local continuum mechanics model, seems to regularize both the strain and the damage variables.

\paragraph{2. Improvement of the Variational Approach to Fracture} One can not fail to realize the similarity between the gradient damage potential energy \eqref{eq:Pgd} and that of the regularized fracture model \eqref{eq:Pualpha} initially proposed in \cite{BourdinFrancfortMarigo:2000} for Griffith's surface energy \eqref{eq:StGriffith}. The presence of the damage gradient can thus be considered as a \emph{hint} through such regularization process. The major difference resides in the fact that in the Variational Approach to Fracture the $\ell$-regularized potential \eqref{eq:Pualpha} is considered as a purely numerical artifact invented to solve the real minimization problem \eqref{eq:fm98}, while here the gradient damage approach is considered to be a genuine physical \emph{model} \emph{per se}. The $\ell$ parameter is considered here as a material parameter and hence possesses a finite value, contrary to the regularized fracture model where $\ell$ is purely numerical in nature and should be chosen as small as possible. The potential energy \eqref{eq:Pgd} captures the Ambrosio--Tortorelli regularization\index{Ambrosio--Tortorelli regularization} via a particular choice of the constitutive functions. Although based on rigorous $\Gamma$-convergence results in \cite{Braides:2002}, a certain link can be established between the gradient damage model and its underlying sharp-interface model as $\ell\to 0$ in a $\Gamma$-convergence sense. Compared to the Variational Approach to Fracture, here the sharp-interface model is merely considered as a limit of the gradient damage model. This interpretation is mainly motivated by the following facts:
\begin{itemize}
\item Recall that the minimization structure behind the Variational Approach to Fracture is based on global minimization (cf. \cref{sec:fm98sharp}). In solid mechanics, a meta-stability principle is a more physically grounded approach (\cite{Nguyen:2000}) and provides a selection criterion of a multitude of \emph{equilibrium} states in damage mechanics, see the work of \cite{ BenallalMarigo:2007}. The meta-stability principle \eqref{eq:qsstability} as used in the quasi-static gradient damage model compares thus only a variation of the current state within a neighborhood, and brutal jumps across energy barriers are prevented. Based on this local minimization principle, the gradient damage model is still capable of accounting for nucleation of complex crack networks \cite{BourdinMarigoMauriniSicsic:2014,SicsicMarigoMaurini:2013}

\item Through analytical \cite{PhamMarigoMaurini:2011,PhamMarigo:2013} and numerical analyses \cite{AmorMarigoMaurini:2009,PhamAmorMarigoMaurini:2011} of gradient damage models, the $\ell$ parameter, or the internal length, admits a material parameter interpretation. Compared to the Griffith's theory, the introduction of such length scale introduces additional physical size effects. Furthermore, the maximum stress $\sigma_\mathrm{m}$ that the material can sustain is directly related to this parameter. For certain damage constitutive models as $\ell\to0$, we have $\sigma_\mathrm{m}\to\infty$ and hence damage initiation is no longer possible with the meta-stability principle \eqref{eq:qsstability}. It is an expected result due to the $\Gamma$-convergence results. That's why we proposed to regard $\ell$ as a material parameter and its calibration should be performed based on experimental facts.
\end{itemize}

In summary, the gradient damage approach can be regarded as a genuine physical modeling of brittle fracture. The internal length parameter $\ell$ contributes to the macroscopic material and structural behavior of such models.

\subsection{Going dynamical}
The dynamic extension \cite{Bourdin:2011} of the regularized variational approach to fracture consists of formally replacing the static equilibrium by the dynamic one in a time-discrete setting. However there the gradient damage interpretation is not acknowledged and the functional approximations used present some mechanical issues (more precisely the lack of a pure elastic zone), as is also the case for other similar methods \cite{BordenVerhooselScottHughesLandis:2012,HofackerMiehe:2012,SchlueterWillenbuecherKuhnMueller:2014}. The scope of this thesis is thus focused on the further exploration of the dynamic gradient damage models. More precisely, we will formulate the dynamic problem in lines with the initial quasi-static formulation \cite{PhamMarigo:2010-1} and consider several tension-compression asymmetry models. Numerical implementation is then detailed covering the spatial and temporal discretization as well as some additional numerical issues therein. Some representative simulation examples will be then given to highlight some of its characteristics. In the first case the crack path will be enforced and we compare with the available analytical results to establish a connection of our gradient damage model with the dynamic Griffith model, hoping to provide some indications of the possible dynamic extension of the quasi-static theoretic results \cite{SicsicMarigo:2013}. We then investigate the quasi-static limit of our dynamic model when the loading speed tends to zero, by considering material inhomogeneities. In the second case we let the cracks choose its own path and we study firstly in an 1d setting interactions of the damage and the elastic waves to highlight some differences between the dynamic model and the quasi-static one. A simplified 2d setting (semi-infinite domain) will be used to investigate dynamic crack path prediction involving the mysterious branching and comparaison with experimental results \cite{Ravi-ChandarKnauss:1984a,Ravi-ChandarKnauss:1984b} will be made. Finally we give some industrial examples to roughly verify and validate this dynamic model with other numerical or experimental results \cite{SongWangBelytschko:2008}.

\subsection{Toward a better understanding of gradient damage models}
The links of this gradient damage model with the fracture model can be done in more mechanical languages by a separation of scale \cite{SicsicMarigo:2013,LorentzCuvilliezKazymyrenko:2011,LorentzCuvilliezKazymyrenko:2012}. 

Moreover, the tension-compression asymmetry observed in several brittle materials such as concrete can be easily modeled and incorporated into the gradient damage model without preoccupying with its $\Gamma$-convergence fracture model counterpart, as already did in \cite{AmorMarigoMaurini:2009,FreddiRoyer-Carfagni:2010,LancioniRoyer-Carfagni:2009}. This point is rather appreciated in dynamics, as the tensile and compressive waves propagate and reflect within the structure.
\begin{figure}[htbp]
\centering
\[
\begin{CD}
\mathcal{P}(\Gamma) @>\text{unilateral effects}>> \widehat{\mathcal{P}}(\Gamma) \\
@VV\text{$\ell$-regul.}V @AA{\ell\to 0?}A \\
\mathcal{P}(\vec{u},\alpha) @>\text{T-C asymetry}>> \widehat{\mathcal{P}}(\vec{u},\alpha)
\end{CD}
\]
\caption{xx} \label{fig:TCunilateral}
\end{figure}

The phase-field model studied in \cite{KarmaKesslerLevine:2001,HakimKarma:2009} constitutes another continuum and regularized approach for quasi-static and dynamic fracture problems. A specific damage-like\footnote{Indeed, the function $1-\phi$ can be seen as the damage using their definition of the phase field $\phi$.} scalar \emph{phase} field $\phi$ is introduced to continously separate the broken state $\phi=0$ and the sound one $\phi=1$. In a quasi-static setting, the governing equations for the displacement $\vec{u}$ and the phase fields $\phi$ can be obtained in a semi-variational way from the total energy $\mathcal{E}(\vec{u},\phi)$, which, using notations used in \cite{HakimKarma:2009}, gives
\begin{align}
0 &= \frac{\partial\mathcal{E}}{\partial\vec{u}}(\vec{u},\phi) \label{eq:PF_u} \\
\chi^{-1}\dot{\phi} &= -\frac{\partial\mathcal{E}}{\partial\phi}(\vec{u},\phi) \label{eq:PF_phi}
\end{align}
where \eqref{eq:PF_u} describes the static equilibrum of the body with potential diffuse or localized damage zones and \eqref{eq:PF_phi} is the standard Ginzburg-Landau equation with $\chi>0$ a kinetic or mobility \cite{KuhnMuller:2010} coefficient controlling the (physical) additional total energy disspation in the form of heat during the crack propagation, as can be seen by following (corrected) equation based on (13) in \cite{HakimKarma:2009}
\begin{equation} \label{eq:PF_dissipation}
\dot{\mathcal{E}}=-\chi\left(\frac{\partial\mathcal{E}}{\partial\phi}(\vec{u},\phi)\right)^2\leq 0.
\end{equation}

Note that the Griffith-like crack creation is the only dissipation mechanism in our gradient damage model and an energy balance condition is added in the formulation to ensure that all the elastic energy released will be used to supplement crack propagation. This is the major formulational difference between our model and the \emph{dissipative} phase field models because of \eqref{eq:PF_dissipation}. A parallel consequence of the appearance of a kinetic coefficient $0<\chi<\infty$ in \eqref{eq:PF_phi}, as discussed in \cite{Bourdin:2011}, is that an evolutionary parabolic equation \eqref{eq:PF_phi} governing the phase field is coupled with the elliptic static equilibrium problem \eqref{eq:PF_u}. Physically it means that the crack can evolve solely with a rate determined by $\chi$, even if the structure is in static equilibrium at $t=T$ with all external loading frozen for all $t>T$. With a physical time being introduced into the model (the dimension of the kinetic coefficient is $[\mathrm{T}]^{-1}$), the coupled system \eqref{eq:PF_u}-\eqref{eq:PF_phi} isn't well suited for quasi-static computations, as numerically the static equilibrium  $\vec{K}\vec{u}=\vec{F}$ should be combined with a specific time-stepping scheme (the explicit Euler scheme in \cite{HakimKarma:2009}) to integrate the evolution problem for the phase field.

Other differences exist and concern mainly the dependence of the total energy on the phase field $\phi\mapsto\mathcal{E}(\vec{u},\phi)$. Our formulation is general in the sense that the stiffness degradation $\alpha\mapsto a(\alpha)$ and the local damage dissipation $\alpha\mapsto w(\alpha)$ are only required to verify some physical properties based on which three particular constitutive laws are studied in this paper. As can be been in the subsequent numerical experiments, all these models are similar in essence and can be used to investigate brittle fracture problems. On the contrary, the dissipative phase field models \cite{KarmaKesslerLevine:2001,HakimKarma:2009} seem to favour a particular set of constitutive functions. As long as these functions verify the properties, they can be seen to be contained in our general gradient damage model.

%Highlight the differences with respect to existing phase-field methods w(alpha)=alpha^2

\begin{itemize}
\item Give an adequate variational framework for dynamic gradient damage models
\item Propose a large-strain extension in an explicit dynamics  context
\item Give a systematic analysis/better understanding of gradient damage models, not just use it to do real complex simulations
\item How it can be used to approximate brittle fracture
\item Use of gradient damage models to explore/understand dynamic fracture phenomena (size effect, branching, velocity effects, crack arrest\ldots)
\end{itemize}

\section*{Summary of this Chapter} \label{sec:summarychap1}
