%!TEX root=../main.tex
\chapter{Avant-propos}

\section*{Industrial Background}
In civil engineering, the mechanical resistance and integrity of the reinforced concrete structures are frequently evaluated through computer simulations. Under severe dynamic loading conditions (such as impact or explosion), the prediction of possible crack nucleation and further space-time evolution until ultimate structural failure calls for more advanced physical and numerical methods.



on se propose dans le cadre de cette thèse d'étudier le comportement du béton (ou assimilé)
par un modèle élastique endommageable en traction seulement dans le formalisme variationnel de la rupture [Francfort et Marigo, 98].

L'objet de la thèse est d'étendre ce formalisme régularisé par champ d'endommagement aux problèmes de dynamique transitoire traités avec un
algorithme explicite. La prise en compte des effets d'inertie nous permettra en particulier d'observer l'influence de la vitesse de chargement sur le trajet spatio-temporel
de fissures et sur l'apparition éventuelle de (multi-)bifurcation. Les études analytiques pourront donc être menées pour mieux comprendre le modèle.

L'implantation et les développements seront réalisés dans le code de dynamique rapide EuroPlexus co-développé par le CEA et EDF. Différentes stratégies et techniques
de la résolution du problème global déplacement-endommagement pourront être pensées : raffinement et déraffinement automatique du maillage, gestion du pas de temps,
parallélisme...

Ces dernières années, une nouvelle approche basée sur les concepts de champ de phase introduits
par Francfort et Marigo au début des années 2000 est apparue pour le traitement des matériaux
fragiles ou quasi fragiles sous chargements quasi statiques: elle consiste à représenter une fissure par
un champ de variable d’endommagement variant entre 0 lorsque le matériau est sain et 1 lorsque la
fissure est ouverte. L’endommagement ne progresse que si son augmentation contribue à libérer
une énergie élastique suffisante pour concentrer l’endommagement et fissurer le matériau.

L’objet de cette thèse est de développer cette approche dans un code de dynamique rapide. La
démarche consistera à s’appuyer sur le triptyque habituel : modélisation, analyse numérique, et
essais et se terminera par une étude de la résistance mécanique d’un élément structural sous impact.

L’objectif final étant de simuler le comportement des structures en béton armé, on choisit de
modéliser séparément le béton, les armatures et leur interaction à l’interface acier-béton. Le sujet
étant vaste, la thèse se limitera à l’amélioration de la modélisation du béton au moyen de l’extension
du formalisme de régularisation par gradient de champ d’endommagement aux problèmes de
dynamique transitoire traités avec un algorithme explicite. 

Les développements seront réalisés dans
le code de dynamique rapide EUROPLEXUS co-développé par le CEA et EDF. La refermeture des
fissures, inévitable lors des retours d’onde, devra être représentée. Le matériau béton sera assimilé à
un matériau élastique, endommageable en tension seulement. Si possible, le formalisme sera
modifié pour pouvoir représenter les effets de vitesse de déformations sur la résistance du béton.


Puisqu’un maillage fin est indispensable pour avoir des résultats corrects dans les zones de
fissuration, il est nécessaire de mettre en place une procédure de maillage adaptatif pour raffiner les
zones endommagées. On s’appuiera sur l’algorithme d’adaptation (raffinement-déraffinement) des
maillages non-conformes disponible dans EUROPLEXUS en accompagnant l'adaptation de maillage
par l'utilisation de la méthode de partitionnement disponible également dans ce code pour optimiser
la gestion des pas de temps pour les éléments de taille différente. En outre, le modèle devra être
pensé dans une optique de calcul parallèle.

valider le modèle et/ou
d’en préciser les limites en terme d’échelle de mécanisme de dégradation.
Pour finir, il testera la modélisation sur des essais d’impact sur des éléments structuraux simples
(poutre, dalle) déjà réalisés.




\section*{Research Background and Outline}
From a modeling point of view, the present work concerns the formulation of mathematical and physical models of the physical phenomenon in an industrial context. Due to the complexity of the problem, numerical simulation is also needed to provide an approximative solution of the previous theoretical models. To ensure the faithfulness of the numerically discretized computer model with respect to the theoretical one, the \emph{verification} step should be first carried out in terms of numerical convergence properties. Finally, \emph{validation} of the physical and the numerical models will be achieved via the comparison between simulation results and experimental observations. The procedures involved in the present work are indicated in bold in \cref{fig:vv}.
\begin{figure}[htbp]
\centering
\[
\xymatrix{
\text{\textbf{Mathematical modeling}} \ar[d]_{\text{\textbf{Verification}}} \ar[rd]^{\text{Theoretic validation}} & \\
\text{\textbf{Numerical implementation}} \ar[r]_{\text{\textbf{Numerical validation}}} & \text{Physical phenomenon}
}
\]
\caption{Verification and validation of theoretical and numerical models against physical observations. The procedures involved in the present work are indicated in bold} \label{fig:vv}
\end{figure}

Concretely, the mathematical modeling will be performed in the framework of solid continuum mechanics. It consists of a macroscopic constitutive 

In the present work, the dynamic fracture behaviors of brittle materials will be characterized by 

a macroscopic non-local constitutive model.




\cref{chap:introduction}
\cref{chap:graddama}
\cref{chap:numerics}
\cref{chap:simulation}
\cref{chap:conclusion}

\section*{Notation Conventions}
General notation conventions adopted in the present work are summarized as follows:
\begin{itemize}
\item Scalar-valued quantities will be denoted by italic Roman or Greek letters. It concerns not only the mathematical and physical constants such as the Young's modulus $E$ or a particular one-dimensional stress measure $\sigma$ but also the temporal and spatial dependence of such scalars. Several examples are the temporal evolution of the crack length $l$ and the spatial damage field $\alpha_t$.

\item Vectors and second-order tensors as well as their matrix representation will be represented by boldface letters. This concerns for example a particular material point in a three-dimensional body $\vec{x}$, the displacement field $\vec{u}_t$, the velocity field $\dot{\vec{u}}_t$ and the stress tensor at that point $\sig_t(\vec{x})$.

\item Higher order tensors will be indicated by sans-serif letters: the elasticity tensor $\tens{A}$ for instance.

\item Tensors are considered as linear operators and intrinsic notation is adopted. If the resulting quantity is not a scalar, the contraction operation will be written without dots, such as $\sig_t=\tens{A}\eps_t=\tens{A}_{ijkl}\eps_{kl}$.

\item Inner products between two tensors of the same order will be denoted with a dot, such as $\tens{A}\eps_t\cdot\eps_t=\tens{A}_{ijkl}\eps_{kl}\eps_{ij}$.

\item Time dependence of the involved quantity will be indicated by a subscript, like $\vec{u}:(t,\vec{x})\mapsto\vec{u}_t(\vec{x})$. In particular, the notation $\vec{u}_t$ is understood as the displacement field at a fixed time $t$, whereas $\vec{u}$ refers to the time evolution of the displacement field.
\end{itemize}