%!TEX root=../main.tex
\chapter{Avant-propos}

\section*{Research Background and Outline}
From a modeling point of view, the present work concerns the \emph{formulation} of mathematical models of the physical phenomenon in an industrial context. Due to the complexity of the problem, numerical simulation is also needed to provide an approximative solution of the previous theoretical models. To ensure the faithfulness of the numerically discretized computer model with respect to the theoretical one, the \emph{verification} step should be first carried out in terms of numerical convergence properties. Finally, \emph{validation} of the physical and the numerical models will be achieved via the comparison between simulation results and experimental observations. All these steps are covered in the present study.

In civil engineering, mechanical performance and integrity of the reinforced concrete structures are of paramount importance for safety. Severe transient dynamic loading conditions (such as impact or explosion) often lead to crack nucleation and its further space-time evolution in the most vulnerable area, which results in ultimate structural failure. A better understanding of the mechanics of defects would thus guide the civil engineers to optimize the dimensioning, the shape and the topology of the initial design. An accurate assessment of structural behaviors in presence of dynamic crack propagation calls for more advanced physical models and their corresponding efficient computer implementations. In this aspect, the present work contributes thus to an improvement of the existing modeling of fracture in industrial structures, both from theoretical and numerical approaches.

Numerical simulation of reinforced concrete structures requires in general a separate modeling of the concrete, the reinforcement and the steel-concrete interaction. Due to the broadness of the subject, we will only focus here on the fracture behaviors of concrete itself. A coupling with the existing steel reinforcement models and in particular the phenomenon of interfacial fracture will be throughly investigated in the future. The mechanical behaviors of concrete fall into the category of brittle materials. Defect evolution in these materials with dynamical or inertia effects are commonly studied in the branch ``dynamic fracture'' of physics of solids. Very little deformation is present away from the fractured region and the strain tensor is essentially localized along the crack band. Without loss of generality, the methodology, modeling and analyses described in the present work should apply to a large class of materials that can be characterized by such constitutive and fracture behaviors.

Concretely, the mathematical modeling of dynamic brittle fracture will be performed in the framework of solid continuum mechanics with the usual Cauchy stress as the main stress measure. Adopting an engineering approach, we concentrate on a macroscopic phenomenological characterization of the constitutive behavior of brittle materials in presence of fracture. In particular, the spatial and temporal evolution of strain localization in a brittle solid will be modeled by the gradient-damage approach that is gaining popularity in the recent years. It consists of introducing a new spatial scalar field $\alpha_t$ that indicates and tracks the localization of cracks. It can be considered as a damage variable since $\alpha_t=0$ refers to an intact material point whereas $\alpha_t=1$ stands for a totally damaged region, \emph{i.e.} a crack or a strain-localization area. Compared to other existing approaches of dynamic fracture, the advantage of such approaches lies in the crack path prediction with arbitrary crack topologies from a theoretic defect evolution modeling point of view. Its variational formulation also permits a direct and consistent numerical implementation both for two-dimensional and three-dimensional problems.

A brief bibliographical study of dynamic brittle fracture is provided in \cref{chap:introduction}. We describe first the kinematics and physics of fracture in brittle materials with inertia, since the objective consists of faithfully and efficiently charactering those phenomena. In order to motivate the present work and to define a research scope, several currently used physical modeling approaches are compared with respect to their aptitude to approximate brittle fracture. This includes the classical Griffith's theory, the variational approach to fracture originated from the pioneer work of \cite{FrancfortMarigo:1998} and the current gradient damage model formulated in the quasi-static setting \cite{PhamMarigo:2010-1}. Based on the literature investigation, the objectives of the present work can then be defined. They are classified depending on the methodology used (theoretical or numerical approach) and using the following four thematic subjects
\begin{itemize}
\item Going dynamical,
\item Bridging the link with phase field approaches,
\item Better understanding of gradient damage modeling of fracture, and
\item Experimental validation.
\end{itemize}
To facilitate the presentation, the main novelty brought by the present study will be summarized at the beginning of each section by using the above classification.

\cref{chap:graddama} regroups the main theoretical contribution of this work. It concerns first a dynamic extension of the previous quasi-static gradient damage model in a variationally consistent framework. In quasi-statics, static equilibrium and the crack evolution of a solid corresponds to a minimum of the potential energy functional. In dynamics, this principle is generalized using an augmented space-time action integral and the temporal evolution of the coupled $(\vec{u},\alpha)$ field is governed by the stationarity of the former. As we shall see in the sequel, the benefits originating directly from the variational nature of the formulation are multi-fold. In explicit dynamics in presence of violent loading conditions, finite rotations of fractured regions are often observed. We also propose a possible approach to incorporate geometrical nonlinearities through the introduction of the Hencky logarithmic strain. The concrete as well as other brittle materials are characterized by asymmetric behaviors in tension and in compression. Accounting for such effects is essential especially in dynamics due to wave reflections at the boundary. We then provide a systematic review of several existing approaches and carry out a theoretic study during a uniaxial traction/compression experiment. Finally we propose a theoretic exploration of the previous variational framework in the case when the damage band is localized along a spatially propagating path. A generalized Griffith criterion is obtained in the dynamic case that governs the temporal evolution of the gradient-damage crack tip. A separation of scales is then achieved by assuming that the internal length is small by comparison with the dimension of the body.

Then in \cref{chap:numerics}, we present an efficient numerical implementation of the theoretic model described in the previous chapter. We follow a typical decoupling of the spatial and temporal discretization of the original continuous model and described separately these two discretization procedures. Since the gradient damage approach consists of describing material constitutive behaviors inside the strain localization region, a relatively fine mesh is needed at least along the potential fracture path. In the present work, high computational needs will be overcome via parallel computing techniques. Efficiency of the numerical model is illustrated and demonstrated by a strong scaling analysis. In terms of final numerical implementation, we provide on the one hand an open-source Python implementation of dynamic gradient damage models based on the FEniCS Project, see \cite{LiMaurini:2015}. On the other hand, the development is also conducted in the industrial explicit dynamics software EPX.

\cref{chap:simulation} constitutes another main contribution of the present work through several well-chosen numerical experiments. These simulations are tailored to highlight specific properties of the dynamic gradient damage model during a complete defect evolution. A \emph{divide and conquer} strategy is adopted and different temporal and spatial phases or events of dynamic fracture are investigated independently: nucleation, initiation, propagation, arrest, kinking, branching, \ldots To facilitate the reading, the ordering of the chapter as well as the objectives of each experiment is first explained. The four thematic subjects initially devised are also used to classify these numerical simulations. Verification of the numerical discretized model is achieved through convergence studies and comparison with theoretical results. We also provide an experimental validation of the proposed model via correlations between numerical and experimental observations. Limitations of the present model/parameters are also given toward improved modeling of dynamic fracture.

Finally some concluding remarks are given in \cref{chap:conclusion}. It consists of a general overview of the gradient damage approach to dynamic fracture both from a theoretical formulation/analysis and numerical implementation/investigation point of view. The presentation is classified using the four thematic subjects given in \cref{chap:introduction}. Possible future work arising from the present study is also indicated. 

\section*{Notation Conventions}
General notation conventions adopted in the present work are summarized as follows:
\begin{itemize}
\item Scalar-valued quantities will be denoted by italic Roman or Greek letters. It concerns not only the mathematical and physical constants such as the Young's modulus $E$ but also the temporal and spatial dependence of such scalars. Several examples include a temporal evolution of the crack length $l$, a particular one-dimensional stress measure $\sigma$ and the spatial damage field $\alpha_t$.

\item Vectors and second-order tensors as well as their matrix representation will be represented by boldface letters. This concerns for example a particular material point in a three-dimensional body $\vec{x}$, the displacement field $\vec{u}_t$, the velocity field $\dot{\vec{u}}_t$ and the stress tensor at that point $\sig_t(\vec{x})$.

\item Higher order tensors will be indicated by sans-serif letters: the elasticity tensor $\tens{A}$ for instance.

\item Tensors are considered as linear operators and intrinsic notation is adopted. If the resulting quantity is not a scalar, the contraction operation will be written without dots, such as $\sig_t=\tens{A}\eps_t=\tens{A}_{ijkl}\eps_{kl}$.

\item Inner products between two tensors of the same order will be denoted with a dot, such as $\tens{A}\eps_t\cdot\eps_t=\tens{A}_{ijkl}\eps_{kl}\eps_{ij}$.

\item Time dependence of the involved quantity will be indicated by a subscript, like $\vec{u}:(t,\vec{x})\mapsto\vec{u}_t(\vec{x})$. In particular, the notation $\vec{u}_t$ is understood as the displacement field at a fixed time $t$, whereas $\vec{u}$ refers to the time evolution of the displacement field.
\end{itemize}