\appendix
\chapter{Detailed calculations}

\section{Calculation of the first-order action variation} \label{sec:calactionvariation}
We will carefully explore the first-order stability principle \eqref{eq:stability} by calculating the action variation with respect to arbitrary displacement and crack variations. The following easily established identities
\begin{align*}
\frac{\md}{\md l_t}\det\nabla\philt(\vec{x}^*) &= \det\nabla\philt(\vec{x}^*)\tr(\nabla\vtheta^*(\vec{x}^*)\nabla\philt(\vec{x}^*)^{-1})=\div\vtheta_t(\vec{x})\det\nabla\philt(\vec{x}^*), \\
\frac{\md}{\md l_t}\nabla\philt(\vec{x}^*)^{-1} &= -\nabla\philt(\vec{x}^*)^{-1}\nabla\vtheta^*(\vec{x}^*)\nabla\philt(\vec{x}^*)^{-1}=-\nabla\philt(\vec{x}^*)^{-1}\nabla\vtheta_t(\vec{x}).
\end{align*}
will be used for all subsequent calculations.

The classical wave equation can be obtained by calculating the action variation with zero crack advance $\delta l=0$
\begin{multline*}
\mathcal{A}'(\vec{u}^*,l)(\vec{w}^*,0)=\int_I\D{t}\int_\domaini\Big(\tens{A}\bigl({\textstyle\frac{1}{2}}\nabla\vec{u}_t^*\nabla\philt^{-1}+{\textstyle\frac{1}{2}}\nabla\philt^{-\mathsf{T}}(\nabla\vec{u}_t^*)^\mT\bigr)\cdot\bigl({\textstyle\frac{1}{2}}\nabla\vec{w}_t^*\nabla\philt^{-1}+{\textstyle\frac{1}{2}}\nabla\philt^{-\mathsf{T}}(\nabla\vec{w}_t^*)^\mT\bigr)\det\nabla\philt \\
-\rho(\dot{\vec{u}}_t^*-\dot{l}_t\nabla\vec{u}_t^*\nabla\philt^{-1}\vtheta^*)\cdot(\dot{\vec{w}}_t^*-\dot{l}_t\nabla\vec{w}_t^*\nabla\philt^{-1}\vtheta^*)\det\nabla\philt\Big)-\mathcal{W}_t^*(\vec{w}^*_t) ,
\end{multline*}
which gives
\begin{multline} \label{eq:aprime1}
\mathcal{A}'(\vec{u}^*,l)(\vec{w}^*,0)=\int_I\D{t}\int_\domaint\bigl(\sig_t\cdot\eps(\vec{w}_t)+\rho\dot{l}_t\dot{\vec{u}}_t\cdot\nabla\vec{w}_t\vtheta_t \bigr)-\mathcal{W}_t(\vec{w}_t) \\
+\underbrace{\int_I\D{t}\int_\domaini\rho\frac{\md}{\md t}\bigl((\dot{\vec{u}}_t^*-\nabla\vec{u}_t^*\nabla\philt^{-1}\,\dot{l}_t\vtheta^*)\det\nabla\philt\bigr)\cdot\vec{w}_t^*}_R.
\end{multline}
To proceed, we observe that the real acceleration $\ddot{\vec{u}}_t$ can be obtained by differentiating \eqref{eq:v}
\begin{equation} \label{eq:acc}
\ddot{\vec{u}}_t(\vec{x})=-\nabla\dot{\vec{u}}_t(\vec{x})\dot{l}_t\vtheta^*(\vec{x}^*)+\frac{\md}{\md t}\left(\dot{\vec{u}}_t^*(\vec{x}^*)-\nabla\vec{u}_t^*(\vec{x}^*)\nabla\philt(\vec{x}^*)^{-1}\dot{l}_t\vtheta^*(\vec{x}^*)\right)
\end{equation}
where $\nabla\dot{\vec{u}}_t$ is the (Eulerian) velocity gradient. Using \eqref{eq:acc}, the last term above can be written
\begin{equation} \label{eq:aprime2}
\begin{aligned}
R &= \int_I\D{t}\int_\domaini\rho(\ddot{\vec{u}}_t\circ\philt+(\nabla\dot{\vec{u}}_t\circ\philt)\dot{l}_t\vtheta^*)\cdot\vec{w}_t^*\det\nabla\philt+\rho\dot{l}_t(\dot{\vec{u}}_t^*-\nabla\vec{u}_t^*\nabla\philt^{-1}\,\dot{l}_t\vtheta^*)\cdot\vec{w}_t^*\tr(\nabla\philt^{-1}\nabla\vtheta^*)\det\nabla\philt \\
&= \int_I\D{t}\int_\domaint\rho\ddot{\vec{u}}_t\cdot\vec{w}_t+\rho\dot{l}_t\nabla\dot{\vec{u}}_t\vtheta_t\cdot\vec{w}_t+\rho\dot{l}_t\dot{\vec{u}}_t\cdot\vec{w}_t\div\vtheta_t \\
&= \int_I\D{t}\int_\domaint\rho\ddot{\vec{u}}_t\cdot\vec{w}_t-\rho\dot{l}_t\dot{\vec{u}}_t\cdot\nabla\vec{w}_t\vtheta_t
\end{aligned}
\end{equation}
where an integration by parts in $\Omega\setminus\Gamma_t$ has been used on establishing the last equality. Regrouping \eqref{eq:aprime1} and \eqref{eq:aprime2}, we obtain thus the spatially weak dynamic equilibrium
\begin{equation} \label{eq:spatiallyweakdyneq}
\mathcal{A}'(\vec{u}^*,l)(\vec{w}^*,0)=\int_I\D{t}\int_\domaint\sig_t\cdot\eps(\vec{w}_t)+\rho\ddot{\vec{u}}_t\cdot\vec{w}_t-\mathcal{W}_t(\vec{w}_t).
\end{equation}
An integration by parts then gives the desired wave equation \eqref{eq:actionvariationzerocrackadvance} for the displacement.

We then evaluate the action variation with respect to arbitrary crack increment $\delta l$ but zero displacement variation
\begin{multline} \label{eq:aprime3}
\mathcal{A}'(\vec{u}^*,l)(\vec{0},\delta l)=\int_I\gc\cdot\delta l_t\D{t}+\int_I\delta l_t\D{t}\int_\domaint\bigl(\psi\bigl(\eps(\vec{u}_t)\bigr)-\kappa(\dot{\vec{u}}_t)\bigr)\div\vtheta_t-\sig_t\cdot(\nabla\vec{u}_t\nabla\vtheta_t)-\div(\vec{f}_t\otimes\vtheta_t)\cdot\vec{u}_t \\
-\underbrace{\int_I\D{t}\int_\domaini\rho(\dot{\vec{u}}^*_t-\dot{l}_t\nabla\vec{u}^*_t\nabla\philt^{-1}\vtheta^*)\cdot(-\nabla\vec{u}_t^*\nabla\philt^{-1}\vtheta^*\cdot\dot{\overline{\delta l}}_t+\dot{l}_t\nabla\vec{u}_t^*\nabla\philt^{-1}\nabla\vtheta^*\nabla\philt^{-1}\vtheta^*\cdot\delta l_t)\det\nabla\philt}_R.
\end{multline}
The last term can be written using integration by parts in the time domain
\[
R=\int_I\delta l_t\D{t}\int_\domaini\rho\frac{\md}{\md t}\bigl((\dot{\vec{u}}^*_t-\dot{l}_t\nabla\vec{u}^*_t\nabla\philt^{-1}\vtheta^*)\cdot(\nabla\vec{u}_t^*\nabla\philt^{-1}\vtheta^*)\det\nabla\philt\bigr)+\int_I\delta l_t\D{t}\int_\domaint\rho\dot{l}_t\dot{\vec{u}}_t\cdot\nabla\vec{u}_t\nabla\vtheta_t\vtheta_t,
\]
which gives
\begin{multline*}
R=\int_I\delta l_t\D{t}\int_\domaini\rho(\ddot{\vec{u}}_t\circ\philt+(\nabla\dot{\vec{u}}_t\circ\philt)\dot{l}_t\vtheta^*)\cdot(\nabla\vec{u}_t^*\nabla\philt^{-1}\vtheta^*)\det\nabla\philt \\
+\rho(\dot{\vec{u}}^*_t-\dot{l}_t\nabla\vec{u}^*_t\nabla\philt^{-1}\vtheta^*)\cdot(\nabla\dot{\vec{u}}^*_t\nabla\philt^{-1}\vtheta^*-\dot{l}_t\nabla\vec{u}^*_t\nabla\philt^{-1}\nabla\vtheta^*\nabla\philt^{-1}\vtheta^*)\det\nabla\philt \\
+\int_I\delta l_t\D{t}\int_\domaint\rho\dot{l}_t\dot{\vec{u}}_t\cdot\nabla\vec{u}_t\vtheta_t\div\vtheta_t+\rho\dot{l}_t\dot{\vec{u}}_t\cdot\nabla\vec{u}_t\nabla\vtheta_t\vtheta_t.
\end{multline*}
We obtain thus
\begin{multline*}
R=\int_I\delta l_t\D{t}\int_\domaint\rho\ddot{\vec{u}}_t\cdot\nabla\vec{u}_t\vtheta_t+\rho\dot{l}_t\nabla\dot{\vec{u}}_t\vtheta_t\cdot\nabla\vec{u}_t\vtheta_t+\rho\dot{l}_t\dot{\vec{u}}_t\cdot\nabla\vec{u}_t\vtheta_t\div\vtheta_t  \\
+\int_I\delta l_t\D{t}\int_\domaini\rho(\dot{\vec{u}}^*_t-\dot{l}_t\nabla\vec{u}^*_t\nabla\philt^{-1}\vtheta^*)\cdot(\nabla\dot{\vec{u}}^*_t\nabla\philt^{-1}\vtheta^*)\det\nabla\philt.
\end{multline*}
Differentiating \eqref{eq:grad} to obtain the material time derivative of the deformation gradient $(\md/\md t)(\nabla\vec{u}_t)$
\[
\frac{\md}{\md t}(\nabla\vec{u}_t(\vec{x}))=\nabla\dot{\vec{u}}_t^*(\vec{x}^*)\nabla\philt(\vec{x}^*)^{-1}-\dot{l}_t\nabla\vec{u}_t(\vec{x})\nabla\vtheta_t(\vec{x}),
\]
and with its definition
\[
\frac{\md}{\md t}(\nabla\vec{u}_t(\vec{x}))=\nabla\dot{\vec{u}}_t(\vec{x})+\nabla^2\vec{u}_t(\vec{x})\dot{l}_t\vtheta^*(\vec{x}^*)
\]
where $\nabla^2\vec{u}_t$ is the second gradient of the displacement field (a third-order tensor), we obtain
\begin{multline*}
R=\int_I\delta l_t\D{t}\int_\domaint\rho\ddot{\vec{u}}_t\cdot\nabla\vec{u}_t\vtheta_t+\rho\dot{l}_t\nabla\dot{\vec{u}}_t\vtheta_t\cdot\nabla\vec{u}_t\vtheta_t+\rho\dot{\vec{u}}_t\cdot\nabla\dot{\vec{u}}_t\vtheta_t+\rho\dot{l}_t\dot{\vec{u}}_t\cdot(\nabla^2\vec{u}_t\vtheta_t)\vtheta_t  \\
+\rho\dot{l}_t\dot{\vec{u}}_t\cdot\nabla\vec{u}_t\nabla\vtheta_t\vtheta_t+\rho\dot{l}_t\dot{\vec{u}}_t\cdot\nabla\vec{u}_t\vtheta_t\div\vtheta_t.
\end{multline*}
Using an integration by parts in the domain $\Omega\setminus\Gamma_t$ knowing that $\vtheta_t=\vec{0}$ on $\partial\Omega$ and $\vtheta_t\cdot\vec{n}=0$ on $\Gamma_t$ due to Assumption \ref{assum:velocityfield}
\[
\int_\domaint\rho\dot{l}_t\dot{\vec{u}}_t\cdot\nabla\vec{u}_t\nabla\vtheta_t\vtheta_t=-\int_\domaint\rho\dot{l}_t\dot{\vec{u}}_t\cdot\nabla\vec{u}_t\vtheta_t\div\vtheta_t+\rho\dot{l}_t\nabla\dot{\vec{u}}_t\vtheta_t\cdot\nabla\vec{u}_t\vtheta_t+\rho\dot{l}_t\dot{\vec{u}}_t\cdot(\nabla^2\vec{u}_t\vtheta_t)\vtheta_t,
\]
we get finally
\[
R=\int_I\delta l_t\D{t}\int_\domaint\rho\ddot{\vec{u}}_t\cdot\nabla\vec{u}_t\vtheta_t+\rho\dot{\vec{u}}_t\cdot\nabla\dot{\vec{u}}_t\vtheta_t
\]
which permits with \eqref{eq:aprime3} to deduce the desired equations \eqref{eq:actionvariationzerodisplacement} and \eqref{eq:Gt}.

\section{Local energy balance condition} \label{sec:ebcalc}
In this section we will use the global energy balance \eqref{eq:eb} to derive the equivalent local condition which gives the desired Griffith's law of motion \eqref{eq:griffithslaw} when combined with the local stability condition \eqref{eq:actionvariationzerodisplacement}. The Lagrangian density defined in \eqref{eq:action} is explicitly dependent on time solely through the external work potential \eqref{eq:externalworki}. Its total derivative can thus be given by
\begin{equation} \label{eq:lagrangianderivativetime}
\frac{\md\mathcal{L}}{\md t}=\frac{\partial\mathcal{L}}{\partial\vec{u}_t^*}\dot{\vec{u}}_t^*+\frac{\partial\mathcal{L}}{\partial\dot{\vec{u}}_t^*}\ddot{\vec{u}}_t^*+\frac{\partial\mathcal{L}}{\partial l_t}\dot{l}_t+\frac{\partial\mathcal{L}}{\partial\dot{l}_t}\ddot{l}_t+\frac{\partial\mathcal{L}}{\partial t}.
\end{equation}
Using the weak dynamic equilibrium \eqref{eq:spatiallyweakdyneq} and the fact that $\dot{\vec{u}}_t^*-\dot{\vec{U}}_t\in\mathcal{C}_0$, we have
\begin{equation} \label{eq:eulerlagrangeu}
\frac{\partial\mathcal{L}}{\partial\vec{u}_t^*}(\dot{\vec{u}}_t^*-\dot{\vec{U}}_t)-\frac{\md}{\md t}\frac{\partial\mathcal{L}}{\partial\dot{\vec{u}}_t^*}(\dot{\vec{u}}_t^*-\dot{\vec{U}}_t)=\vec{0}.
\end{equation}
Plugging \eqref{eq:eulerlagrangeu} into \eqref{eq:lagrangianderivativetime}, we obtain
\begin{equation} \label{eq:eblocal1}
\frac{\md\mathcal{L}}{\md t}=\frac{\md}{\md t}\left(\frac{\partial\mathcal{L}}{\partial\dot{\vec{u}}_t^*}\dot{\vec{u}}_t^*\right)+\frac{\partial\mathcal{L}}{\partial\vec{u}_t^*}\dot{\vec{U}}_t-\frac{\md}{\md t}\frac{\partial\mathcal{L}}{\partial\dot{\vec{u}}_t^*}\dot{\vec{U}}_t+\frac{\partial\mathcal{L}}{\partial l_t}\dot{l}_t+\frac{\partial\mathcal{L}}{\partial\dot{l}_t}\ddot{l}_t+\frac{\partial\mathcal{L}}{\partial t}.
\end{equation}

With all necessary temporal regularity, we note that the energy balance condition \eqref{eq:eb} can be equivalently written as
\begin{equation} \label{eq:eblocal2}
\frac{\md\mathcal{H}}{\md t}=\frac{\md}{\md t}\left(\mathcal{L}+2\mathcal{K}\right)=\frac{\md}{\md t}\left(\mathcal{L}-\frac{\partial\mathcal{L}}{\partial\dot{\vec{u}}_t^*}\dot{\vec{u}}_t^*-\frac{\partial\mathcal{L}}{\partial\dot{l}_t}\dot{l}_t\right)=\frac{\partial\mathcal{L}}{\partial\vec{u}_t^*}\dot{\vec{U}}_t-\frac{\md}{\md t}\frac{\partial\mathcal{L}}{\partial\dot{\vec{u}}_t^*}\dot{\vec{U}}_t+\frac{\partial\mathcal{L}}{\partial t}.
\end{equation}
Comparing \eqref{eq:eblocal1} and \eqref{eq:eblocal2}, we obtain the desired local energy balance condition
\[
\left(\frac{\partial\mathcal{L}}{\partial l_t}-\frac{\md}{\md t}\frac{\partial\mathcal{L}}{\partial\dot{l}_t}\right)\cdot\dot{l}_t=0.
\]

\section{Derivation of the elastodynamic equation at large displacement} \label{sec:elastodyna}
Denoting the variation $\vec{v}-\vec{u}$ by $\vec{w}$ and testing \eqref{eq:vi} with $\beta=\alpha$, we obtain after an integration by parts in the time domain supposing the displacement solution $\vec{u}$ is sufficiently regular in time
\[
\mathcal{A}'(\vec{u},\alpha)(\vec{w},0)=\int_I\D{t}\int_\Omega\bigl(\vtau\bigl(\eps(\vec{u}_t),\alpha_t\bigr)\cdot\eps'(\vec{u}_t)\vec{w}_t+\rho\ddot{\vec{u}}_t\cdot\vec{w}_t\bigr)-\overline{\mathcal{W}}_t(\vec{w}_t)=0
\]
where the Kirchhoff stress tensor $\vtau$ is given by \eqref{eq:tau} and $\eps'(\vec{u}_t)\vec{w}_t$ denotes the derivative of the Hencky strain in the direction of $\vec{w}_t$. The equality $\mathcal{A}'(\vec{u},\alpha)(\vec{w},0)=0$ follows given that $\mathcal{C}_t$ is a vector space. We will now use the work conjugacy condition satisfied by the Hencky's hyperelastic model \cite{XiaoChen:2002}
\begin{equation} \label{eq:workconjugacy}
\dot{w}_t=\vtau_t\cdot\vec{D}_t=\vtau_t\cdot\dot{\vec{h}}_t\implies\vtau_t\cdot\symgrad\vec{v}_t=\vtau_t\cdot\eps'(\vec{u}_t)\vec{v}_t
\end{equation}
where $\dot{w}_t$ is the rate of work per unit volume in the reference configuration and $\vec{D}_t$ is the stretching, \emph{i.e.} the symmetrized part of the velocity gradient $\nabla\vec{v}_t=\dot{\vec{F}}_t\vec{F}_t^{-1}$. Passing to the current configuration, we obtain thus
\[
\mathcal{A}'(\vec{u},\alpha)(\vec{w},0)=\int_I\D{t}\int_{\Omega_t}\bigl(\sig\bigl(\eps(\vec{u}_t),\alpha_t\bigr)\cdot\symgrad\vec{w}_t+\rho_t\ddot{\vec{u}}_t\cdot\vec{w}_t\bigr)-\mathcal{W}_t(\vec{w}_t)
\]
with $\rho_t=\rho/J_t$ the density in the current configuration and
\[
\mathcal{W}_t(\vec{w}_t)=\int_{\Omega_t}\vec{f}_t\cdot\vec{w}_t+\int_{\vphi_t(\partial\Omega_F)}\vec{F}_t\cdot\vec{w}_t
\]
the external power corresponding to the body forces $\vec{f}_t$ and the surface tractions $\vec{F}_t$ densities transformed to the deformed configuration \cite{Ciarlet:1993aa}. If we suppose further that the Cauchy stress $\sig_t=\sig\bigl(\eps(\vec{u}_t),\alpha_t\bigr)$ is sufficiently regular in space, an integration by parts in space along with the fundamental lemma of calculus of variations gives finally the desired elastodynamic equation \eqref{eq:waveeq}.