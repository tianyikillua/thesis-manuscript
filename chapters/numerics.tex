%!TEX root=../main.tex
\chapter{Numerical implementation}
\minitoc

This section is devoted to the numerical implementation of Model \ref{model:dynagraddama} applied to both explicit and implicit dynamic situations. The two damage constitutive laws \eqref{eq:at2} and \eqref{eq:at1} will be used. The elastic energy density split \eqref{eq:elasticTC} is also adopted to take into account tension-compression asymmetry. Our implementation can be considered as an adaptation of the previous work of \cite{PhamAmorMarigoMaurini:2011,Bourdin:2011,BordenVerhooselScottHughesLandis:2012} on quasi-static and dynamic phase-field like models for fracture. In this work the irreversiblity condition and the variational inequality \eqref{eq:vi} will be effectively implemented, which amounts to solve numerically the weak wave equation \eqref{eq:weakform} and the minimality principle for damage \eqref{eq:crackmin} at every time step. However it is shown in \cite{LarsenOrtnerSuli:2010} that the time-discrete numerical model will also balance energy as required in \eqref{eq:dyngdeb}, when the time increment becomes small. Their constructive proof makes use of the implicit Euler scheme used in \cite{Bourdin:2011}, however our experience suggests the same for the explicit central difference scheme as we describe below.

In this contribution the spatial and temporal discretization is as usual decoupled and will be discussed below separately. Space-time finite element methods will exploit fully the variational nature of the formulation \eqref{eq:vi} and can be considered as a possible improvement in the future.

\section{Spatial Discretization} \label{sec:spatial}
Classical finite element method is used to discretize in space the displacement $\vec{u}_t$ and the damage field $\alpha_t$ based on a same mesh $\Omega_h$. It should be ideally unstructured and uniform in mesh sizes otherwise some directions may be preferred when cracks propagate \cite{PhamAmorMarigoMaurini:2011}. The typical element size $h$ of the mesh should be preferably smaller with respect to the internal length $\ell$ in order to calculate correctly the damage band profile, the dissipation energy \eqref{eq:surface} and the material response inside the crack process zone of order $\mathcal{O}(\ell)$, see \cite{BourdinFrancfortMarigo:2008}.

\subsection{Displacement problem}
The displacement $\vec{u}_t$ and the damage field $\alpha_t$ will be both discretized with linear isoparametric finite elements. For two-dimensional plane problems, an arbitrary element possesses at every node 3 nodal degrees of freedom corresponding to 2 components of the displacement and 1 scalar value of the damage. The symbols $\uvec$ and $\dvec$ are used to denote the current global displacement and damage nodal vectors. Inside a given element $\Omega_\me\in\Omega_h$, their local nodal vectors $\uvec^\me$ and $\dvec^\me$ achieve an interpolation of the displacement and damage fields as well as their derivatives
\begin{align*}
& \vec{u}_t(\vec{x})=\vec{N}(\vec{x})\uvec^\me\quad\text{and}\quad\eps(\vec{u}_t)(\vec{x})=\vec{B}(\vec{x})\uvec^\me\, , \\
& \alpha_t(\vec{x})=\vec{N}_\alpha(\vec{x})\dvec^\me\quad\text{and}\quad\nabla\alpha_t(\vec{x})=\vec{B}_\alpha(\vec{x})\dvec^\me
\end{align*}
where $\vec{N}$'s and $\vec{B}$'s are respectively the interpolation and differentiation matrices applied on local nodal vectors $\uvec^\me$ and $\dvec^\me$ specific to the element $\Omega_\mathrm{e}$. \emph{Linear} elements are used due to their low computational cost. According to \cite{BourdinFrancfortMarigo:2008}, linear interpolations for the displacement and damage fields perform equally well compared to higher-order elements in terms of $\Gamma$-convergence of the dissipation energy. Finally this P1-P1 finite element discretization is not forbidden according to \cite{SimoneAskesPeerlingsSluys:2003}. The weak form of the wave equation \eqref{eq:weakform} after discretization in space reads
\begin{equation} \label{eq:waveeqsdis}
\vec{M}\ddot{\uvec}=\vec{F}_\mathrm{ext}-\vec{F}_\mathrm{int}(\uvec,\dvec)
\end{equation}
with $\vec{M}$ the classical lumped mass matrix, $\vec{F}_\mathrm{ext}$ the external force vector corresponding to the potential \eqref{eq:power} and $\vec{F}_\mathrm{int}$ the internal force vector assembled from the elementary vectors given by
\begin{equation} \label{eq:Fint}
\begin{aligned}
& \vec{F}_\mathrm{int}^\me=\int_{\Omega_\me}\vec{B}^\mT\sig\bigl(\vec{B}\uvec^\me,\vec{N}_\alpha\dvec^\me\bigr)\dx \\
&= \int_{\Omega_\me}\vec{B}^\mT\bigl(\mathsf{a}(\vec{N}_\alpha\dvec^\me)\sig_0^+(\vec{B}\uvec^\me)+\sig_0^-(\vec{B}\uvec^\me)\bigr)\dx.
\end{aligned}
\end{equation}
We note here that in explicit dynamics a residual stiffness $k_\mathrm{res}$ is not needed in the stiffness degradation function since no matrix inversion is needed, contrary to the implicit cases \cite{PhamAmorMarigoMaurini:2011,SchlueterWillenbuecherKuhnMueller:2014}.

\subsection{Damage problem}
Using the damage constitutive laws \eqref{eq:at2} and \eqref{eq:at1}, the total damageable energy $\mathcal{E}+\mathcal{S}$ is quadratic with respect to the damage vector $\dvec$ and \eqref{eq:crackmin} after spatial discretization reads 
\begin{equation} \label{eq:crackdis}
q_{\uvec}(\dvec)\leq q_{\uvec}(\underline{\vec{\beta}})\text{ for all $\underline{\vec{\beta}}$ that $0\leq\dvec\leq\underline{\vec{\beta}}\leq 1$}
\end{equation}
with the quadratic functional defined by
\begin{equation} \label{eq:Hb}
q_{\uvec}(\dvec)=\frac{1}{2}\dvec^\mT\vec{H}(\uvec)\dvec-\vec{b}(\uvec)^\mT\dvec.
\end{equation}
The Hessian matrix $\vec{H}$ and the second member vector $\vec{b}$ depend solely on the current deformation state $\uvec$ and hence are constant during the solving process of the damage problem. Their exact forms depend on the damage constitutive law used. For the \eqref{eq:at1} law for instance, they can be assembled from the elementary matrix and vector given by
\begin{align*}
\vec{H}^\me &= \int_{\Omega_\me}\left(2\psi_0^+(\vec{B}\uvec^\me)\vec{N}_\alpha^\mT\vec{N}_\alpha+2w_1\ell^2 \vec{B}_\alpha^\mT\vec{B}_\alpha\right)\dx, \\
\vec{b}^\me &= \int_{\Omega_\me}\left(2\psi_0^+\bigl(\vec{B}\uvec^\me\bigr)-w_1\right)\vec{N}_\alpha\dx.
\end{align*}

\section{Temporal Discretization} \label{sec:explicitnewmark}
Given an arbitrary discretization $(t^n)$ of the time interval of interest $I$ where the superscript $n$ denotes a quantity evaluated at the $n$-th time step, our objective here is to solve the spatially discretized wave equation \eqref{eq:waveeqsdis} coupled with the crack minimality condition \eqref{eq:crackdis} at these steps. In this contribution we consider dynamic fracture problems in brittle materials under impact-type loadings. In general the time scale involved is typically of order $\mathcal{O}(\SI{1}{ms})\ll\mathcal{O}(\SI{1}{s})$, thus in absence of a costly matrix inversion the explicit Newmark scheme with a lumped mass matrix is very suitable for this kind of situations. This time-stepping method is known to be conditionally stable. After spatial discretization the constraint prescribed on the current time increment $\Delta t$ is often determined by the CFL condition $\Delta t<\Delta t_\mathrm{CFL}=\min(h/c)$ where $h$ is the mesh size, $c$ is the material sound speed and the smallest value is chosen among all elements. In the calculation of the material sound speed, current damage state as well as the tension-compression split formulation is taken into account. Thus a totally damaged element under tension does not penalize the total computational time.

We note that after temporal discretization \eqref{eq:crackdis} reads
\begin{equation} \label{eq:crackstdis}
q_{\,\uvec^n}(\dvec^n)\leq q_{\,\uvec^n}(\underline{\vec{\beta}}) \text{ for all $\underline{\vec{\beta}}$ that $0\leq\dvec^{n-1}\leq\underline{\vec{\beta}}\leq 1$}
\end{equation}
where the Hessian matrix and the second member vector in \eqref{eq:Hb} are evaluated at $\uvec^n$. It can be translated to an effective minimization of the quadratic functional $q$ under the constraint that the current sought damage state $\dvec^n$ is pointwise non-decreasing with respect to its previous value. In absence of the temporal derivative of the damage field $\dot{\alpha}_t$, the above equation \eqref{eq:crackstdis} is not a genuine time evolution problem as the sole time dependence is introduced via the irreversibility condition. The current damage $\dvec^n$ can be accurately calculated as long as the current deformation state $\uvec^n$ is known.

In the time-continuous model the wave equation \eqref{eq:weakform} and the damage minimality condition \eqref{eq:crackmin} are coupled in the first-order stability principle \eqref{eq:vi}. After temporal discretization $\uvec$ and $\dvec$ evaluated at the last and current iterations are in general involved in an implicit fashion. In our case it turns out that the explicit Newmark scheme automatically decouples the time evolution system in $(\uvec,\dvec)$ and the two subproblems separately on $\uvec$ and on $\dvec$ can be independently solved one from the other at every time step. It is due to the fact that the current acceleration $\ddot{\uvec}^n$ can be obtained based on the current known deformation state $\uvec^n$ and the current damage $\dvec^n$ calculated from $\uvec^n$. The time-stepping scheme can be summarized in Algorithm \ref{algo:numericalmodel}. A bound-constrained minimization problem appears in every time iteration of the elastodynamic equation, and thus its efficient implementation is crucial. We observe that the initial damage is recomputed $\dvec^{-1}\mapsto\dvec^0$ in the step 2. The role of $\dvec^{-1}$ is to bring some \emph{a priori} knowledge of the damage field resulting from a previous calculation or more frequently to represent an initial crack $\dvec^{-1}=1$ on $\Gamma_0$. The initial step 2 thus renders it compatible with the initial displacement condition and the energy minimization structure.
\begin{algorithm*}[htbp]
\caption{Discretized numerical model of Model \ref{model:dynagraddama}.} \label{algo:numericalmodel}
\begin{algorithmic}[1]\linespread{1.2}\selectfont\normalsize
\State Given initial conditions $\uvec^0$, $\dot{\uvec}^0$ and $\dvec^{-1}$.
\State Reinitialize the damage $\dvec^0=\operatorname{arg min}q_{\,\uvec^0}(\cdot)$ subjected to constraints $0\leq\dvec^{-1}\leq\dvec^0\leq 1$.
\State Initialize the acceleration $\vec{M}\ddot{\uvec}^0=\vec{F}_\mathrm{ext}^0-\vec{F}_\mathrm{int}(\uvec^0,\dvec^0)$.
\For{every successive time step $n\geq 0$}
  \State Update $\dot{\uvec}^{n+1/2}=\dot{\uvec}^{n}+\frac{\Delta t}{2}\ddot{\uvec}^n$.
  \State Update $\uvec^{n+1}=\uvec^n+\Delta t\dot{\uvec}^{n+1/2}$.
  \State Update $\dvec^{n+1}=\operatorname{arg min}q_{\,\uvec^{n+1}}(\cdot)$ subjected to constraints $0\leq\dvec^n\leq\dvec^{n+1}\leq 1$.
  \State Update $\vec{M}\ddot{\uvec}^{n+1}=\vec{F}_\mathrm{ext}^{n+1}-\vec{F}_\mathrm{int}(\uvec^{n+1},\dvec^{n+1})$.
  \State Update $\dot{\uvec}^{n+1}=\dot{\uvec}^{n+1/2}+\frac{\Delta t}{2}\ddot{\uvec}^{n+1}$.
\EndFor
\end{algorithmic}
\end{algorithm*}

\section{Implementation}
Contrary to a pure explicit dynamic calculation, the presence of an implicit damage problem \eqref{eq:crackstdis} calls for a parallel linear algebra backend for manipulation of sparse matrices and vectors. As in the work of \cite{MesgarnejadBourdinKhonsari:2014}, the library PETSc \cite{PETSc:2015} is adopted since it provides also an efficient numerical scheme GPCG initially proposed in \cite{MoreToraldo:1991}, designed for quadratic bound-constrained minimization problems. It consists of several gradient projections to identify the \emph{active} nodes, \emph{i.e.} those either $\dvec^n=\dvec^{n-1}$ or $\dvec^n=1$. Then it applies the preconditioned conjugate gradient (PCG) method to minimize an unconstrained reduced problem of the \emph{free} variables, \emph{i.e.} those satisfying $\dvec^{n-1}<\dvec^n<1$. The incomplete Cholesky factorization preconditioner is applied block-wise onto each decomposed subdomain. Our simulation results demonstrate that this scheme is robust and efficient. Computational load is also well balanced \cite{BensonMcInnesMore:2001} in parallel computations based on domain decomposition.

Algorithm \ref{algo:numericalmodel} has been fully implemented in EUROPLEXUS, an explicit dynamics program dedicated to transient phenomena involving fluid-structure interaction \cite{EPX:2015}. Meanwhile an open-source implementation of the model is also available \cite{LiMaurini:2015}. It is based on the FEniCS Project \cite{LoggMardalWells:2012} for automated solution of PDE's.