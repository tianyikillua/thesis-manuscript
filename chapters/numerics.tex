%!TEX root=../main.tex
\chapter{Numerical implementation}
\minitoc

This chapter is devoted to the numerical implementation of the dynamic gradient damage model outlined in \cref{def:dynagraddama}. Semi-spatial discretization is performed in \cref{sec:spatial} based on the classical $C^0$ finite element method. The coupled two-field evolution problem is then discretized in the time domain based on the general $\beta$-Newmark scheme, see \cref{sec:timedis}. Some implementation issues are indicated in \cref{sec:implementation}.

In this contribution the spatial and temporal discretization is as usual decoupled and will be discussed below separately. Space-time finite element methods (\cite{HughesHulbert:1988}) exploits fully the variational nature of the formulation \eqref{eq:vi} and can be considered as a possible improvement in the future. The elastic energy density split \eqref{eq:elasticTC} is adopted to take into account tension-compression asymmetry. The two damage constitutive laws \eqref{eq:at2} and \eqref{eq:at1} will be used.  In this work the irreversiblity condition and the variational inequality \eqref{eq:vi} will be effectively implemented, which amounts to solve numerically the weak wave equation \eqref{eq:weakform} and the minimality principle for damage \eqref{eq:crackmin} at every time step.

\section{Spatial Discretization} \label{sec:spatial}
The spatial discretization described here constitutes a dynamic extension of the work of \cite{BourdinFrancfortMarigo:2008,PhamAmorMarigoMaurini:2011} for the regularized fracture model \eqref{eq:Pualpha} and the quasi-static gradient damage model outlined in \cref{def:qsgraddama}. A discussion on the large-displacement situations is indicated at the end. The thematic subjects covered here are thus summarized in \cref{tab:summsd}.
\begin{table}[htbp]
\centering
\caption{Thematic subjects covered in this section} \label{tab:summsd}
\begin{tabular}{ccccc} \toprule
& Going dynamical & $\alpha\leftrightarrow\phi$ & $\nabla\alpha\to\Gamma$ & Experimental validation \\ \midrule
Theoretics & & & & \\
Numerics & \rightthumbsup & & & \\ \bottomrule
\end{tabular}
\end{table}

A semi spatial discretization is performed from a mesh $\Omega_h$ of the original domain $\Omega$. It should be ideally unstructured and uniform in mesh sizes otherwise some directions may be preferred when cracks propagate, see \cite{Negri:1999,PhamAmorMarigoMaurini:2011} and references therein for a discussion on the ideal mesh properties in order to achieve a better modeling of fracture via gradient-damage approaches. In particular, the typical element size $h$ of the mesh should be preferably smaller with respect to the internal length $\ell$ in order to calculate correctly the damage band profile, the dissipation energy \eqref{eq:surface} and the material response inside the crack process zone of order $\mathcal{O}(\ell)$. It leads often to a computationally demanding calculation which requires massive parallelization.

The displacement $\vec{u}_t$ and the damage field $\alpha_t$ will be both discretized with classical $C^0$ isoparametric finite elements based on $\Omega_h$. For two-dimensional plane problems, an arbitrary element possesses at every node 3 nodal degrees of freedom corresponding to 2 components of the displacement and 1 scalar value of the damage, see for example \cref{fig:q4}.
\begin{figure}[htbp]
\centering
\includegraphics[width=0.25\textwidth]{q4.pdf}
\caption{Local interpolation of $(\vec{u},\alpha)$ based on nodal values inside the element} \label{fig:q4}
\end{figure}
The symbols $\uvec$ and $\dvec$ are used to denote the current global displacement and damage nodal vectors. Inside a given physical element $\Omega_\me\in\Omega_h$, their local nodal vectors $\uvec^\me$ and $\dvec^\me$ achieve an interpolation of the displacement and damage fields as well as their derivatives
\begin{equation} \label{eq:interpolation}
\begin{aligned}
& \vec{u}_t(\vec{x})=\vec{N}(\vec{x})\uvec^\me\quad\text{and}\quad\eps(\vec{u}_t)(\vec{x})=\vec{B}(\vec{x})\uvec^\me\, , \\
& \alpha_t(\vec{x})=\vec{N}_\alpha(\vec{x})\dvec^\me\quad\text{and}\quad\nabla\alpha_t(\vec{x})=\vec{B}_\alpha(\vec{x})\dvec^\me
\end{aligned}
\end{equation}
where $\vec{N}$'s and $\vec{B}$'s are respectively the interpolation and differentiation matrices. The Voigt notation is assumed for the strain tensor $\eps(\vec{u}_t)$ and the damage gradient $\nabla\alpha_t$, which are written as column vectors. \emph{Linear} interpolation is used due to its low computational cost in a dynamic calculation. According to \cite{BourdinFrancfortMarigo:2008}, linear elements for the displacement and damage fields perform equally well compared to higher-order elements in terms of $\Gamma$-convergence of the damage dissipation energy. Finally this P1-P1 finite element discretization is not forbidden according to \cite{SimoneAskesPeerlingsSluys:2003}.

\paragraph{Displacement problem}
Plugging \eqref{eq:interpolation} into the weak elastic-damage dynamic wave equation \eqref{eq:weakform} and adopting the same interpolation for the testing function $\vec{w}_t$, we obtain the following spatially discretized dynamic evolution equation
\begin{equation} \label{eq:waveeqsdis}
\vec{M}\ddot{\uvec}=\vec{F}_\mathrm{ext}-\vec{F}_\mathrm{int}(\uvec,\dvec)
\end{equation}
with $\vec{M}$ the consistent mass matrix, $\vec{F}_\mathrm{ext}$ the external force vector corresponding to the potential \eqref{eq:power} and $\vec{F}_\mathrm{int}$ the internal force vector. The mass matrix is assembled from the elementary mass matrices which reads inside $\Omega_\mathrm{e}$
\[
\vec{M}=\int_{\Omega_\mathrm{e}}\rho\vec{N}^\mT\vec{N}\dx.
\]
For explicit dynamic situations, this matrix is lumped $\vec{M}\mapsto \overline{\vec{M}}$ using the classical row-summing techniques to improve computational efficiency and spectral properties, see \cite{Hughes:2010}. The global internal force vector is constructed from its elementary counterparts which are given by
\begin{equation} \label{eq:Fint}
\begin{aligned}
& \vec{F}_\mathrm{int}^\me=\int_{\Omega_\me}\vec{B}^\mT\sig\bigl(\vec{B}\uvec^\me,\vec{N}_\alpha\dvec^\me\bigr)\dx \\
&= \int_{\Omega_\me}\vec{B}^\mT\bigl(\mathsf{a}(\vec{N}_\alpha\dvec^\me)\sig_0^+(\vec{B}\uvec^\me)+\sig_0^-(\vec{B}\uvec^\me)\bigr)\dx\,,
\end{aligned}
\end{equation}
where the stress expression \eqref{eq:stressposneg} that takes into account tension-compression asymmetry is used. In explicit dynamics a residual stiffness $k_\mathrm{res}$ is not needed in the stiffness degradation function since no matrix inversion is needed, contrary to the implicit cases \cite{PhamAmorMarigoMaurini:2011,SchlueterWillenbuecherKuhnMueller:2014}.

For simplex finite elements (triangular and tetrahedral elements) the Jacobian of the transformation (between the reference and the physical elements) as well as the differentiation matrix $\vec{B}$ are constant, thus \eqref{eq:Fint} can be integrated exactly using an effective stress
\begin{equation} \label{eq:FintSim}
\vec{F}_\mathrm{int}^\me=\abs{\Omega_\me}\vec{B}^\mT\sig_\mathrm{eff}=\abs{\Omega_\me}\vec{B}^\mT(\mathsf{a}_\mathrm{eff}\sig_0^++\sig_0^-)\,,
\end{equation}
where $\abs{\Omega_\me}$ denotes the measure (area of volume) of $\Omega_\me$ and $\mathsf{a}_\mathrm{eff}$ stands for the effective stiffness degradation
\[
\mathsf{a}_\mathrm{eff}=\frac{1}{\abs{\Omega_\me}}\int_{\Omega_\me}\mathsf{a}(\vec{N}_\alpha\dvec)\dx.
\]
Explicit expression of $\mathsf{a}_\mathrm{eff}$ can be readily computed and are given as follows.
\begin{itemize}
\item Triangular elements
\[
\mathsf{a}_\mathrm{eff}=\frac{1}{3}\left(\left(1-\frac{\alpha_1}{2}-\frac{\alpha_2}{2}\right)^2+\left(1-\frac{\alpha_2}{2}-\frac{\alpha_3}{2}\right)^2+\left(1-\frac{\alpha_1}{2}-\frac{\alpha_3}{2}\right)^2\right)
\]

\item Tetrahedral elements
\[
\mathsf{a}_\mathrm{eff}=1+\frac{1}{10}\sum_{i=1}^4\sum_{j=1}^4\alpha_i\alpha_j-\frac{1}{2}\sum_{i=1}^4\alpha_i
\]
\end{itemize}
The use of the effective stress $\sig_\mathrm{eff}$ improves computational efficiency and may also be served as an output of the current element.

For tensor product finite elements, a loop on the Gauss points is necessary and our experience suggests that 4 Gauss points for quadrilateral elements and 8 for hexagonal elements are sufficient. An effective stress can no longer be defined and the stress evaluated at these Gauss points will be used as the output variable.

\begin{remark}[Extension to large-displacements]
Numerical implications of the large-displacement extension proposed in \cref{sec:extensions} are discussed here. It is based on the explicit dynamics situations, where at the beginning of every time step the current configuration $\vphi_t(\Omega)$ is known and is updated from the last iteration. The Eulerian elastic-damage dynamic wave equation \eqref{eq:waveeq} will be naturally solved in the current deformed mesh obtained by constantly updating the mesh coordinates $\vec{x}_t=\vec{x}+\vec{u}_t(\vec{x})\in\Omega_t$.

This implies that the interpolation of the displacement and the damage \eqref{eq:interpolation}, as well as the calculation of the global internal force vector \eqref{eq:Fint} should be performed in the deformed element $\vphi_t(\Omega_\mathrm{e})$ and the interpolation and differentiation matrices $\vec{N}$'s and $\vec{B}$'s are constantly updated.

The Hencky logarithmic strain $\vec{h}_t$ defined in \eqref{eq:logstrain} replaces the use of the linearized strain in the calculation of the stress tensor in \eqref{eq:Fint}. Although the polar decomposition is indeed unnecessary by remarking that $\log\vec{V}_t=\frac{1}{2}\log\vec{B}_t$, spectral decomposition of the left Cauchy-Green tensor $\vec{B}_t=\vec{F}_t\vec{F}_t^\mT$ is unavoidable.
\end{remark}

\paragraph{Damage problem}
Plugging the interpolation expression \eqref{eq:interpolation} into the energy minimization principle \eqref{eq:crackmin} and using the elastic energy density split \eqref{eq:elasticTC} leads to the following spatially discretized variational inequality
\begin{equation} \label{eq:crackdis}
q_{\uvec}(\dvec)\leq q_{\uvec}(\underline{\vec{\beta}})\text{ for all $\underline{\vec{\beta}}$ that $0\leq\dvec\leq\underline{\vec{\beta}}\leq 1$}
\end{equation}
which involves the following discrete functional which reads inside each element
\begin{equation} \label{eq:qdiscrete}
q_{\uvec}^\mathrm{e}(\dvec)=\int_{\Omega_\mathrm{e}}\bigl(\mathsf{a}(\vec{N}_\alpha\dvec^\me)\psi_0^+(\vec{B}\uvec^\me)+w(\vec{N}_\alpha\dvec^\me)+w_1\eta^2\dvec^{\me\mT}\vec{B}_\alpha^\mT\vec{B}_\alpha\dvec^\me\bigr)\dx.
\end{equation}
For sophisticated damage constitutive laws such as that used in \cite{LorentzGodard:2011}, it is a nonlinear convex functional where the Hessian matrix depends on the damage state during numerical solving. Fortunately, using the damage constitutive laws \eqref{eq:at2} and \eqref{eq:at1}, this functional $q_{\uvec}(\dvec)$ is quadratic with respect to the damage vector $\dvec$ and is given by
\begin{equation} \label{eq:Hb}
q_{\uvec}(\dvec)=\frac{1}{2}\dvec^\mT\vec{H}(\uvec)\dvec-\vec{b}(\uvec)^\mT\dvec.
\end{equation}
The Hessian matrix $\vec{H}$ and the second member vector $\vec{b}$ depend solely on the current deformation state $\uvec$ and hence are constant during the solving process of the damage problem. Their exact forms depend on the damage constitutive law used:
\begin{itemize}
\item For the \eqref{eq:at1} law for instance, they can be assembled from the elementary matrix and vector given by
\begin{align*}
\vec{H}^\me &= \int_{\Omega_\me}\left(2\psi_0^+(\vec{B}\uvec^\me)\vec{N}_\alpha^\mT\vec{N}_\alpha+2w_1\eta^2 \vec{B}_\alpha^\mT\vec{B}_\alpha\right)\dx, \\
\vec{b}^\me &= \int_{\Omega_\me}\left(2\psi_0^+\bigl(\vec{B}\uvec^\me\bigr)-w_1\right)\vec{N}_\alpha\dx.
\end{align*}

\item For the \eqref{eq:at2} model, they are given by
\begin{align*}
\vec{H}^\me &= \int_{\Omega_\me}\left(2\left(\psi_0^+\bigl(\vec{B}\uvec\bigr)+w_1\right)\vec{N}_\alpha^\mT \vec{N}_\alpha+2w_1\eta^2\vec{B}_\alpha^\mT \vec{B}_\alpha\right), \\
\vec{b}^\me &= \int_{\Omega_\me}2\psi_0^+\bigl(\vec{B}\uvec\bigr)\vec{N}_\alpha.
\end{align*}
\end{itemize}
These quantities can be evaluated exactly for simplex finite elements. For quadrilateral and hexahedral elements, an approximated using a numerical integration with respectively 4 and 8 Gauss points will be used.

\begin{remark}[Extension to large-displacements]
According to \eqref{eq:crackmin}, the damage minimality condition is naturally formulated in the initial reference configuration. From a physical point of view, this corresponds to the fact that the damage dissipation energy is destined to measure the length or the area of cracks defined in the reference configuration. Technically the concerned energetic functional can be written in the deformed mesh (at the expense of additional unpleasant $\vec{F}_t$ and $J_t$ terms), however it is not necessary \cite{PieroLancioniMarch:2007}.

From a implementation point of view, it implies that the interpolation and differentiation matrices $\vec{N}$'s and $\vec{B}$'s that corresponds to the initial configuration should be stored and will be used throughout the calculation.
\end{remark}

\section{Temporal Discretization} \label{sec:timedis}
This section described a temporal discretization of the dynamic gradient damage model. The thematic subjects covered here are thus summarized in \cref{tab:summtd}.
\begin{table}[htbp]
\centering
\caption{Thematic subjects covered in this section} \label{tab:summtd}
\begin{tabular}{ccccc} \toprule
& Going dynamical & $\alpha\leftrightarrow\phi$ & $\nabla\alpha\to\Gamma$ & Experimental validation \\ \midrule
Theoretics & & & & \\
Numerics & \rightthumbsup & & & \\ \bottomrule
\end{tabular}
\end{table}

Consider an arbitrary discretization $(t^n)$ of the time interval $I=[0,T]$ where the superscript $n$ denotes a quantity evaluated at the $n$-th time step. We will mainly focus on the time stepping procedures bringing the current known states $(\uvec^n,\dot{\uvec}^n,\ddot{\uvec}^n,\dvec^n)$ to the next time step $(\uvec^{n+1},\dot{\uvec}^{n+1},\ddot{\uvec}^{n+1},\dvec^{n+1})$. In the time-continuous model the weak elastic-damage dynamic wave equation \eqref{eq:weakform} and the damage minimality condition \eqref{eq:crackmin} are coupled in the first-order stability principle \eqref{eq:vi}. After temporal discretization $\uvec$ and $\dvec$ evaluated at the last time step $t=t^n$ and the current time step $t=t^{n+1}$ are in general involved in an implicit fashion. However, we observe that the energy minimization principle \eqref{eq:crackmin} for damage is not a genuine time evolution problem since time dependence is only introduced via the irreversibility condition. After a temporal discretization \eqref{eq:crackdis} at time $t=t^{n+1}$ reads
\begin{equation} \label{eq:crackstdis}
q_{\,\uvec^{n+1}}(\dvec^{n+1})\leq q_{\,\uvec^{n+1}}(\underline{\vec{\beta}}) \text{ for all $\underline{\vec{\beta}}$ that $0\leq\dvec^n\leq\underline{\vec{\beta}}\leq 1$}
\end{equation}
where the Hessian matrix and the second member vector in \eqref{eq:Hb} are evaluated at $\uvec^{n+1}$. The equation \eqref{eq:crackstdis} can be interpreted as a numerical minimization problem of the quadratic functional $q$ under the irreversible constraint that the current sought damage state $\dvec^{n+1}$ is pointwise within the bound $(\dvec^n,1)$
\begin{equation} \label{eq:cracknummin}
\dvec^{n+1}=\operatorname{arg min}q_{\,\uvec^{n+1}}(\cdot)\text{ subjected to the constraints $0\leq\dvec^n\leq\dvec^{n+1}\leq 1$}.
\end{equation}
The next damage state $\dvec^{n+1}$ can thus be accurately calculated as long as the next deformation state $\uvec^{n+1}$ is known.

It remains then to specify the temporal discretization method used for the $\vec{u}$-problem. In this work we adopt the classical second-order accurate Newmark-$\beta$ integrator, which assumes the following time-stepping procedure
\begin{align}
\dot{\uvec}^{n+1} &= \dot{\uvec}^n+\frac{\Delta t}{2}\left(\ddot{\uvec}^n+\ddot{\uvec}^{n+1}\right)\,, \label{eq:un2} \\
\uvec^{n+1} &= \uvec^n+\Delta t\dot{\uvec}^n+\frac{1-2\beta}{2}\Delta t^2\ddot{\uvec}^n+\beta\Delta t^2\ddot{\uvec}^{n+1}. \label{eq:un1}
\end{align}
The implicit method $0<\beta\leq\frac{1}{2}$ may be suitable for intermediate situations between a quasi-static and an explicit dynamic calculation. In the present work it will be used to numerically investigate the quasi-static limit of the dynamic model, see \cref{sec:antiplane}. The explicit method $\beta=0$ with a lumped mass matrix $\overline{\vec{M}}$ should be preferred mainly in terms of computational efficiency for applications where the loading speed or the crack propagation speed is comparable to the material sound speed, since in general the time scale involved is typically of order $\mathcal{O}(\SI{1}{ms})\ll\mathcal{O}(\SI{1}{s})$. The explicit method is conditionally stable. The constraint prescribed on the current time increment $\Delta t$ is often determined by the CFL condition $\Delta t<\Delta t_\mathrm{CFL}=\min(h/c)$ where $h$ is the mesh size, $c$ is the material sound speed and the smallest value is chosen among all elements. This is not a very inconvenient feature since in presence of high geometrical and material nonlinearities even unconditionally stable implicit schemes need a small time increment comparable to $\Delta t_\mathrm{CFL}$. In the calculation of the material sound speed, the current damage state $\alpha_t$ as well as the tension-compression split formulation \eqref{eq:elasticTC} is taken into account. Thus a totally damaged element under tension does not penalize the total computational time.

The spatially discretized elastic-damage equation reads
\begin{equation} \label{eq:waveeqsdistdis}
\vec{M}\ddot{\uvec}^{n+1}=\vec{F}_\mathrm{ext}^{n+1}-\vec{F}_\mathrm{int}(\uvec^{n+1},\dvec^{n+1}).
\end{equation}
It can be seen that the determination of the the new acceleration $\ddot{\uvec}^{n+1}$ requires the knowledge of the new deformation state $\uvec^{n+1}$ which itself determines the new damage field at time $t=t^{n+1}$ via \eqref{eq:cracknummin}. For the implicit Newmark method $\beta\neq 0$, \eqref{eq:un1} can thus be regarded as a nonlinear equation in $\uvec^{n+1}$, where nonlinearity results from the irreversibility condition when minimizing the total energy \eqref{eq:cracknummin}. To decouple the $(\uvec^{n+1},\dvec^{n+1})$ problem, we use a staggered time-stepping procedure as used in \cite{Bourdin:2011,BordenVerhooselScottHughesLandis:2012,HofackerMiehe:2012} among others. The idea is to update the acceleration $\ddot{\uvec}^{n+1}$ while fixing the damage state at its \emph{previous} known value $\dvec^n$. Hence \eqref{eq:waveeqsdistdis} becomes
\begin{equation} \label{eq:waveeqsdistdis2}
\vec{M}\ddot{\uvec}^{n+1}\approx\vec{F}_\mathrm{ext}^{n+1}-\vec{F}_\mathrm{int}(\uvec^{n+1},\dvec^n).
\end{equation}
When a relatively small time-step is used, it is expected that the damage increment $\dvec^{n+1}-\dvec^n$ is bounded and the staggered time-discrete model will converge to the continuous one, cf. \cite{LarsenOrtnerSuli:2010}. Introducing the displacement prediction at time $t=t^{n+1}$
\[
\tilde{\uvec}^{n+1}=\uvec^n+\Delta t\dot{\uvec}^n+\frac{1-2\beta}{2}\Delta t^2\ddot{\uvec}^n\,,
\]
from \eqref{eq:un1} we obtain the equation for $\uvec^{n+1}$
\begin{equation} \label{eq:getuvecnp1}
\vec{M}\frac{\uvec^{n+1}-\tilde{\uvec}^{n+1}}{\Delta t^2}=\beta\bigl(\vec{F}_\mathrm{ext}^{n+1}-\vec{F}_\mathrm{int}(\uvec^{n+1},\dvec^n)\bigr).
\end{equation}
In general due to the use of tension-compression asymmetry models \eqref{eq:elasticTC}, it is a nonlinear equation which requires several Newton iterations and the evaluation of the current tangent stiffness. The time-stepping procedure for the dynamic gradient damage model based on the implicit Newmark-$\beta$ method in a prediction-correction form is summarized in \cref{algo:implicit}.
\begin{algorithm*}[htbp]
\caption{Staggered time-stepping procedure for the dynamic gradient damage model based on the implicit Newmark-$\beta$ method.} \label{algo:implicit}
\begin{algorithmic}[1]\linespread{1.2}\selectfont\normalsize
\For{every successive time step $n\geq 0$}
  \State Prediction $\tilde{\uvec}^{n+1}=u^n+\Delta t\dot{\uvec}^n+\frac{1-2\beta}{2}\Delta t^2\ddot{\uvec}^n$.
  \State Prediction $\tilde{\underline{\vec{v}}}^{n+1}=\dot{\uvec}^n+\frac{\Delta t}{2}\ddot{\uvec}^n$.
  \State Solve $\uvec^{n+1}$ via \eqref{eq:getuvecnp1}.
  \State Update $\ddot{\uvec}^{n+1}=(\uvec^{n+1}-\tilde{\uvec}^{n+1})/(\beta\Delta t^2)$.
  \State Update $\dot{\uvec}^{n+1}=\tilde{\underline{\vec{v}}}^{n+1}+\frac{\Delta t}{2}\ddot{\uvec}^{n+1}$.
  \State Solve $\dvec^{n+1}$ via \eqref{eq:cracknummin}.
\EndFor
\end{algorithmic}
\end{algorithm*}

In the explicit case when $\beta=0$, it turns out that the time evolution system in $(\uvec,\dvec)$ is automatically decoupled and the two subproblems separately in $\uvec^{n+1}$ and in $\dvec^{n+1}$ can be solved in a sequence at every time step, see \cref{fig:updatestate}.
\begin{figure}[htbp]
\normalsize
\centering
\[
\xymatrix{ & \vec{H}(\uvec^{n+1}) \ar[dr] & & & & & \\
\uvec^{n+1} \ar[ur] \ar[r] & \vec{b}(\uvec^{n+1}) \ar[r] & \dvec^{n+1} \ar[r] & \vec{F}_\mathrm{int}(\uvec^{n+1},\dvec^{n+1}) \ar[r] & \ddot{\uvec}^{n+1} \ar[r] & \dot{\uvec}^{n+3/2} \ar[r] & \uvec^{n+2} \\
& \dvec^n \ar[ur] & & \vec{F}_\mathrm{ext}^{n+1} \ar[ur] & \dot{\uvec}^{n+1/2} \ar[ur] & \uvec^{n+1} \ar[ur] &}
\]
\caption{Updating flow of the explicit time-stepping procedure for the discretized dynamic gradient damage model at the time step $t^{n+1}$}
\label{fig:updatestate}
\end{figure}
Indeed according to \eqref{eq:un1}, the current displacement state $\uvec^{n+1}$ can be directly determined from the previous kinematic states, which itself determines then the damage state via \eqref{eq:cracknummin}. Introducing the middle-step velocity
\[
\dot{\uvec}^{n+1/2}=\dot{\uvec}^n+\frac{\Delta t}{2}\ddot{\uvec}^n\,,
\]
the Newmark time-stepping \eqref{eq:un2} and \eqref{eq:un1} reads
\begin{align*}
\uvec^{n+1} &= \uvec^n+\Delta t\dot{\uvec}^{n+1/2}\,, \\
\dot{\uvec}^{n+1} &= \dot{\uvec}^{n+1/2}+\frac{\Delta t}{2}\ddot{\uvec}^{n+1}\,, \\
\dot{\uvec}^{n+1/2} &= \dot{\uvec}^{n-1/2}+\Delta t\ddot{\uvec}^{n}.
\end{align*}
The explicit time-stepping procedure for the discretized dynamic gradient damage model is then summarized in \cref{algo:explicit}.
\begin{algorithm*}[htbp]
\caption{Explicit time-stepping procedure for the dynamic gradient damage model.} \label{algo:explicit}
\begin{algorithmic}[1]\linespread{1.2}\selectfont\normalsize
\For{every successive time step $n\geq 0$}
  \State Update $\dot{\uvec}^{n+1/2}=\dot{\uvec}^{n}+\frac{\Delta t}{2}\ddot{\uvec}^n$.
  \State Update $\uvec^{n+1}=\uvec^n+\Delta t\dot{\uvec}^{n+1/2}$.
  \State Solve $\dvec^{n+1}$ via \eqref{eq:cracknummin}.
  \State Solve $\ddot{\uvec}^{n+1}$ via \eqref{eq:waveeqsdistdis} with the lumped mass matrix $\overline{\vec{M}}$.
  \State Update $\dot{\uvec}^{n+1}=\dot{\uvec}^{n+1/2}+\frac{\Delta t}{2}\ddot{\uvec}^{n+1}$.
\EndFor
\end{algorithmic}
\end{algorithm*}

The initialization phase for the above implicit and explicit time-stepping procedure is described in \cref{algo:init}. We observe that the initial damage is recomputed $\dvec^{-1}\mapsto\dvec^0$ in the step 2. The role of $\dvec^{-1}$ is to bring some \emph{a priori} knowledge of the damage field resulting from a previous calculation or more frequently to represent an initial crack $\dvec^{-1}=1$ on $\Gamma_0$. This step thus renders it compatible with the initial displacement condition and the energy minimization structure. Remark that in phase field approaches for example in \cite{BordenVerhooselScottHughesLandis:2012}, the initial crack is introduced via an initial history field $\mathcal{H}_0$ which is localized along the desired initial defect. It is due to the fact that the irreversible constraint is dropped during energy minimization \eqref{eq:crackmin} in their models.
\begin{algorithm*}[htbp]
\caption{Initialization for the implicit and explicit time-stepping procedure described in \cref{algo:implicit,algo:explicit}.} \label{algo:init}
\begin{algorithmic}[1]\linespread{1.2}\selectfont\normalsize
\State Given initial conditions $\uvec^0$, $\dot{\uvec}^0$ and $\dvec^{-1}$.
\State Reinitialize the damage $\dvec^0=\operatorname{arg min}q_{\,\uvec^0}(\cdot)$ subjected to constraints $0\leq\dvec^{-1}\leq\dvec^0\leq 1$.
\State Initialize the acceleration $\vec{M}\ddot{\uvec}^0=\vec{F}_\mathrm{ext}^0-\vec{F}_\mathrm{int}(\uvec^0,\dvec^0)$.
\end{algorithmic}
\end{algorithm*}

\begin{remark}[Temporal convergence]
Remark that only the irreversiblity condition $\dot{\alpha}_t\geq 0$ and the variational inequality \eqref{eq:vi} is effectively implemented. The former condition is reinforced during energy minimization \eqref{eq:cracknummin}. It is shown in \cite{LarsenOrtnerSuli:2010} that the time-discrete numerical model summarized by \cref{algo:implicit,algo:explicit} will also balance energy as required in \eqref{eq:dyngdeb}, when the time increment becomes small. Their constructive proof makes use of the implicit Euler scheme used in \cite{Bourdin:2011}, however our experience suggests the same for the implicit and explicit Newmark schemes.
\end{remark}

\begin{remark}[Approximation of the Hencky logarithmic strain]
An approximation of the Hencky strain based on the following remarkable property established in \cite{XiaoBruhnsMeyers:1997} can be used: within a proper corotational frame the rate of the Hencky strain equals the stretching tensor. An increment of the Hencky strain can thus be given by $\Delta\vec{h}^n\approx\symgrad(\Delta\vec{u}^n)$ where $\symgrad$ denotes the symmetrized gradient operator and the displacement increment $\Delta\vec{u}^n=\vec{u}^n-\vec{u}^{n-1}$ is written in the current deformed configuration. To have better accuracy, a second-order approximation based on the incremental Almansi strain is used
\[
\Delta\vec{h}^n\approx\frac{1}{2}\bigl(\nabla\Delta\vec{u}^n+\nabla^\mT\Delta\vec{u}^n-(\nabla^\mT\Delta\vec{u}^n)(\nabla\Delta\vec{u}^n)\bigr).
\]
Finally the current Hencky strain is incremented $\vec{h}^n=\vec{h}^{n-1}+\Delta\vec{h}^n$.
\end{remark}

\section{Implementation} \label{sec:implementation}
This section provides some details on the implementation of the discretized numerical model summarized by \cref{algo:implicit,algo:explicit}. The thematic subjects covered here are thus summarized in \cref{tab:summim}.
\begin{table}[htbp]
\centering
\caption{Thematic subjects covered in this section} \label{tab:summim}
\begin{tabular}{ccccc} \toprule
& Going dynamical & $\alpha\leftrightarrow\phi$ & $\nabla\alpha\to\Gamma$ & Experimental validation \\ \midrule
Theoretics & & & & \\
Numerics & \rightthumbsup & & & \\ \bottomrule
\end{tabular}
\end{table}

\subsection{Numerical solving of the damage problem}
In this work, the damage energy minimization problem \eqref{eq:cracknummin} will be directly solved at structural scale. It could be converted into an equivalent local problem defined for each element. This possibility requires introduction of Lagrange multipliers at every Gauss points and an identification between the nodal damage field and the damage defined at Gauss points, see \cite[p.~49]{Beaurain:2011}. The principal motivation of this kind of Lagriangian modeling is the ability in the software Code-Aster to impose more naturally the irreversibility condition, however the computational cost is higher due to additional unknown variables.

Consequently, \eqref{eq:cracknummin} can be regarded as a large scale bound-constrained minimization problem defined at the structural level. Due to the damage constitutive requirement, the problem is convex and gradient-based numerical algorithms can be applied. These methods calls for the evaluation of the objective functional \eqref{eq:Hb} as well as its first and second derivatives (Hessian matrix $\vec{H}$). An efficient storage and manipulation of these objects is essential especially in an explicit dynamics context.

Here the library PETSc is adopted due to its optimized efficiency, see \cite{PETSc:2015}. As a high-level C++/Fortran numerical linear algebra backend, PETSc provides in particular the \texttt{Vec} and the \texttt{Mat} classes for defining vectors and matrices. The actual storage format is not fixed in advance and it suffices to call the function \texttt{Vec/MatSetType} to affect a particular format. The computational arrays (the CSR format for example) can be extracted for output but in general all operations of these objects \texttt{Vec}/\texttt{Mat} are independent of the storage format used and thus transparent to the user. As an example, a unique interface is available to conduct a matrix-vector product
\begin{mdframed}[hidealllines=true,backgroundcolor=gray!20]
\begin{minted}[breaklines,breakanywhere]{fortran}
CALL MATMULT(H,X,GRAD,PETSCERR)
\end{minted}
\end{mdframed}
where the vector \texttt{GRAD} stocks the result of $\vec{H}\dvec$. PETSc is built on MPI and only one generic version exists which can be run either in sequential or parallel. The calling functions are almost the same in both cases, thus the code could be parallelized with minor interface changes.

Specifically the algorithm Gradient Projection Conjugate Gradient (GPCG) proposed in \cite{MoreToraldo:1991} is used to numerically solve \eqref{eq:cracknummin}. It is already implemented in the TAO package of PETSc and is designed for quadratic bound-constrained minimization problems. It consists of several gradient projections to identify the \emph{active} nodes, \emph{i.e.} those either $\dvec^n=\dvec^{n-1}$ or $\dvec^n=1$. Then it applies the preconditioned conjugate gradient (PCG) method to minimize an unconstrained reduced problem of the \emph{free} variables, \emph{i.e.} those satisfying $\dvec^{n-1}<\dvec^n<1$. The incomplete Cholesky factorization preconditioner is applied block-wise into each decomposed subdomain. Computational load is also well balanced in parallel computations based on domain decomposition, see \cite{BensonMcInnesMore:2001}.

Using the PETSc library, both the implicit and explicit time-stepping \cref{algo:implicit,algo:explicit} are implemented as a Python package named ``FEniCS Dynamic Gradient Damage'', see \cite{LiMaurini:2015}. It is based on the FEniCS Project \cite{LoggMardalWells:2012} for automated solution of partial differential equations. A specific C++ interface \texttt{PETScTAOSolver} is developed which brings various mathematical optimization algorithms defined in PETSc/TAO into FEniCS. In this implementation the tension-compression asymmetry is not taken into account. A linearized strain theory is also assumed.

The explicit time-stepping procedure for the dynamic gradient damage model summarized in \cref{algo:explicit} is also implemented in EUROPLEXUS, an explicit dynamics program dedicated to transient phenomena involving fluid-structure interaction \cite{EPX:2015}. Here all the tension-compression asymmetry models discussed in \cref{sec:TC} are implemented. The large-displacement extension, see \cref{sec:extensions}, is also by default activated in the software.

\subsection{Details of implementation in EUROPLEXUS}

\paragraph{Initialization}
The initialization \texttt{INIT\_ENDOGRAD} of the gradient damage model is performed at the end of the routine \texttt{initia.ff}. It has several objectives
\begin{itemize}
\item Allocate various Fortran arrays needed in the model: the damage field \texttt{ALPHA}, the damage dissipation energy for all elements \texttt{EFISS}\ldots

\item Initialize the PETSc solver and several objects: the minimization solver \texttt{TAOSOLV}, the global Hessian matrix for damage \texttt{HALPHA}, the PETSc viewer \texttt{VIEWER} to have access to convergence information\ldots
\end{itemize}

To guarantee a good efficiency when assembling sparse matrices, it is essential that we preallocate enough memories by indicating to PETSc the number of non-zero elements at every row of the matrix. In parallel, the non-zeros elements should be treated separately depending whether they appear in the diagonal or off the diagonal of the current processor. If this preallocation phase is omitted, PETSc will allocate the needed memory by calling \texttt{PetscMalloc} every time it encounters a new unexpected non-zero term, which causes the execution time to increase significantly, see \cite{PETSc:2015}. An overestimation of the sparse structure of the mesh connectivity is performed: for every node in the mesh, we consider all the elements containing that node and count the total node numbers possessed by these elements. Only one loop on all the elements is needed during the initialization. This overestimation does virtually no harm to the overall computational efficiency, as during the final assembly of matrices in PETSc, the routines \texttt{MatAssemblyBegin/End} will be called to release all unnecessary preallocated memories.

During a MPI parallel computation, a permutation of nodes is needed according to the PETSc convention described by the documentation of the routine \texttt{MatMPIAIJSetPreallocation}. It is achieved during initialization by a loop on all the processors based on the mesh partition performed by EPX. At the end two integer Fortran arrays are provided: \texttt{CONV\_EP(I)} which gives the position of the $i$-th EPX node in the PETSc convention, while \texttt{CONV\_PE} converts the PETSc node ordering to the EPX one.

\paragraph{Time-stepping} The change to the EPX time iteration routines \texttt{calcul.ff} and \texttt{d\_calcul.ff} is minimal. We simply add the following pseudo code concerning the assembly and the solving of the damage problem before the elastodynamics part.
\begin{mdframed}[hidealllines=true,backgroundcolor=gray!20]
\begin{minted}[breaklines]{fortran}
...
IF (L_ENDOGRAD) THEN
  CALL SEQ/MPI_ASSEM_ALPHA(...)
  CALL RESO_ALPHA(...)
END IF
...
CALL CELEM(...)
...
\end{minted}
\end{mdframed}

We note that the routine \texttt{SEQ/MPI\_ASSEM\_ALPHA} constructs the Hessian matrix $\vec{H}$ and the second member $\vec{b}$ associated to the quadratic damage minimization problem by performing a first loop on all the elements. The copy and the correspondence between the global damage vector and the local damage vector specific to each element are conducted without modifying the arguments of all existing routines: \texttt{celem.ff}, \texttt{loopelm.ff}, \texttt{wpgen.ff} and all element routines \texttt{triang.ff} for example. The elementary variables associated with the current element $\alpha^\mathrm{e}$, $\vec{H}^\mathrm{e}$ and $\vec{b}^\mathrm{e}$ are all stored globally inside the module \texttt{m\_endograd\_data.ff}. They are preallocated during the initialization phase depending on the maximum local nodes. This current implementation is not thread-safe, thus OpenMP parallelization is not available.

\paragraph{Element routines} At every time step, the following two loops on elements are needed:
\begin{enumerate}
\item Knowing the current configuration $\vec{x}^n$ and $\vec{u}^n$, the routine \texttt{SEQ/MPI\_ASSEM\_ALPHA} opens a first loop on elements to calculate the elementary matrices $\vec{H}^\me$ and vectors $\vec{b}^\me$ for the damage problem \eqref{eq:cracknummin}. During this first loop, the current damage field is not yet known.

\item After energy minimization \eqref{eq:cracknummin}, the current damage is solved in \texttt{RESO\_ALPHA}. We then open a loop for the second time in \texttt{CELEM} on all the elements to calculate the internal force vector $\vec{F}_\mathrm{int}^\me$ which depends on the current damage state.
\end{enumerate}

Since it is needed to perform two loops on elements at every time step, the element routines should be carefully restructured to minimize the computational cost. Some logical indicators are specially used and introduced to partition the code:
\begin{itemize}
\item \texttt{MATPRI(NTH) == 115} indicates if the current element is affected by a gradient damage material. If so, the following two logical indicators are then relevant.

\item \texttt{L\_FIRSTLOOP} encloses the lines that should be run during the calling of \texttt{SEQ/MPI\_ASSEM\_ALPHA}, \emph{i.e.} in order to calculate the elementary matrices $\vec{H}^\me$ and vectors $\vec{b}^\me$ for the damage problem. In particular, the strain tensor will be calculated at this stage via \texttt{DSDX2D} or \texttt{DSDX3D}. The tension-compression asymmetry is also performed here to calculate $\sig_0^+$ and $\sig_0^-$. The positive stress is stored temporarily to the \texttt{SIG} array while the negative stress to the \texttt{ECR} vector.

\item \texttt{L\_SECONDLOOP} encloses the lines that should be run during the calling of \texttt{CELEM} where the stress tensor and the internal force vector $\vec{F}_\mathrm{int}^\me$ will be calculated. In particular the stress tensor will be updated via \eqref{eq:stressposneg}
\end{itemize}

\paragraph{Improved method of calculating eigenvalues}
For the tension-compression asymmetry model described in \cite{MieheHofackerWelschinger:2010,FreddiRoyer-Carfagni:2010}, we need to perform the spectral decomposition of the strain tensor, see for example \cref{algo:epspos3d,algo:epspos2d}. The \texttt{DSYEV} routine of LAPACK is purely numerical (iterative method) and consequently robust but slow. Here we use and implement an analytical method described in \cite{Scherzinger:2008aa} that is both robust (in presence of two or three identical or near eigenvalues) and computationally efficient. Consider the problem
\[
\vec{A}\approx\mathbb{I}=\begin{bmatrix}
1 & &  \\
& 1+\frac{\epsilon}{4} & \frac{3\sqrt{\epsilon}}{4} \\
& \frac{3\sqrt{\epsilon}}{4} & 1+\frac{3\epsilon}{4}
\end{bmatrix}
\]
with $\epsilon=\num{1e-7}$. Theoretical solution $(d_i,\vec{v}_i)$ is given by
\begin{equation} \label{eq:eigenvalueana}
\begin{aligned}
& d_1=1+\epsilon\,,\quad d_2=d_3=1, \\
& \vec{v}_1=(0, 1/2, \sqrt{3}/2)\,,\quad \vec{v}_2=(-1, 0, 0)\quad\text{and}\quad\vec{v}_3=(0, \sqrt{3}/2, -1/2).
\end{aligned}
\end{equation}

The analytical solution \eqref{eq:eigenvalueana} is compared with the numerical ones given by \texttt{DSYEV} and the method described in \cite{Scherzinger:2008aa}:
\begin{itemize}
\item \texttt{DSYEV}: 15 significant figures for eigenvalues and eigenvectors, within \SI{63.7}{s} for \num{1e8} calculations.
\item New method: 15 significant figures for eigenvalues and 10 for eigenvectors, within \SI{23.5}{s} for \num{1e8} calculations.
\end{itemize}
As can be seen, a factor of 2.7 in time is gained.