\chapter{Simulation results}
\minitoc

\section{Better Understanding of Gradient Damage Models}

\subsection{1d bar under shock}

\subsection{Antiplane tearing}

\subsection{Dynamic crack branching}

\subsection{Edge-cracked plate under shearing impact} \label{sec:kalthoff}

\section{Experimental Validations}

\subsection{Crack arrest due to the presence of a hole}

\subsection{Dynamic fracture of concrete L-specimen}
%This dynamic tensile test on a L-shaped concrete specimen is proposed in \cite{OzboltBedeSharmaMayer:2015}. Its geometry and loading conditions are summarized in Fig. \ref{fig:L-specimen}. A hard device (displacement control) is applied on the lower left arm \SI{30}{mm} from the edge through a disc of $D=\SI{45}{mm}$ to suppress if possible local damages due to concentration. A quadrilateral mesh refined near \emph{a priori} known crack propagation region is used, arriving at approximately quadrilateral \num{200000} elements. The concrete material properties (density, elastic moduli and fracture toughness) are taken from \cite{OzboltBedeSharmaMayer:2015}.
%\begin{figure}[htbp]
%\centering
%\includegraphics[width=0.45\linewidth]{L-specimen.pdf}
%\caption{Geometry and loading conditions for the L-specimen problem. Damage field $\alpha_t$ at $t=\SI{6e-4}{s}$ ranging from 0 (gray) to 1 (white), for $v=\SI{0.74}{m/s}$.} \label{fig:L-specimen}
%\end{figure}
%
%The load is applied through unilateral contact (EPX keyword \texttt{IMPACT}) with a hard device with imposed velocity $\vec{V}(t)=\overline{V}f(t)\vec{e}_2$. The intensity is scaled via the factor $\overline{V}$. The function $f(t)$ defined below ensures that at crack initiation the loading velocity is approximately $V$. Three loading speeds are tested: $\SI{0.74}{m/s}$, $\SI{1}{m/s}$ and $\SI{1.5}{m/s}$. From Figs. \ref{fig:L-specimen} and \ref{fig:alpha_L}, we observe that the initial propagation angle slightly increases with the prescribed velocity (see also Tab. \ref{tab:initial_angle}) and crack branching may produce, as it is also reported in experiments.
%\begin{mdframed}[hidealllines=true,backgroundcolor=gray!20]
%\begin{minted}[breaklines]{text}
%FONC 1 TABL 5 0D0 0D0 1.5D-4 1D0 2D-4 1D0 4D-4 2D0 1D0 2D0
%\end{minted}
%\end{mdframed}
%
%\begin{figure}[htbp]
%\centering
%\begin{subfigure}[b]{0.48\textwidth}
%\centering
%\includegraphics[height=5cm]{v1000.jpg}
%\caption{$v=\SI{1}{m/s}$}
%\end{subfigure}
%\begin{subfigure}[b]{0.48\textwidth}
%\centering
%\includegraphics[height=5cm]{v1500.jpg}
%\caption{$v=\SI{1.5}{m/s}$} \label{fig:v15ell1}
%\end{subfigure}
%\caption{Damage field $\alpha_t$ ranging from 0 (blue) to 1 (red) for different loading speeds.} \label{fig:alpha_L}
%\end{figure}
%
%\begin{table}[htbp]
%\centering
%\caption{Initial crack propagation direction with different loading speeds for the L-specimen problem.} \label{tab:initial_angle}
%\begin{tabular}{llll} \toprule
%& $\SI{0.74}{m/s}$ & $\SI{1}{m/s}$ & $\SI{1.5}{m/s}$ \\ \midrule
%Initial propagation direction & $\SI{64}{\degree}$ & $\SI{71}{\degree}$ & $\SI{77}{\degree}$ \\ \bottomrule
%\end{tabular}
%\end{table}
%
%We then turn to the global dynamic structural response obtained with different loading rates. As it is expected, the peak load increases with the prescribed velocity, cf. Fig. \ref{fig:F-t-L-sim}. A good agreement with the experimental measurement is also found at $v=\SI{1.5}{m/s}$ as illustrated in Fig. \ref{fig:F-t-L}. Knowing that no strain rates effects is taken into account during material constitutive modeling, this increase of peak load for higher loading rates can be attributed to inertial itself. According to \cite{OzboltBedeSharmaMayer:2015}, this progressive increase of resistance is a pure consequence of inertial effects and not from velocity-dependent material strength or fracture energy.
%\begin{figure}[htbp]
%\centering
%\begin{subfigure}[b]{0.48\textwidth}
%\centering
%\includegraphics[height=5cm]{F_v.pdf}
%\caption{Simulation results.} \label{fig:F-t-L-sim}
%\end{subfigure}
%\begin{subfigure}[b]{0.48\textwidth}
%\centering
%\includegraphics[height=5cm]{F_v1500.pdf}
%\caption{Comparison with experimental measurement at $v=\SI{1.5}{m/s}$.} \label{fig:F-t-L}
%\end{subfigure}
%\caption{Load-time histories for different loading speeds.}
%\end{figure}
%
%It should be noted that these calculations have been performed with a relatively small internal length $\ell=\SI{1}{mm}$, which amounts to overestimate the tensile strength $\sigma_\mathrm{max}\approx \SI{27}{MPa}$ of the concrete. If we use instead the correct value of $\sigma_\mathrm{max}=\SI{3.12}{MPa}$ as indicated in \cite{OzboltBedeSharmaMayer:2015}, the internal length becomes \SI{73}{mm}, a size comparable to that of the specimen and which may undermine the approximation of brittle fracture via damage. Hopefully as the crack initiates at the center corner in presence of a stress singularity, the internal length should not influence much crack initiation, contrary to what is claimed in \cite{MesgarnejadBourdinKhonsari:2014}. The peak load as well as the global structural response is not sensible to the internal length, as can be seen from Fig. \ref{fig:alphav1500ell2x0F}. However with $\ell=\SI{2}{mm}$ the damage field obtained is slightly different from the case $\ell=\SI{1}{mm}$ in Fig. \ref{fig:v15ell1}, and crack nucleation from the foundation is also observed due to a smaller maximal tensile stress $\sigma_\mathrm{max}\approx \SI{19}{MPa}$ (yet still larger than the real value). Since in experiments no secondary crack initiation is present, we must ask the question how the critical stress $\sigma_\mathrm{max}=\SI{3.12}{MPa}$ is obtained in \cite{OzboltBedeSharmaMayer:2015}.
%\begin{figure}[htbp]
%\centering
%\begin{subfigure}[b]{0.48\textwidth}
%\centering
%\includegraphics[height=5cm]{v1500-ell2x0.png}
%\caption{Damage field with $\ell=\SI{2}{mm}$.} \label{fig:alphav1500ell2x0}
%\end{subfigure}
%\begin{subfigure}[b]{0.48\textwidth}
%\centering
%\includegraphics[height=5cm]{F_v1500_ell.pdf}
%\caption{Load-time histories.} \label{fig:alphav1500ell2x0F}
%\end{subfigure}
%\caption{Influence of the internal length $\ell$ on damage field and global structural response at $v=\SI{1.5}{m/s}$.}
%\end{figure}