\chapter{Conclusion and Perspectives} \label{chap:conclusion}

The attentive reader can not fail to realize the essential role played by the variational nature of the formulation in the derivation of these concepts, which is applicable for a large class of damage constitutive laws. Using the three physical principles of irreversibility, stability and energy balance, analogies between different models can be rigorously formalized. Properties derived in one model can be translated to the others, which is the case observed for the variational dynamic fracture model and the dynamic gradient damage model during the crack propagation phase, see Tab. \ref{tab:analogy}. In particular, the equation of motion governing the crack tip can be obtained by calculating the first-order action variation with respect to arbitrary crack evolution and by using the energy balance condition. This procedure could be repeated for other variational formulations of crack evolutions. An interesting extension would be the gradient damage model coupled with plasticity \cite{AlessiMarigoVidoli:2015}.
\begin{table}[htbp]
\caption{Analogies between the Variational Dynamic Fracture Model and the Dynamic Gradient Damage Model during the crack propagation phase} \label{tab:analogy}
\centering
\begin{tabular}{lll} \toprule
 & Variational Dynamic Fracture Model & Dynamic Gradient Damage Model \\ \midrule
Irreversibility & $\dot{l}_t\geq 0$ & $\dot{\alpha}_t\geq 0$ and $\dot{l}_t\geq 0$ \\
Elastic energy & $\mathcal{E}^*(\vec{u}^*_t,l_t)$ & $\mathcal{E}^*(\vec{u}^*_t,\alpha^*_t,l_t)$ \\
Kinetic energy & $\mathcal{K}^*(\vec{u}_t^*,\dot{\vec{u}}_t^*,l_t,\dot{l}_t)$ & $\mathcal{K}^*(\vec{u}_t^*,\dot{\vec{u}}_t^*,l_t,\dot{l}_t)$ \\
Dissipated energy & $\mathcal{S}(l_t)=\gc\cdot l_t$ & $\mathcal{S}^*(\alpha^*_t,l_t)$ \\
Stability condition & $\mathcal{A}'(\vec{u}^*,l)(\vec{v}^*-\vec{u}^*,\delta l)\geq 0$ & $\mathrm{A}'(\vec{u}^*,\alpha^*,l)(\vec{v}^*-\vec{u}^*,\beta^*-\alpha^*,\delta l)\geq 0$ \\
Eq. for $\vec{u}$ & $\rho\ddot{\vec{u}}_t=\div\tens{A}\eps(\vec{u}_t)+\vec{f}_t$ & $\rho\ddot{\vec{u}}_t=\div\tens{A}(\alpha_t)\eps(\vec{u}_t)+\vec{f}_t$ \\
Eq. for $l$ & Griffith's law \eqref{eq:gtgc} & Generalized Griffith criterion \eqref{eq:GgriffithlawJ} \\
Energy release rate & Classical $J-$integral \eqref{eq:Jdyn} & Generalized $\widehat{J}$-integral \eqref{eq:GtGandJdynG} \\ \bottomrule
\end{tabular}
\end{table}

Another novelty concerns the application of shape derivative methods \cite{Destuynder:1981} to the gradient damage model. Thanks to a well-defined diffeomorphism, in the sharp-interface fracture model the current cracked material configuration on which mechanical quantities are defined is transformed to the initial cracked one. Similarly in the phase-field approach the current damage field representing a propagating crack is mapped from a damage profile field which corresponds to a stationary initial crack. This Lagrangian formalism gives a rigorous sense to the shape derivative of the action integral with respect to the current crack length, which leads in return to the definition of an energy release rate even in absence of stress singularities.

The most essential assumption behind the generalized Griffith criterion resides in the non-positivity of the generalized $J$-integral. A theoretic proof of Hypothesis \ref{eq:Jleq0} calls for a careful singularity analysis similar to that conducted in \cite{SicsicMarigo:2013}. Let's recall that during the analysis the crack topology is restricted to a single straight line. Following the discussion at the end of \cref{chap:griffithrevis}, predefined curved crack paths can as well be considered. When several cracks are present in the body, as long as a diffeomorphism similar to \eqref{eq:philt} can be constructed between the initial cracked configuration and a perturbed multi-cracked configuration (generally speaking when those cracks do not interact with each other), the formalism described in this paper can still be applied. By relaxing furthermore the hypothesis of a fixed crack propagation direction, we may hope to identify a macroscopic kinking/branching criterion hidden behind the stability condition \eqref{eq:vi}. An interesting challenge would be to use more adequate shape derivative methods \cite{Hintermuller:2011} in order to differentiate the action integral with respect to the propagation angle. Furthermore we assume that the totally damaged zone corresponds to a subset of measure zero with respect to $\md\vec{x}$. When it is not the case, more energy would be dissipated during crack propagation which could represent an increase of the apparent fracture toughness observed during dynamic crack microbranching processes \cite{SharonFineberg:1996}. Finally, only the crack propagation phase is considered in this paper. The establishment of an initial damage field in a body with possible defects is subject to the irreversibility condition, the damage criterion \eqref{eq:localdamagefirstorder} and the consistency condition \eqref{eq:damageconsis}. The generalized Griffith criterion stated in \cref{prop:Ggriffithlaw} and the asymptotic Griffith's law stated in \cref{prop:whenellpetit} no longer apply since an initial crack is absent and a separation of scales is not possible. It refers to the dynamic phase-field crack \emph{nucleation} problem which will be analyzed through numerical simulations in this work.

