%!TEX root=../main.tex
\chapter{Conclusion and Perspectives} \label{chap:conclusion}
In this chapter some concluding remarks of the present work are given. In \cref{sec:remtheo}, the focus is on the gradient damage approach itself as a genuine physical model of fracture. The theoretic and numerical investigations from the previous chapters are discussed and some future work are indicated. In \cref{sec:remnum} we concentrate on the numerical discretization of the model. Some possible improvements of the current implementation are suggested. Finally we discuss the application of the dynamic gradient damage model to real-world concrete structures, in \cref{sec:remconcrete}.

\section{Toward an Improved Modeling of Fracture} \label{sec:remtheo}

\subsection{Position with respect to other approaches}
\cref{chap:introduction} provides an overview of several existing modeling approaches to brittle fracture. In particular, the Griffith's theory of dynamic fracture and the variational approach to fracture are respectively recalled in \cref{sec:griffithfreund} and \cref{sec:fm98}. The choice of these two models is not arbitrary. Indeed, the gradient damage model can be regarded as a genuine physical model of fracture \emph{per se} that lies between these two approaches. One the one hand, the gradient damage model agrees with the Griffith's law \eqref{eq:gtgc} when applied to pre-existing cracks that propagate in the body without complex topology changes, see \cref{sec:linkDF,sec:antiplane} and a summary below in \cref{sec:phases}. It can thus be regarded as a superset of the classical linear elastic fracture mechanics. 

On the other hand, the gradient damage approach can be regarded as an effective numerical implementation of the variational approach to fracture using the $\Gamma$-convergence theory, at least in the quasi-static setting, see \cref{sec:graddamage}. However in the gradient damage model, the $\ell$-parameter is regarded as a physical parameter and its role should be investigated with respect to the material and structural behavior of a gradient-damage body, see a summary in \cref{sec:role}. Furthermore, in dynamics the exact behavior when $\ell\to 0$ is still unknown. It requires a precise modeling of a sharp-interface dynamic fracture model and recalls for collaborations from both the mathematics and mechanics community. The relationship between these models is summarized in \cref{fig:griffithvariagraddama}.
\begin{figure}[htbp]
\centering
\[
\begin{CD}
\mathcal{A}(\vec{u},l) @<\text{Pre-existing}<\text{\emph{simple} cracks}< \mathcal{A}(\vec{u},\alpha) @>\text{$\ell\to 0$}>> \mathcal{A}(\vec{u},\Gamma)?
\end{CD}
\]
\caption{One the one hand, the dynamic gradient damage model (symbolized by $\mathcal{A}(\vec{u},\alpha)$ in \eqref{eq:actionG}) agrees with the Griffith's theory (symbolized as $\mathcal{A}(\vec{u},l)$, see \cref{chap:griffithrevis} for a variational reformulation) when applied to pre-existing cracks that propagate in the body without complex topology changes. On the other hand, the (quasi-static) gradient damage approach is connected to the variational approach to fracture ($\mathcal{P}(\Gamma)$ in \eqref{eq:PES}) through the $\Gamma$-convergence theory in the quasi-static setting. The exact behavior when $\ell\to 0$ in dynamics is currently unknown} \label{fig:griffithvariagraddama}
\end{figure}

\subsection{Variational formulation}
The variational formulation of the dynamic gradient damage model outlined in \cref{def:dynagraddama} settles down a general framework of an elastic body that undergoes a brittle fracture behavior. The first-order stability condition \cref{eq:vi} states that the true dynamic coupled evolution of the displacement field and the damage field corresponds to a stationary generalized space-time action integral \eqref{eq:actionG}. By virtue of the energy minimization principle \cref{eq:crackmin}, the spatial (kinking, branching, \ldots) and temporal evolution (initiation, arrest, \ldots) of the gradient-damage crack can be predicted without any additional physical or numerical criteria.

The formulation of the dynamic gradient damage model itself can be regarded as a generalization of the Griffith's theory, which is also variational in nature. In \cref{chap:griffithrevis}, a rigorous reformulation of the dynamic Griffith's law \eqref{eq:gtgc} is also achieved by exploiting the stationarity of a similar sharp-interface space-time action integral \eqref{eq:action}. The difference between \eqref{eq:actionG} and \eqref{eq:action} resides in the definition of the dissipated energy $\mathcal{S}$ due to fracture. In the gradient damage model a smeared description is adopted and the dissipated energy \eqref{eq:surface} is written as a volume integral in the body involving only the damage variable. On the contrary, the Griffith's theory adopts a sharp-interface description and the dissipated energy corresponds to the original Griffith's hypothesis of energy dissipation \eqref{eq:Selllt}. The link between these two descriptions is achieved via the $\Gamma$-convergence result \eqref{eq:griffith} as well as the definition \eqref{eq:gcingd} of the fracture toughness in gradient damage models.

The major constraint prescribed in the Griffith's theory refers to a predefined crack path and an existing crack. By applying the stability principle \eqref{eq:stability} in the variational formulation of the Griffith's theory of dynamic fracture, one obtains the well known Griffith's law \eqref{eq:gtgc} which is then effectively used to derive an equation of motion of the crack tip. The exact procedure is then performed for the dynamic gradient damage model in \cref{sec:linkDF}, in order to obtain a similar scalar equation governing the crack tip of the gradient-damage crack. The attentive reader can not fail to realize the essential role played by the variational nature of the formulation in the derivation of several energy release rate concepts (the conventional dynamic energy release rate \eqref{eq:GtC} and the damage dissipation rate \eqref{eq:Gammat}) in the gradient damage model, which is applicable for a large class of damage constitutive laws. Using the three physical principles of irreversibility, stability and energy balance, analogies between these two models can be rigorously formalized. Properties derived in the one model can be translated to the other, see \cref{tab:analogy}. In particular, the equation of motion governing the crack tip can be obtained by calculating the first-order action variation with respect to arbitrary crack evolution and by using the energy balance condition. This procedure could be repeated for other variational formulations of crack evolutions. An interesting extension would be the gradient damage model coupled with plasticity \cite{AlessiMarigoVidoli:2015}.
\begin{table}[htbp]
\caption{Analogies between the Variational Dynamic Fracture Model (\cref{def:griffith}) and the Dynamic Gradient Damage Model (\cref{def:dyngraddama}) during the crack propagation phase} \label{tab:analogy}
\centering
\begin{tabular}{lll} \toprule
 & Variational Dynamic Fracture Model & Dynamic Gradient Damage Model \\ \midrule
Irreversibility & $\dot{l}_t\geq 0$ & $\dot{\alpha}_t\geq 0$ and $\dot{l}_t\geq 0$ \\
Elastic energy & $\mathcal{E}^*(\vec{u}^*_t,l_t)$ & $\mathcal{E}^*(\vec{u}^*_t,\alpha^*_t,l_t)$ \\
Kinetic energy & $\mathcal{K}^*(\vec{u}_t^*,\dot{\vec{u}}_t^*,l_t,\dot{l}_t)$ & $\mathcal{K}^*(\vec{u}_t^*,\dot{\vec{u}}_t^*,l_t,\dot{l}_t)$ \\
Dissipated energy & $\mathcal{S}(l_t)=\gc\cdot l_t$ & $\mathcal{S}^*(\alpha^*_t,l_t)$ \\
Stability condition & $\mathcal{A}'(\vec{u}^*,l)(\vec{v}^*-\vec{u}^*,\delta l)\geq 0$ & $\mathrm{A}'(\vec{u}^*,\alpha^*,l)(\vec{v}^*-\vec{u}^*,\beta^*-\alpha^*,\delta l)\geq 0$ \\
Eq. for $\vec{u}$ & $\rho\ddot{\vec{u}}_t=\div\tens{A}\eps(\vec{u}_t)+\vec{f}_t$ & $\rho\ddot{\vec{u}}_t=\div\tens{A}(\alpha_t)\eps(\vec{u}_t)+\vec{f}_t$ \\
Eq. for $l$ & Griffith's law \eqref{eq:gtgc} & Generalized Griffith criterion \eqref{eq:GgriffithlawJ} \\
Energy release rate & Classical $J-$integral \eqref{eq:Jdyn} & Generalized $\widehat{J}$-integral \eqref{eq:GtGandJdynG} \\ \bottomrule
\end{tabular}
\end{table}

\subsection{Crack evolution law} \label{sec:phases}
\paragraph{Nucleation} Compared to the Griffith's approach, crack nucleation from a perfectly sound body can be predicted in the gradient damage model. If damage initiation is governed by the stress-based Kuhn-Tucker conditions given by \eqref{eq:localdamagefirstorder} and \eqref{eq:damageconsis}, crack nucleation (when the damage variables attains 1 somewhere in the body) could be subject to structural effects depending on the relative size between the material internal length $\ell$ and the dimension of the body. This size effect is numerically investigated in \cref{sec:1d} for a one-dimensional bar under impact. Simulation results confirm the well-acknowledged belief: ``smaller is stronger''. Indeed, longer bars tend to break as long as the maximal stress is reached, however the fracture instant for smaller bars is delayed, see \cref{fig:tfell}.

A comparison between the gradient damage model and local strain-softening models is also performed. In the limit $\ell\to 0$, the fracture behaviors with the gradient-damage approach converge to that predicted by ill-posed local models. This result enhances the stand that the parameter $\ell$ should be considered as a material parameter and not as a purely numerical parameter, see also \cref{sec:twointerpretations}.

\paragraph{Initiation, propagation and arrest} The apparent crack evolution in gradient damage models under the Griffith's fundamental hypothesis (predefined crack path and an existing crack) is first analyzed from a theoretic approach in \cref{sec:linkDF}. It is shown that the crack tip equation of motion is governed by the generalized Griffith criterion (\cref{prop:Ggriffithlaw}) and the asymptotic Griffith's law (\cref{prop:whenellpetit}). Generally speaking, when the internal length is small compared to the dimension of the body, the gradient-damage crack behaves exactly like a Griffith's one, in absence of complex topology changes. Note that the crack nucleation phase is not governed by these two principles since an initial crack is absent and a separation of scales is not possible.

The novelty concerns the application of shape derivative methods \cite{Destuynder:1981} to the gradient damage model. Thanks to a well-defined diffeomorphism \eqref{eq:philt}, in the sharp-interface Griffith's fracture model the current cracked material configuration on which mechanical quantities are defined is transformed to the initial cracked one. Similarly in the phase-field approach the current damage field representing a propagating crack is mapped from a damage profile field which corresponds to a stationary initial crack. This Lagrangian formalism gives a rigorous sense to the shape derivative of the action integral with respect to the current crack length, which leads in return to the definition of an energy release rate even in absence of stress singularities.

The most essential assumption behind the generalized Griffith criterion resides in the non-positivity of the generalized $J$-integral. A theoretic proof of \cref{eq:Jleq0} calls for a careful singularity analysis similar to that conducted in \cite{SicsicMarigo:2013}. Let's recall that during the analysis the crack topology is restricted to a single straight line. Following the discussion at the end of \cref{chap:griffithrevis}, predefined curved crack paths can as well be considered. When several cracks are present in the body, as long as a diffeomorphism similar to \eqref{eq:philt} can be constructed between the initial cracked configuration and a perturbed multi-cracked configuration (generally speaking when those cracks do not interact with each other), the formalism described here can still be applied. By relaxing furthermore the hypothesis of a fixed crack propagation direction, we may hope to identify a macroscopic kinking/branching criterion hidden behind the stability condition \eqref{eq:vi}. An interesting challenge would be to use more adequate shape derivative methods \cite{Hintermuller:2011} in order to differentiate the action integral \eqref{eq:actionG} with respect to the propagation angle. Furthermore we assume that the totally damaged zone corresponds to a subset of measure zero with respect to $\md\vec{x}$. When it is not the case, more energy would be dissipated during crack propagation which could represent an increase of the apparent fracture toughness observed during dynamic crack microbranching processes investigated in \cref{sec:branching}, see also a summary below. Future work will be devoted to this point to theoretically investigate the micro-branching phenomenon.

In \cref{sec:antiplane} a numerical verification of these points is performed for an antiplane tearing experiment. The quasi-static limit of the dynamic gradient damage model is also investigated in presence of material inhomogeneities.
\begin{itemize}
\item In the dynamic tearing example of a homogeneous plate, it is verified that the crack evolution is governed by the asymptotic Griffith's law, as long as the material internal length is sufficiently small to establish a separation of scales between the inner damage problem and the outer LEFM problem. The conventional dynamic energy release rate \eqref{eq:GtC} is numerically computed and verified as a tool to translate gradient damage mechanics results in fracture mechanics terminology. We conduct a comparison with the 1-d peeling problem \cite{DumouchelMarigoCharlotte:2008} analytically studied with the classical Griffith's theory of dynamic fracture. A good agreement between them can be found in terms of the crack speeds prediction as a function of the loading speed.

\item We then investigate the quasi-static limits of the dynamic gradient damage model. In absence of \emph{brutal} or \emph{unstable} crack propagation where the classical static Griffith's theory fails, the dynamic model (\cref{def:dyngraddama}) converges to the first-order quasi-static gradient damage model (\cref{def:firstorderqs}), when the loading speed decreases. However when the crack may propagate at a speed comparable to the material sound speed, the dynamic model should be preferred to correctly account for inertial effects. The crack evolution in the dynamic gradient damage model is in quantitative accordance with the LEFM predictions on the 1-d peeling problem.
\end{itemize}

These numerical experiments provide hence a justification of the dynamic gradient damage model, when it is used as a genuine physical model for complex real-world dynamic fracture problems.

\paragraph{Kinking and branching}
Numerical investigation of the kinking

\subsection{Role of several modeling parameters} \label{sec:role}

\section{Numerical Implementation and Analysis} \label{sec:remnum}

\section{Application to Real-World Structures} \label{sec:remconcrete}
