%!TEX root=../main.tex
\chapter{Conclusion and Future Work} \label{chap:conclusion}
\minitoc

In this chapter some concluding remarks of the present work are given. It provides an updated state-of-the-art in the gradient-damage modeling of dynamic fracture following the objectives fixed in \cref{sec:scope}. In \cref{sec:dynamicalcon}, the dynamic extension of the variational formulation as well as the numerical implementation is reviewed. Then in \cref{sec:linkphasecon}, we give some remarks concerning the use of several modeling parameters in the gradient-damage model and how it is linked with other phase-field approaches. A better understanding of the gradient damage model constitutes another major contribution of the present work. It is reviewed in \cref{sec:bettercon}. Finally we discuss the application of the dynamic gradient damage model to real-world structures, in \cref{sec:valicon}.

To facilitate the reading, the future work arising from the current studies will be \future{underlined}.

\section{Going Dynamical} \label{sec:dynamicalcon}
This section provides some concluding remarks concerning the dynamic extension of the original quasi-static gradient damage model. The thematic subjects covered are summarized in \cref{tab:summcondyn}.
\begin{table}[htbp]
\centering
\caption{Thematic subjects covered in this section} \label{tab:summcondyn}
\begin{tabular}{ccccc} \toprule
& Going dynamical & $\alpha\leftrightarrow\phi$ & $\nabla\alpha\to\Gamma$ & Experimental validation \\ \midrule
Theoretics & \rightthumbsup & & & \\
Numerics & \rightthumbsup & & & \\ \bottomrule
\end{tabular}
\end{table}

\subsection{Position with respect to other approaches}
\cref{chap:introduction} provides an overview of several existing modeling approaches to brittle fracture. In particular, the Griffith's theory of dynamic fracture and the variational approach to fracture are respectively recalled in \cref{sec:griffithfreund} and \cref{sec:fm98}. The choice of these two models is not arbitrary. Indeed, the gradient damage model can be regarded as a genuine physical model of fracture \emph{per se} that lies between these two approaches. On the one hand, the gradient damage model agrees with the Griffith's law \eqref{eq:gtgc} when applied to preexisting cracks that propagate along a predefined path in the body without complex topology changes, see \cref{sec:linkDF,sec:antiplane}. It can thus be regarded as a superset of the Griffith's theory which is represented below by $\mathcal{A}(\vec{u},l)$, see \cref{chap:griffithrevis} for its variational reformulation.

On the other hand, the gradient damage approach can be regarded as an effective numerical implementation of the variational approach to fracture using the $\Gamma$-convergence theory, at least in the quasi-static setting, see \cref{sec:graddamage}. The potential energy $\mathcal{P}(\vec{u},\alpha)$ of the quasi-static gradient damage model outlined in \cref{def:qsgraddama} can be formally obtained by omitting the kinetic energy $\mathcal{K}$ in the generalized action integral $\mathcal{A}(\vec{u},\alpha)$ in the dynamic model. As $\ell\to 0$, the $\ell$-regularized functional $\mathcal{P}(\vec{u},\alpha)$ converges in a certain sense to the sharp-interface Francfort-Marigo variational fracture model $\mathcal{P}(\Gamma)$. In the gradient damage model, the $\ell$-parameter is regarded as a physical parameter and its role should be investigated with respect to the material and structural behavior of a gradient-damage body. Nevertheless, the limit $\ell\to 0$ defines a limiting behavior of the gradient damage model for small internal lengths and provides a better understanding of the model. In dynamics, \future{the exact behavior for the dynamic gradient damage model when $\ell\to 0$ is still unknown}. Furthermore, \future{a precise modeling of a sharp-interface dynamic fracture model is still lacking} and recalls for collaborations from both the mathematics and mechanics community. The relationship between these models is summarized in \cref{fig:griffithvariagraddama}.
\begin{figure}[htbp]
\centering
\[
\begin{CD}
\mathcal{A}(\vec{u},l) @<\text{Griffith's hypothesis}<< \mathcal{A}(\vec{u},\alpha) @>{\ell\to 0}>> ? \\
@. @VV{-\mathcal{K}}V @AA{+\mathcal{K}}A \\
@. \mathcal{P}(\vec{u},\alpha) @>{\ell\to 0}>> \mathcal{P}(\Gamma)
\end{CD}
\]
\caption{On the one hand, the dynamic gradient damage model $\mathcal{A}(\vec{u},\alpha)$ agrees with the Griffith's theory $\mathcal{A}(\vec{u},l)$ when applied to pre-existing cracks that propagate in the body without complex topology changes. On the other hand, the gradient damage approach is connected to the variational approach to fracture $\mathcal{P}(\Gamma)$ through the $\Gamma$-convergence theory in the quasi-static setting. The exact behavior when $\ell\to 0$ in dynamics is currently unknown and a precise modeling of a sharp-interface dynamic fracture model is still lacking, which is here symbolized by adding $\mathcal{K}$ in $\mathcal{P}(\Gamma)$} \label{fig:griffithvariagraddama}
\end{figure}

\subsection{Variational nature of the formulation}
The variational formulation of the dynamic gradient damage model outlined in \cref{def:dynagraddama} settles down a general framework of an elastic body that undergoes a brittle fracture behavior. The first-order stability condition \cref{eq:vi} states that the true dynamic coupled evolution of the displacement field and the damage field corresponds to a stationary generalized space-time action integral \eqref{eq:actionG}. By virtue of the energy minimization principle \eqref{eq:crackmin}, the spatial (kinking, branching, \ldots) and temporal evolution (initiation, arrest, \ldots) of the gradient-damage crack can be predicted without any additional physical or numerical criteria.

The formulation of the dynamic gradient damage model itself can be regarded as a generalization of the Griffith's theory, which is also variational in nature. In \cref{chap:griffithrevis}, a rigorous reformulation of the dynamic Griffith's law \eqref{eq:gtgc} is also achieved by exploiting the stationarity of a similar sharp-interface space-time action integral \eqref{eq:action}. The difference between \eqref{eq:actionG} and \eqref{eq:action} resides in the definition of the dissipated energy $\mathcal{S}$ due to fracture. In the gradient damage model a smeared description is adopted and the dissipated energy \eqref{eq:surface} is written as a volume integral in the body involving only the damage variable. On the contrary, the Griffith's theory adopts a sharp-interface description and the dissipated energy corresponds to the original Griffith's hypothesis of energy dissipation \eqref{eq:Selllt}. The link between these two descriptions is achieved via the $\Gamma$-convergence result \eqref{eq:griffith} as well as the definition \eqref{eq:gcingd} of the fracture toughness in gradient damage models.

The major constraint prescribed in the Griffith's theory refers to a predefined crack path and an existing crack. By applying the stability principle \eqref{eq:stability} in the variational formulation of the Griffith's theory of dynamic fracture, one obtains the well known Griffith's law \eqref{eq:gtgc} which is then effectively used to derive an equation of motion of the crack tip. The exact procedure is then performed for the dynamic gradient damage model in \cref{sec:linkDF}, in order to obtain a similar scalar equation governing the crack tip of the gradient-damage crack. The attentive reader can not fail to realize the essential role played by the variational nature of the formulation in the derivation of several energy release rate concepts (the conventional dynamic energy release rate \eqref{eq:GtC} and the damage dissipation rate \eqref{eq:Gammat}) in the gradient damage model, which is applicable for a large class of damage constitutive laws. Using the three physical principles of irreversibility, stability and energy balance, analogies between these two models can be rigorously formalized. Properties derived in the one model can be translated to the other, see \cref{tab:analogy}. In particular, the equation of motion governing the crack tip can be obtained by calculating the first-order action variation with respect to arbitrary crack evolution and by using the energy balance condition. This procedure could be repeated for other variational formulations of crack evolutions. \future{An interesting extension would be the gradient damage model coupled with plasticity} proposed in \cite{AlessiMarigoVidoli:2015}.
\begin{table}[htbp]
\caption{Analogies between the variational formulation of the Griffith's theory of dynamic fracture (\cref{def:griffith}) and the dynamic gradient damage evolution law for a propagating crack (\cref{def:dynagraddamanew})} \label{tab:analogy}
\centering
\begin{tabular}{lll} \toprule
 & Griffith's theory & Dynamic gradient damage model \\ \midrule
Irreversibility & $\dot{l}_t\geq 0$ & $\dot{\alpha}_t\geq 0$ and $\dot{l}_t\geq 0$ \\
Elastic energy & $\mathcal{E}^*(\vec{u}^*_t,l_t)$ & $\mathcal{E}^*(\vec{u}^*_t,\alpha^*_t,l_t)$ \\
Kinetic energy & $\mathcal{K}^*(\vec{u}_t^*,\dot{\vec{u}}_t^*,l_t,\dot{l}_t)$ & $\mathcal{K}^*(\vec{u}_t^*,\dot{\vec{u}}_t^*,l_t,\dot{l}_t)$ \\
Dissipated energy & $\mathcal{S}(l_t)=\gc\cdot l_t$ & $\mathcal{S}^*(\alpha^*_t,l_t)$ \\
Stability condition & $\mathcal{A}'(\vec{u}^*,l)(\vec{v}^*-\vec{u}^*,\delta l)\geq 0$ & $\mathrm{A}'(\vec{u}^*,\alpha^*,l)(\vec{v}^*-\vec{u}^*,\beta^*-\alpha^*,\delta l)\geq 0$ \\
Eq. for $\vec{u}$ & $\rho\ddot{\vec{u}}_t=\div\tens{A}\eps(\vec{u}_t)+\vec{f}_t$ & $\rho\ddot{\vec{u}}_t=\div\tens{A}(\alpha_t)\eps(\vec{u}_t)+\vec{f}_t$ \\
Eq. for $l$ & Griffith's law \eqref{eq:gtgc} & Generalized Griffith criterion \eqref{eq:GgriffithlawJ} \\
Energy release rate & Classical $J-$integral \eqref{eq:Jdyn} & Generalized $\widehat{J}$-integral \eqref{eq:GtGandJdynG} \\ \bottomrule
\end{tabular}
\end{table}

\subsection{Numerical implementation}
In \cref{chap:numerics} a space-time discretization of the continuous dynamic gradient damage model is performed. In the quasi-static setting the numerical implementation exploits fully the variational nature since the static equilibrium can be effectively solved by a mathematical optimization algorithm. In the present work, the spatial discretization is decoupled from the temporal one and a temporal finite difference scheme is used for the time-stepping of the spatially discretized finite element system. \future{Future work can be devoted to the investigation of space-time finite element methods}, see \cite{HughesHulbert:1988}.

The dynamic gradient damage model has been successfully implemented by the author in the EPX software \cite{EPX:2015}. The two damage constitutive laws \eqref{eq:at1} and \eqref{eq:at2}, as well as several tension-compression asymmetry models mentioned in \cref{sec:TC} are available. In the meanwhile, an open-source implementation \cite{LiMaurini:2015} based on the FEniCS project \cite{LoggMardalWells:2012} is also performed by the author.

In an explicit dynamics context, the damage problem \eqref{eq:cracknummin} appears as a bound-constrained convex optimization problem which is solved at the structural scale at every time step. Fortunately, by using the \eqref{eq:at1} damage constitutive law, the numerical solving of the current damage field together with the construction of the matrices represent approximately 50\% of the total computational time, which is quite reasonable for other classical local nonlinear material constitutive laws. Furthermore, thanks to an efficient parallel implementation using the PETSc framework, a quasi-perfect strong scaling diagram is obtained, see \cref{fig:scaling}. The computational wall time can thus be significantly reduced by using several processors.

The implementation is also applied to three-dimensional dynamic problems in \cref{sec:brazilian,sec:beam}. One of the advantage of the gradient-damage approach (and other models based on a smeared description of cracks) consists of a unified treatment of 2-d and 3-d fracture problems without any other additional numerical crack surface tracking techniques.

\section{Link with Phase-Field Approaches} \label{sec:linkphasecon}
This section discusses several modeling parameters in the gradient damage model and strengthens the link with other phase-field approaches. The thematic subjects covered are summarized in \cref{tab:summconph}.
\begin{table}[htbp]
\centering
\caption{Thematic subjects covered in this section} \label{tab:summconph}
\begin{tabular}{ccccc} \toprule
& Going dynamical & $\alpha\leftrightarrow\phi$ & $\nabla\alpha\to\Gamma$ & Experimental validation \\ \midrule
Theoretics & & \rightthumbsup & & \\
Numerics & & \rightthumbsup & & \\ \bottomrule
\end{tabular}
\end{table}

\subsection{Damage constitutive laws}
In this work two particular damage constitutive laws \eqref{eq:at1} and \eqref{eq:at2} are compared with respect to their aptitude to model brittle fracture inside the variational framework of the dynamic gradient damage model. In particular, the latter model leads to the widely-used regularized crack functional in the phase-field community. The present work can thus be considered as a bridge between these two communities. According to the crack branching experiment \cref{sec:branching}, it is confirmed that the \eqref{eq:at1} model performs better than the \eqref{eq:at2} one in terms of physical properties and computational efficiency due to a small damage band. \future{Future work could be devoted to the application of more sophisticated damage constitutive laws in dynamic fracture problems}.

In terms of computational efficiency, a bound-constrained minimization algorithm is needed for the \eqref{eq:at1} model. On the other hand, a simple linear system is required for the \eqref{eq:at2} one. However it is illustrated in \cref{sec:branching} that such nonlinear solvers is not very costly. \future{For more complex damage constitutive laws, the Hessian matrix for the damage problem is no more constant and the numerical efficiency should be reevaluated}.

\subsection{Tension-compression asymmetry formulations}
In \cref{sec:TC} we give a variational interpretation of several existing tension-compression asymmetry formulations. Their respective properties are then highlighted through a uniaxial traction experiment. The variational structure \eqref{eq:variationalepspos} provides a novel interpretation of these models. Future work can be devoted to the proposal of several new elastic energy splits \eqref{eq:elasticTC} within the framework.

The tension-compression asymmetry model of \cite{MieheHofackerWelschinger:2010}, that is widely used among the phase field community does not fit into this variational setting. Furthermore, it develops diffuse damage with a strange strain-hardening behavior under the uniaxial compression state thanks to a theoretic study in \cref{sec:uniaxial}. This property is numerically verified for the Kalthoff experiment in \cref{sec:kalthoff}. Due to this reason, we recommend the elastic energy split developed for masonry-like materials, \emph{i.e.} that of \cite{FreddiRoyer-Carfagni:2010}. With this tension-compression asymmetry formulation, correct propagation path for brittle materials such as PMMA and concrete is reproduced by comparison with the experimental observations, cf. \cref{sec:kalthoff,sec:brazilian,sec:L-specimen,sec:beam}.

However the existing tension-compression asymmetry models, including that proposed by \cite{FreddiRoyer-Carfagni:2010}, do not correctly account for true non-interpenetration condition, as is illustrated during several numerical simulations, see \cref{sec:kalthoff,sec:brazilian}. It is due to the fact that these energy splits no not depend on the current damage state as well as the damage gradient which approximates the crack normal direction. In particular, it appears that stress can be transmitted parallel to the gradient-damage crack with the \cite{FreddiRoyer-Carfagni:2010} model. \future{Future work could be devoted to a thorough investigation of the erosion mechanism available in the EPX software to realize a transition from a continuous to a discrete description of cracks}. \future{Another possibility is to consider an equivalent cohesive crack that replaces the gradient-damage crack}, see for instance \cite{CazesCoretCombescureGravouil:2009,CuvilliezFeyelLorentzMichel-Ponnelle:2012}. \future{Other numerical techniques frequently used in the sharp-interface fracture models can also be borrowed}: for instance the X-FEM method, see \cite{Comi:2007}.

\section{Better Understanding of Gradient Damage Modeling} \label{sec:bettercon}
In this section we provide a summary of the properties of dynamic gradient damage models when applied to investigate dynamic brittle fracture. Different temporal or spatial events or phases as outlined in \cref{sec:kinematics} are studied both theoretically and numerically in this work. The thematic subjects covered are summarized in \cref{tab:summconbetter}.
\begin{table}[htbp]
\centering
\caption{Thematic subjects covered in this section} \label{tab:summconbetter}
\begin{tabular}{ccccc} \toprule
& Going dynamical & $\alpha\leftrightarrow\phi$ & $\nabla\alpha\to\Gamma$ & Experimental validation \\ \midrule
Theoretics & & & \rightthumbsup & \\
Numerics & & & \rightthumbsup & \\ \bottomrule
\end{tabular}
\end{table}

\subsection{Nucleation}
Compared to the Griffith's approach, crack nucleation from a perfectly sound body is able to be predicted in the gradient damage model. If damage initiation is governed by a local condition \eqref{eq:damageinitiationsound}, crack nucleation (when the damage variables attains 1 somewhere in the body) could be subject to structural effects depending on the relative size between the material internal length $\ell$ and the dimension of the body. This size effect is numerically investigated in \cref{sec:1d} for a one-dimensional bar under impact and in \cref{sec:brazilian} for a three-dimensional Brazilian test for concrete cylinders. Simulation results, cf. \cref{fig:tfell,fig:sizeeffectv1,fig:brazilian_size} confirm the well-acknowledged belief: ``smaller is stronger''.

A comparison between the gradient damage model and local strain-softening models is also performed in \cref{sec:1d}. In the limit $\ell\to 0$, the fracture behaviors with the gradient-damage approach converge to that predicted by ill-posed local models. This result enhances the stand that the parameter $\ell$ should be considered as a material parameter and not as a purely numerical parameter, see \cref{sec:twointerpretations}. \future{Future work could be devoted to a theoretic investigation of this one-dimensional problem and in particular the size effect}.

\subsection{Initiation, propagation and arrest}
\paragraph{Theoretic investigation} The apparent crack evolution in gradient damage models under the Griffith's fundamental hypothesis (predefined crack path and an existing crack) is first analyzed from a theoretic approach in \cref{sec:linkDF}. It is shown that the crack tip equation of motion is governed by the generalized Griffith criterion (\cref{prop:Ggriffithlaw}) and the asymptotic Griffith's law (\cref{prop:whenellpetit}). Generally speaking, when the internal length is small compared to the dimension of the body, the gradient-damage crack behaves exactly like a Griffith's one, in the absence of complex topology changes. Note that the crack nucleation phase is not governed by these two principles since an initial crack is absent and a separation of scales is not possible.

The novelty concerns the application of shape derivative methods \cite{Destuynder:1981} to the gradient damage model. Thanks to a well-defined diffeomorphism \eqref{eq:philt}, in the sharp-interface Griffith's fracture model the current cracked material configuration on which mechanical quantities are defined is transformed to the initial cracked one. Similarly in the phase-field approach the current damage field representing a propagating crack is mapped from a damage profile field which corresponds to a stationary initial crack. This Lagrangian formalism gives a rigorous sense to the shape derivative of the action integral with respect to the current crack length, which leads in return to the definition of an energy release rate even in the absence of stress singularities.

The most essential assumption behind the generalized Griffith criterion resides in the non-positivity of the generalized $J$-integral. A theoretic proof of \cref{eq:Jleq0} calls for a \future{careful singularity analysis} similar to that conducted in \cite{SicsicMarigo:2013}. Let's recall that during the analysis the crack topology is restricted to a single straight line. Following the discussion at the end of \cref{chap:griffithrevis}, predefined \future{curved} crack paths can as well be considered. When \future{several cracks} are present in the body, as long as a diffeomorphism similar to \eqref{eq:philt} can be constructed between the initial cracked configuration and a perturbed multi-cracked configuration (generally speaking when those cracks do not interact with each other), the formalism described here can still be applied. \future{By relaxing furthermore the hypothesis of a pre-defined crack propagation path, we may hope to identify a macroscopic kinking/branching criterion} hidden behind the stability condition \eqref{eq:vi}. \future{An interesting challenge would be to use more adequate shape derivative methods} \cite{Hintermuller:2011} in order to \future{differentiate the action integral} \eqref{eq:actionG} \future{with respect to the propagation angle}. Furthermore we assume that the totally damaged zone corresponds to a subset of measure zero with respect to $\md\vec{x}$. When it is not the case, more energy would be dissipated during crack propagation which could represent an increase of the apparent fracture toughness observed during dynamic crack microbranching processes investigated in \cref{sec:branching}. \future{Future work could be devoted to this point to theoretically investigate the micro-branching phenomenon}.

\paragraph{Numerical investigation} In \cref{sec:antiplane} a numerical verification of the generalized Griffith criterion (\cref{prop:Ggriffithlaw}) and the asymptotic Griffith's law (\cref{prop:whenellpetit}) is performed for an antiplane tearing experiment. The conventional dynamic energy release rate \eqref{eq:GtC} is numerically computed and verified as a tool to translate gradient damage mechanics results in fracture mechanics terminology. The crack length evolution predicted by the dynamic gradient damage model agrees well with the Griffith's theory applied to the 1-d peeling problem.

The quasi-static limit of the dynamic gradient damage model is also investigated. Convergence of the dynamic model toward the quasi-static one should be observed in the absence of unstable crack propagation. Otherwise a full dynamic analysis should be preferred.

\subsection{Kinking}
Numerical investigation of crack kinking is investigated in \cref{sec:kinking}. It is verified that the kinking angle for an initially stationary crack predicted by the dynamic gradient damage model corresponds to several commonly used kinking criteria in quasi-static fracture mechanics, see \cref{fig:kinking_comp}. \future{Toughness anisotropy may be needed to discriminate between them}. \future{Future work could be devoted to a full dynamic path analysis for a crack that propagates initially at a velocity}, since it is not sure that the crack will kink or branch.

According to \cite{ChambolleFrancfortMarigo:2009}, kinking is always accompanied by a temporal brutal or unstable crack propagation. Following the work in \cref{sec:antiplane}, we could expect that the \future{crack length at arrest is different for the first-order quasi-static model} (\cref{def:firstorderqs}) \future{and the dynamic model} (\cref{def:dynagraddama}) \future{after the kink}. \future{This unstable propagation in the presence of a kink needs further investigation and a parametric study on the ratio $K_2/K_1$ could be performed}.

\subsection{Branching}
In \cref{sec:branching}, some physical insights into the branching mechanism predicted by the dynamic gradient damage model are provided. It is observed that in dynamics the damage field perpendicular to the gradient-damage crack may not correspond always to the optimal damage profile \eqref{eq:defoptimaldamage} which defines an equivalent fracture toughness \eqref{eq:gcingd} in gradient damage models. On the contrary, widening of the damage band takes place whenever a critical speed is reached. From a macroscopic modeling point of view, this may corresponds to the nucleation and subsequent interaction of several micro-branches along the main crack. Additional energy dissipation is thus reproduced. According to our numerical simulations, this critical speed is estimated to be $v_\mc\approx0.4c_\mathrm{R}$, which agrees well with experimental results for brittle materials.

A space-time zoom at the micro-branching event is performed. The stress distribution satisfies a generalized Yoffe-type criterion. Indeed, it is observed that the hoop stress variation is no longer maximal in front of the crack for some distance $r$, when the velocity of the main crack reaches $v_\mc$. A smaller value of $v_\mc$ compared to the original Yoffe criterion (see \cref{fig:yoffe}) can also be attributed to the presence of a mode-II perturbation. This also agrees with some experimental findings, see \cite{BoueCohenFineberg:2015}.

\section{Experimental Validation} \label{sec:valicon}
The dynamic gradient damage model as well as its current numerical implementation in the EPX software are applied to several real-world dynamic fracture problems for experimental validation. The thematic subjects covered are summarized in \cref{tab:summconexp}.
\begin{table}[htbp]
\centering
\caption{Thematic subjects covered in this section} \label{tab:summconexp}
\begin{tabular}{ccccc} \toprule
& Going dynamical & $\alpha\leftrightarrow\phi$ & $\nabla\alpha\to\Gamma$ & Experimental validation \\ \midrule
Theoretics & & & & \\
Numerics & & & & \rightthumbsup \\ \bottomrule
\end{tabular}
\end{table}

\paragraph{Application to materials possessing a small damage band} In \cref{sec:gregoire}, the dynamic gradient damage model is used to investigate crack evolution inside a pre-cracked PMMA plate. Using standard values of the critical stress, the use of the \eqref{eq:at1} model leads to an internal length much smaller than 1 millimeter. This leads to a relatively fine mesh and hence a computationally demanding simulation for a normal mechanical or industrial specimen or structure. However since the internal length is small compared to the dimension of the body, a separation between the damage process zone and the outer linear elastic fracture problem can be achieved. Diffuse damage does not takes place and the gradient-damage approach is expected to give an appropriate modeling of brittle fracture. For this crack arrest problem, a quantitative comparison between the gradient-damage prediction and the experimental observation is performed in terms of the crack tip evolution. A good agreement is found between them.

\paragraph{Application to concrete} For concrete, the damage band corresponds to several centimeters. In the Brazilian splitting test in \cref{sec:brazilian}, this does not constitute a problem since the crack path is \emph{a priori} determined due to a particular stress distribution. The crack nucleates at the center of the cylinder, which then propagates along the vertical diameter direction. The global temporal evolution of the applied load agrees with the elasticity prediction. Similarly to experimental observations, vertical splitting of the concrete cylinder is reproduced.

For the L-shaped concrete specimen in \cref{sec:L-specimen}, a relatively large internal length begins to raise difficulties when crack path prediction is the main objective of the numerical simulation. Due to a large damage zone, not only the crack is extremely diffused in the structure leading to a difficult identification of the crack path, the gradient-damage modeling of fracture itself could also be questionable. The strain-softening region spreads to a large subset of the body, the material behavior can no longer be considered to be brittle. Hence, the crack path prediction is less satisfactory.

A major reason behind it lies in the simplicity of the \eqref{eq:at1} model, since no additional parameters are introduced. By using more sophisticated damage constitutive laws such as that proposed by \cite{LorentzGodard:2011}, the damage band itself can be considered as a modeling parameter. \future{Future work could be devoted to the investigation of a better damage constitutive law for the concrete material}.

Nevertheless, if a small internal length is assumed in \cref{sec:L-specimen}, more or less satisfactory simulation results are obtained. Dynamical effects are then investigated by varying the loading speed. On the one hand, crack branching is reproduced which corresponds to the velocity effects observed in brittle materials, cf. \cite{Schardin:2012}. On the other hand, the peak load also increases with the external loading speed. Since no material strain-rate effects are introduced (yet), this corresponds to pure inertia effects as concluded by experimental observations of \cite{OzboltBedeSharmaMayer:2015}.

Finally, the gradient damage model is applied to the CEA impact test on concrete beams. Apart from a relatively large internal length with respect to the height direction, it appears that plastic effects and rate-dependency should be introduced in the material constitutive modeling to account for the complex behaviors of concrete under compression. \future{Future work could be devoted to the formulation and analysis of a better constitutive modeling of concrete coupled with gradient damage approaches}.
