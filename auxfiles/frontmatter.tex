\chapter{Avant-propos}
Simulating dynamic fracture phenomena via continuum material constitutive modeling

\label{chap:basicassummtions}
\begin{itemize}
\item Constitutive mechanical modeling based on the theory of continuum mechanics.
\item Macroscopic modeling without microscopic justification
\end{itemize}

General notation conventions adopted in this thesis are summarized as follows. Scalar-valued quantities will be denoted by italic roman or greek letters like the crack length $t\mapsto l_t$ or the damage field $\alpha$. Vector-valued (in $\mathbb{R}^2$ in particular) quantities will be represented by boldface letters such as the displacement field $\vec{u}$. Second or higher order tensors considered as linear operators will be indicated by sans-serif letters: the elasticity tensor $\tens{A}$ for instance. Intrinsic notation is adopted and their contraction on lower-order tensors will be written without dots $\tens{A}\eps=\tens{A}_{ijkl}\eps_{kl}$. Inner products between two vectors or tensors of the same order will be denoted with a dot, such as $\tens{A}\eps\cdot\eps=\tens{A}_{ijkl}\eps_{kl}\eps_{ij}$. The spatial integration measures will be generally omitted since the usual Lebesgue or Hausdorff measures will be used for integration on the domain $\Omega$ or its boundary $\partial\Omega$. Time dependence will be noted at the subscripts of the involved quantities, as $\vec{u}:(t,\vec{x})\mapsto\vec{u}_t(\vec{x})$.