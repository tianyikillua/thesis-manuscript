%% General packages
\usepackage[utf8]{inputenc}
\usepackage[english]{babel}
\usepackage{caption}
\usepackage[top=2.5cm,bottom=2cm,left=3cm,right=2.5cm,headheight=15pt]{geometry}
\usepackage{fancyhdr}
\usepackage{graphicx}
\usepackage{enumerate}
\usepackage{booktabs}
\usepackage{xspace}
\usepackage{xcolor}
\usepackage[pagebackref]{hyperref}
\usepackage{quotchap}
\usepackage[nottoc,notindex,notlot,notlof]{tocbibind}
\usepackage{newtxtext}
\usepackage{multicol}
\usepackage{minitoc}
\usepackage{makeidx}
\usepackage{silence}
\usepackage{microtype}
\usepackage[framemethod=tikz]{mdframed}
\usepackage{dingbat}
\usepackage[square]{natbib}
\usepackage{bibentry}

\let\openbox\relax
\usepackage{amsmath}
\usepackage{amssymb,amscd}
\usepackage{amsthm,thmtools}
\usepackage{algpseudocode}
\usepackage{algorithm}
\usepackage{siunitx}
\usepackage{braket}
\usepackage{newtxmath}
\usepackage[all,pdf]{xy}
\usepackage[capitalise,noabbrev]{cleveref}

%% Settings
\setcounter{minitocdepth}{2}
\WarningFilter{minitoc(hints)}{W0023}
\WarningFilter{minitoc(hints)}{W0024}
\WarningFilter{minitoc(hints)}{W0028}
\WarningFilter{minitoc(hints)}{W0030}

\hypersetup{unicode=true,colorlinks=true,linkcolor=blue,citecolor=blue,urlcolor=blue,breaklinks=true}
\newcommand{\print}{\hypersetup{linkcolor=black,citecolor=black,urlcolor=black}}
\renewcommand*{\backreflastsep}{ and }
\renewcommand*{\backreftwosep}{ and }
\renewcommand*{\backref}[1]{}
\renewcommand*{\backrefalt}[4]{\ifcase #1 \relax
\or Cited on page #2.
\else Cited on pages #2.
\fi}

\renewcommand{\cite}{\citep}
\setlength{\bibsep}{0pt plus 0.3ex}

\fancyhf{}
\fancyhead[LE,RO]{\thepage}
\fancyhead[RE]{\nouppercase{\leftmark}}
\fancyhead[LO]{\nouppercase{\rightmark}}
\captionsetup[table]{skip=10pt}
\renewcommand*{\sectfont}{\bfseries}

\graphicspath{{../thesis/Papers/figs/model/}{../thesis/Papers/figs/kalthoff/}{../thesis/Papers/figs/plate/}{../thesis/Papers/figs/antiplane/}{../thesis/Papers/figs/brazilian/}{../thesis/Papers/figs/num/}{../thesis/Papers/figs/TC/}{../thesis/Papers/figs/others/}{../thesis/Papers/figs/gregoire/}{../thesis/Papers/figs/L-specimen/}{../thesis/Papers/figs/branching/}{../thesis/Papers/figs/kinking/}{../thesis/Papers/figs/bar/}}

\makeindex

\declaretheorem[within=chapter]{theorem}
\declaretheorem[style=plain]{proposition}
\declaretheorem[sibling=theorem,style=definition]{definition}
\declaretheorem[sibling=theorem,style=definition]{hypothesis}
\declaretheorem[style=remark,numbered=no]{remark}

\makeatletter
\newcommand{\pushright}[1]{\ifmeasuring@#1\else\omit\hfill$\displaystyle#1$\fi\ignorespaces}
\makeatother

%% Definitions
\def\vec#1{\ensuremath{\mathchoice
{\mbox{\boldmath$\displaystyle\mathbf{#1}$}}
{\mbox{\boldmath$\textstyle\mathbf{#1}$}}
{\mbox{\boldmath$\scriptstyle\mathbf{#1}$}}
{\mbox{\boldmath$\scriptscriptstyle\mathbf{#1}$}}}}
\def\tens#1{\ensuremath{\mathsf{#1}}}

\newcommand{\D}[1]{\,\mathrm{d}#1}
\renewcommand{\set}[1]{\Set{#1}}
\newcommand{\tr}{\operatorname{tr}}
\newcommand{\eps}{\vec{\varepsilon}}
\newcommand{\sig}{\vec{\sigma}}
\newcommand{\mT}{\mathsf{T}}
\newcommand{\mI}{\mathbb{I}}
\newcommand{\mc}{\mathrm{c}}
\newcommand{\me}{\mathrm{e}}
\newcommand{\md}{\mathrm{d}}
\newcommand{\mH}{\operatorname{H}}
\newcommand{\inp}[1]{\Braket{#1}}
\newcommand{\abs}[1]{\left\lvert{#1}\right\rvert}
\newcommand{\norm}[1]{\left\lVert{#1}\right\rVert}
\renewcommand{\tens}[1]{\mathsf{#1}}
\newcommand{\gc}{G_\mathrm{c}}
\renewcommand{\div}{\operatorname{div}}
\newcommand{\dev}{\operatorname{dev}}
\newcommand{\vtheta}{\vec{\theta}}
\newcommand{\vtau}{\vec{\tau}}
\newcommand{\vphi}{\vec{\varphi}}
\newcommand{\RN}[1]{\textup{\uppercase\expandafter{\romannumeral#1}}}
\newcommand{\symgrad}{\nabla^\mathrm{s}}
\newcommand{\uvec}{\underline{\vec{u}}}
\newcommand{\dvec}{\underline{\vec{\alpha}}}
\newcommand{\domaint}{{\Omega\setminus\Gamma_t}}
\newcommand{\domaini}{{\Omega\setminus\Gamma_0}}
\newcommand{\philt}{\vec{\phi}}
\newcommand{\ovtheta}{\overline{\vtheta}}
\newcommand{\dx}{\D{\vec{x}}}
\newcommand{\ds}{\D{\vec{s}}}
\newcommand{\dxx}{\D{\vec{x}^*}}
\newcommand{\uhat}{\widehat{u}}
\newcommand{\ahat}{\widehat{\alpha}}
\newcommand{\xhat}{\widehat{x}}
\newcommand{\that}{\widehat{t}}