%% General packages
\usepackage[utf8]{inputenc}
\usepackage[american]{babel}
\usepackage{hyperref}
\usepackage{caption}
\usepackage[top=2.5cm,bottom=2cm,left=3cm,right=2.5cm,headheight=15pt]{geometry}
\usepackage{fancyhdr}
\usepackage{graphicx}
\usepackage{enumerate}
\usepackage{booktabs}
\usepackage{xspace}
\usepackage{xcolor}
\usepackage{quotchap}
\usepackage[nottoc,notindex,notlot,notlof]{tocbibind}
\usepackage{newtxtext}
\usepackage{multicol}
\usepackage{minitoc}
\usepackage{makeidx}
\usepackage{silence}
\usepackage{microtype}
\usepackage[framemethod=tikz]{mdframed}
\usepackage{dingbat}
\usepackage{csquotes}
\usepackage[style=apa,backend=biber,apabackref=true,
            hyperref=true,doi=false,abbreviate=false,
            uniquename=false,eprint=false]{biblatex}
\usepackage[normalem]{ulem}

\let\openbox\relax
\usepackage{amsmath}
\usepackage{amssymb,amscd}
\usepackage{amsthm,thmtools}
\usepackage{algpseudocode}
\usepackage{algorithm}
\usepackage{siunitx}
\usepackage{braket}
\usepackage{newtxmath}
\usepackage[all,pdf]{xy}
\usepackage[capitalise,noabbrev]{cleveref}

\declaretheorem[within=chapter]{theorem}
\declaretheorem[style=plain]{proposition}
\declaretheorem[sibling=theorem,style=definition]{definition}
\declaretheorem[sibling=theorem,style=definition]{hypothesis}
\declaretheorem[style=remark,numbered=no]{remark}

\setcounter{minitocdepth}{2}

\hypersetup{unicode=true,colorlinks=true,linkcolor=blue,
            citecolor=blue,urlcolor=blue,breaklinks=true}
\newcommand{\print}{\hypersetup{linkcolor=black,citecolor=black,urlcolor=black}}
\DeclareLanguageMapping{american}{american-apa}
\DeclareListFormat{pageref}{%
\ifthenelse{\value{listcount}=1}%
{\hyperpage{#1}}%
{\ifthenelse{\value{listcount}<\value{liststop}}%
{\addcomma\addspace\hyperpage{#1}}%
{\ifnumequal{\value{listcount}}{\value{liststop}}%
{\addspace{}and\addspace\hyperpage{#1}}{}}}}

\renewbibmacro{in:}{%
\ifentrytype{article}{}{%
\printtext{\bibstring{in}\intitlepunct}}}

\AtEveryBibitem{
\clearfield{month}{}
\clearlist{language}{}}

\setlength{\bibitemsep}{0pt plus 0.3ex}
\renewcommand*{\bibfont}{\small}

\makeatletter
\let\abx@macro@citeOrig\abx@macro@cite
\renewbibmacro{cite}{%
\bibhyperref{%
\let\bibhyperref\relax\relax%
\abx@macro@citeOrig}}
\let\abx@macro@parenciteOrig\abx@macro@parencite
\newbibmacro{parencite}{%
\bibhyperref{%
\let\bibhyperref\relax\relax%
\abx@macro@parenciteOrig}}
\makeatother
\renewcommand{\cite}{\parencite}

\WarningFilter{minitoc(hints)}{W0010}
\WarningFilter{minitoc(hints)}{W0023}
\WarningFilter{minitoc(hints)}{W0024}
\WarningFilter{minitoc(hints)}{W0028}
\WarningFilter{minitoc(hints)}{W0030}

\fancyhf{}
\fancyhead[LE,RO]{\thepage}
\fancyhead[RE]{\nouppercase{\leftmark}}
\fancyhead[LO]{\nouppercase{\rightmark}}
\captionsetup[table]{skip=10pt}
\renewcommand*{\sectfont}{\bfseries}

\newcommand{\future}{\dashuline}

\graphicspath{{../thesis/Papers/figs/model/}{../thesis/Papers/figs/kalthoff/}{../thesis/Papers/figs/plate/}{../thesis/Papers/figs/antiplane/}{../thesis/Papers/figs/brazilian/}{../thesis/Papers/figs/num/}{../thesis/Papers/figs/TC/}{../thesis/Papers/figs/others/}{../thesis/Papers/figs/gregoire/}{../thesis/Papers/figs/L-specimen/}{../thesis/Papers/figs/branching/}{../thesis/Papers/figs/kinking/}{../thesis/Papers/figs/bar/}{../thesis/Papers/figs/beam/}}

\makeatletter
\newcommand{\pushright}[1]{\ifmeasuring@#1\else\omit\hfill$\displaystyle#1$\fi\ignorespaces}
\makeatother

\makeindex

%% Definitions
\def\vec#1{\ensuremath{\mathchoice
{\mbox{\boldmath$\displaystyle\mathbf{#1}$}}
{\mbox{\boldmath$\textstyle\mathbf{#1}$}}
{\mbox{\boldmath$\scriptstyle\mathbf{#1}$}}
{\mbox{\boldmath$\scriptscriptstyle\mathbf{#1}$}}}}
\def\tens#1{\ensuremath{\mathsf{#1}}}

\newcommand{\D}[1]{\,\mathrm{d}#1}
\renewcommand{\set}[1]{\Set{#1}}
\newcommand{\tr}{\operatorname{tr}}
\newcommand{\eps}{\vec{\varepsilon}}
\newcommand{\sig}{\vec{\sigma}}
\newcommand{\mT}{\mathsf{T}}
\newcommand{\mI}{\mathbb{I}}
\newcommand{\mc}{\mathrm{c}}
\newcommand{\me}{\mathrm{e}}
\newcommand{\md}{\mathrm{d}}
\newcommand{\mH}{\operatorname{H}}
\newcommand{\inp}[1]{\Braket{#1}}
\newcommand{\abs}[1]{\left\lvert{#1}\right\rvert}
\newcommand{\norm}[1]{\left\lVert{#1}\right\rVert}
\renewcommand{\tens}[1]{\mathsf{#1}}
\newcommand{\gc}{G_\mathrm{c}}
\renewcommand{\div}{\operatorname{div}}
\newcommand{\dev}{\operatorname{dev}}
\newcommand{\vtheta}{\vec{\theta}}
\newcommand{\vtau}{\vec{\tau}}
\newcommand{\vphi}{\vec{\varphi}}
\newcommand{\RNN}[1]{\textup{\uppercase\expandafter{\romannumeral#1}}}
\newcommand{\symgrad}{\nabla^\mathrm{s}}
\newcommand{\uvec}{\underline{\vec{u}}}
\newcommand{\dvec}{\underline{\vec{\alpha}}}
\newcommand{\domaint}{{\Omega\setminus\Gamma_t}}
\newcommand{\domaini}{{\Omega\setminus\Gamma_0}}
\newcommand{\philt}{\vec{\phi}}
\newcommand{\ovtheta}{\overline{\vtheta}}
\newcommand{\dx}{\D{\vec{x}}}
\newcommand{\ds}{\D{\vec{s}}}
\newcommand{\dxx}{\D{\vec{x}^*}}
\newcommand{\uhat}{\widehat{u}}
\newcommand{\ahat}{\widehat{\alpha}}
\newcommand{\xhat}{\widehat{x}}
\newcommand{\that}{\widehat{t}}